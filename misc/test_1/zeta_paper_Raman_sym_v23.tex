
   
%%%%%%%%%%%%%%%%%%%%%%%%%%%%%%%%%%%%%%%%%%%%%%%%%%%%%%%%%%%%%%

%\documentclass[onecolumn,preprintnumbers,amsmath,amssymb]{revtex4}
%\documentclass[onecolumn,preprintnumbers,amsmath,amssymb]{elsarticle}
%\documentclass[preprint,10pt]{elsarticle}

%\usepackage{amssymb}
%\usepackage{graphicx}% Include figure files
%\usepackage{dcolumn}% Align table columns on decimal point
%\usepackage{bm}% bold math



%% 
%% Copyright 2007, 2008, 2009 Elsevier Ltd
%% 
%% This file is part of the 'Elsarticle Bundle'.
%% ---------------------------------------------
%% 
%% It may be distributed under the conditions of the LaTeX Project Public
%% License, either version 1.2 of this license or (at your option) any
%% later version.  The latest version of this license is in
%%    http://www.latex-project.org/lppl.txt
%% and version 1.2 or later is part of all distributions of LaTeX
%% version 1999/12/01 or later.
%% 
%% The list of all files belonging to the 'Elsarticle Bundle' is
%% given in the file `manifest.txt'.
%% 

%% Template article for Elsevier's document class `elsarticle'
%% with numbered style bibliographic references
%% SP 2008/03/01
%%
%% 
%%
%% $Id: elsarticle.cls,v 1.20 2008-10-13 04:24:12 cvr Exp $
%%
%%

%\documentclass[preprint, review,12pt]{elsarticle}

%\documentclass[preprint, review,10pt]{elsarticle}

%\documentclass[final,3p]{elsarticle}
\documentclass[11pt]{elsarticle}


\usepackage{amssymb}
\usepackage{amsmath}


\usepackage{rotating}
\usepackage{hyperref}

\usepackage{lineno}
%\linenumbers

\setcounter{tocdepth}{4}
%\UseRawInputEncoding



%\usepackage[margin=1cm,nohead]{geometry}
\usepackage[margin=1cm,nohead]{geometry} %27 pages
%\usepackage[margin=2cm,nohead]{geometry} %29 pages
%\usepackage[margin=1in,nohead]{geometry} %31 pages

\setlength{\footskip}{10pt}



%% The amsthm package provides extended theorem environments
%% \usepackage{amsthm}

%% The lineno packages adds line numbers. Start line numbering with
%% \begin{linenumbers}, end it with \end{linenumbers}. Or switch it on
%% for the whole article with \linenumbers after \end{frontmatter}.
%% \usepackage{lineno}

%% natbib.sty is loaded by default. However, natbib options can be
%% provided with \biboptions{...} command. Following options are
%% valid:

%%   round  -  round parentheses are used (default)
%%   square -  square brackets are used   [option]
%%   curly  -  curly braces are used      {option}
%%   angle  -  angle brackets are used    <option>
%%   semicolon  -  multiple citations separated by semi-colon 
%%   colon  - same as semicolon, an earlier confusion
%%   comma  -  separated by comma
%%   numbers-  selects numerical citations
%%   super  -  numerical citations as superscripts
%%   sort   -  sorts multiple citations according to order in ref. list
%%   sort&compress   -  like sort, but also compresses numerical citations
%%   compress - compresses without sorting
%%
%% \biboptions{comma,round}

% \biboptions{}

%\nofiles

%\journal{Journal of Number Theory}
\journal{Zenodo}


\begin{document}

%\pagestyle{plain}
%\renewcommand\headrulewidth{0pt}

\begin{frontmatter}

%\preprint{}

\title{On a new method towards proof of Riemann's Hypothesis}% Force line breaks with \\


\author{Akhila Raman }

\address{University of California at Berkeley. Email: akhila.raman@berkeley.edu. }

%\affiliation{University of California at Berkeley, CA-94720. Email: akhila.raman@berkeley.edu. Ph: 510-540-5544}


%\date{\today}% It is always \today, today,
             %  but any date may be explicitly specified

\begin{abstract}
%\textbf{Abstract}\\

We consider the analytic continuation of Riemann's Zeta Function derived from \textbf{Riemann's Xi function} $\xi(s)$ which is evaluated at $s = \frac{1}{2} + \sigma + i \omega$, given by $\xi(\frac{1}{2} + \sigma + i \omega)= E_{p\omega}(\omega)$, where $\sigma, \omega$ are real and $-\infty \leq \omega \leq \infty$ and compute its inverse Fourier transform given by $E_p(t)$.\\

We use a new method and show that the Fourier Transform of $E_p(t)$ given by $E_{p\omega}(\omega) = \xi(\frac{1}{2} + \sigma + i \omega)$ \textbf{does not have zeros} for finite and real $\omega$ when $0 < |\sigma| < \frac{1}{2}$, corresponding to the critical strip \textbf{excluding} the critical line and prove Riemann's hypothesis.\\

More importantly, the new method \textbf{does not} contradict the existence of non-trivial zeros on the critical line with real part of $s = \frac{1}{2}$ and \textbf{does not} contradict Riemann Hypothesis. It is shown that the new method is \textbf{not} applicable to Hurwitz zeta function and related functions and  \textbf{does not} contradict the existence of their non-trivial zeros away from the critical line. \\

If the specific solution presented in this paper is incorrect, it is \textbf{hoped} that the new method discussed in this paper will lead to a correct solution by other researchers.
\end{abstract}

\begin{keyword}
Riemann \sep Hypothesis \sep Zeta \sep Xi \sep exponential functions

%% PACS codes here, in the form: \PACS code \sep code

%% MSC codes here, in the form: \MSC code \sep code
%% or \MSC[2008] code \sep code (2000 is the default)

\end{keyword}

\end{frontmatter}
   
                              %display desired
%\maketitle

%% main text

%\section{}
%\label{Introduction}

%% The Appendices part is started with the command \appendix;
%% appendix sections are then done as normal sections
%% \appendix

%% \section{}
%% \label{}

%% References
%%
%% Following citation commands can be used in the body text:
%% Usage of \cite is as follows:
%%   \cite{key}         ==>>  [#]
%%   \cite[chap. 2]{key} ==>> [#, chap. 2]
%% 



\section{\label{sec:level2} \textbf{Introduction} \protect\\  \lowercase{} }

It is well known that Riemann's Zeta function given by $\zeta(s) = \sum\limits_{m=1}^{\infty} \frac{1}{m^{s}}$ converges in the half-plane where the real part of $s$ is greater than 1. Riemann proved that  $\zeta(s)$  has an analytic continuation to the whole s-plane apart from a simple pole at $s = 1$ and that $\zeta(s)$  satisfies a symmetric functional equation given by $\xi(s) = \xi(1-s) =  \frac{1}{2} s (s-1) \pi^{-\frac{s}{2}} \Gamma(\frac{s}{2}) \zeta(s) $ where $\Gamma(s) = \int_{0}^{\infty} e^{-u} u^{s-1} du$ is the Gamma function.$^{\citep{FWE}}$ $^{\citep{JBC}}$ We can see that if Riemann's Xi function has a zero in the critical strip, then Riemann's Zeta function also has a zero at the same location. Riemann made his conjecture in his 1859 paper, that all of the non-trivial zeros of $\zeta(s)$ lie on the critical line with real part of $s = \frac{1}{2}$, which is called the Riemann Hypothesis.$^{\citep{BR}}$ \\

Hardy and Littlewood later proved that infinitely many of the zeros of $\zeta(s)$ are on the critical line with real part of $s=\frac{1}{2}$.$^{\citep{GH}}$ It is well known that $\zeta(s)$ does not have non-trivial zeros when real part of $s = \frac{1}{2} + \sigma + i \omega$, given by $\frac{1}{2} + \sigma \geq 1$ and $\frac{1}{2} + \sigma \leq 0$. In this paper, \textbf{critical strip} $0 < Re[s] < 1 $ corresponds to $0 \leq |\sigma| < \frac{1}{2}$.\\


In this paper, a \textbf{new method} is discussed and a specific solution is presented to prove Riemann's Hypothesis. If the specific solution presented in this paper is incorrect, it is \textbf{hoped} that the new method discussed in this paper will lead to a correct solution by other researchers.\\



In Section~\ref{sec:Section_2}, we prove Riemann's hypothesis by taking the analytic continuation of Riemann's Zeta Function derived from Riemann's Xi function  $\xi(\frac{1}{2} + \sigma + i \omega)= E_{p\omega}(\omega)$ and compute inverse Fourier transform of $E_{p\omega}(\omega)$ given by $E_p(t)$ and show that its Fourier transform $E_{p\omega}(\omega)$ does not have zeros for finite and real $\omega$ when $0 < |\sigma| < \frac{1}{2}$, corresponding to the critical strip \textbf{excluding} the critical line.\\
% evaluated at $s = \frac{1}{2} + \sigma + i \omega$, given by





In ~\ref{sec:appendix_A} to ~\ref{sec:Appendix_D_5}, well known results which are used in this paper are re-derived. \\

We present an \textbf{outline} of the new method below.
% and a short \textbf{video} presentation in this \href{https://www.ocf.berkeley.edu/~araman/files/math_z/Z_clip_full_3.mp4}{link}.

 
%\clearpage

\subsection{\label{sec:Section_1_1} \textbf{Step 1: Inverse Fourier Transform of $\xi(\frac{1}{2} + i \omega)$ } \protect\\  \lowercase{} }


Let us start with Riemann's Xi Function $\xi(s)$ evaluated at $s = \frac{1}{2} + i \omega$ given by $\xi(\frac{1}{2} + i \omega)= \Xi(\omega) =  E_{0\omega}(\omega)$, where $-\infty \leq \omega \leq \infty$. Its inverse Fourier Transform is given by $ E_0(t)= \frac{1}{2 \pi}  \int_{-\infty}^{\infty} E_{0\omega}(\omega) e^{i\omega t} d\omega $, where $\omega, t$ are real, as follows \href{https://www.ams.org/notices/200303/fea-conrey-web.pdf#page=5}{(link)}.$^{\citep{ECT}}$ This is re-derived in ~\ref{sec:appendix_H}.
%

\begin{equation}\label{sec_intro_eq_1}
E_{0}(t) = \Phi(t) = 2 \sum_{n=1}^{\infty}  [ 2 n^{4}  \pi^{2} e^{\frac{9t}{2}}    - 3 n^{2} \pi  e^{\frac{5t}{2}} ]  e^{- \pi n^{2} e^{2t}} =    2 \sum_{n=1}^{\infty}  [ 2 \pi^{2} n^{4} e^{4t}    - 3 \pi n^{2}   e^{2t} ]  e^{- \pi n^{2} e^{2t}} e^{\frac{t}{2}}   
\end{equation}

We see that $E_{0}(t)=E_{0}(-t)$ is a real and \textbf{even} function of $t$, given that  $E_{0\omega}(\omega) = E_{0\omega}(-\omega)$ because $\xi(s)=\xi(1-s)$ and hence $\xi(\frac{1}{2} + i \omega)=\xi(\frac{1}{2} - i \omega)$ when evaluated at $s = \frac{1}{2} + i \omega$.  \\

The inverse Fourier Transform of   $\xi(\frac{1}{2}+ \sigma + i \omega) = E_{p\omega}(\omega)$ is given by the real function $E_p(t)$.  We can write $E_p(t)$ as follows for $0 < |\sigma| < \frac{1}{2}$ and this is shown in detail in ~\ref{sec:appendix_A} using contour integration.

\begin{equation}\label{sec_intro_eq_2}
E_{p}(t) = E_{0}(t) e^{-\sigma t} =   2 \sum_{n=1}^{\infty}  [ 2 \pi^{2} n^{4} e^{4t}    - 3 \pi n^{2}   e^{2t} ]  e^{- \pi n^{2} e^{2t}} e^{\frac{t}{2}} e^{-\sigma t} 
\end{equation}

We can see that $E_{p}(t)$ is an analytic function in the interval $|t| \leq \infty$, given that the sum and product of exponential functions are analytic in the same interval and hence infinitely differentiable in that interval.


\subsection{\label{sec:Section_1_2} \textbf{Step 2: Taylor's series representation of $E_p(t)$} \protect\\  \lowercase{} }

We can substitute $z= e^{2t}$ in Eq.~\ref{sec_intro_eq_2} as follows. 


\begin{eqnarray*}\label{app_B1_eq_1}  
E_{p}(t) =    2 \sum_{n=1}^{\infty}  [ 2 \pi^{2} n^{4} e^{4t}    - 3 \pi n^{2}   e^{2t} ]  e^{- \pi n^{2} e^{2t}} e^{\frac{t}{2}} e^{-\sigma t} = E_{0}(t) e^{-\sigma t} = f(e^{2t}) e^{\frac{t}{2}} e^{-\sigma t} \\ 
f(z)= 2 \sum_{n=1}^{\infty}  [ 2 \pi^{2} n^{4} z^{2}    - 3 \pi n^{2}   z ]  e^{- \pi n^{2} z}   
\end{eqnarray*}
\begin{equation} \end{equation}

We can expand the real analytic function $f(z)$ using Taylor series expansion \textbf{around} $z=1$ as follows.

\begin{eqnarray*}\label{app_B1_eq_2}  
f(z)=  \sum_{n=1}^{\infty}  a_n z^2 [ \displaystyle\sum\limits_{k=0}^{\infty}  d_{nk}  (z-1)^{k}] -  b_n z [\displaystyle\sum\limits_{k=0}^{\infty} d_{nk} (z-1)^{k}]  \\
a_{n} = 4 \pi^{2} n^{4} e^{- \pi n^{2}}, \quad  b_{n} =    6 \pi n^{2} e^{- \pi n^{2}}, \quad  d_{nk}= \frac{(- \pi n^{2})^{k}}{ !(k)}  
\end{eqnarray*}
\begin{equation} \end{equation}


Now we substitute $z= e^{2t}$ in Eq.~\ref{app_B1_eq_2} and we can write the Taylor series expansion of $E_p(t)$ as follows and we use binomial series expansion  $(e^{2t}-1)^v =  \displaystyle\sum\limits_{p=0}^{v} \binom{v}{p} (-1)^p e^{2t(v-p)}$ for $v$ is a positive integer. 

\begin{eqnarray*}\label{app_B1_eq_3}   
E_p(t)= [ \sum_{n=1}^{\infty}  a_n e^{4t} [ \displaystyle\sum\limits_{k=0}^{\infty}  d_{nk}  (e^{2t}-1)^{k}] -  b_n e^{2t} [\displaystyle\sum\limits_{k=0}^{\infty} d_{nk} (e^{2t}-1)^{k}]  ]  e^{\frac{t}{2}}  e^{-\sigma t} \\
E_p(t)= [  \sum_{n=1}^{\infty}   a_{n} \displaystyle\sum\limits_{k=0}^{\infty}  d_{nk} [ \displaystyle\sum\limits_{p=0}^{k} \binom{k}{p} (-1)^p e^{2t(k+2-p)}] - b_{n} \displaystyle\sum\limits_{k=0}^{\infty}  d_{nk} [ \displaystyle\sum\limits_{p=0}^{k} \binom{k}{p} (-1)^p e^{2t(k+1-p)}]  ]  e^{\frac{t}{2}}  e^{-\sigma t} 
\end{eqnarray*}
\begin{equation} \end{equation}

This equation  can be simplified as follows, using shorthand notation.

\begin{eqnarray*}\label{app_B1_eq_4}   
E_p(t)=  \displaystyle\sum\limits_{n,k,r,p}  c_{nkrp} e^{b_{krp} t} e^{-\sigma t}\\
b_{krp} = (2k+\frac{5}{2}+2 r - 2p), \quad \displaystyle\sum\limits_{n,k,r,p}  c_{nkrp} =   \sum_{r=0}^{1}  \sum_{n=1}^{\infty}  e_{nr}  \displaystyle\sum\limits_{k=0}^{\infty}  d_{nk} \displaystyle\sum\limits_{p=0}^{k} \binom{k}{p} (-1)^p, \quad e_{n1} = a_{n}, \quad e_{n0} =  - b_{n} ,
\end{eqnarray*}
\begin{equation} \end{equation}


In Section~\ref{sec:Section_1_1}, we showed that $E_0(t)=E_0(-t)$ and we can write $E_p(t) = E_{0}(t) e^{-\sigma t} $ as an\textbf{ infinite summation} of two-sided decaying exponential functions as follows.

\begin{eqnarray*}\label{app_B1_eq_2}   
E_p(t)=   [ \displaystyle\sum\limits_{n,k,r,p}  c_{nkrp} e^{b_{krp} t} u(-t) + \displaystyle\sum\limits_{n,k,r,p}  c_{nkrp} e^{-b_{krp} t} u(t)  ] e^{-\sigma t}
\end{eqnarray*}
\begin{equation} \end{equation}


\subsection{\label{sec:Section_1_3} \textbf{Step 3: Two-sided decaying exponentials and zeros in their Fourier transform} \protect\\  \lowercase{} }


We know that a real two-sided decaying exponential function $g_0(t) = e^{b t} u(-t) +  e^{-a t} u(t) $, where $u(t)$ is Heaviside unit step function and $a, b > 0$ and $t$ are real, has Fourier Transform $G_0(\omega)$, where $\omega$ is real. \href{https://web.stanford.edu/class/ee102/lectures/fourtran#page=6}{(link)}


\begin{eqnarray*}\label{sec_0_1_eq_1_1}  
G_0(\omega) = \int_{-\infty}^{\infty} g_0(t) e^{-i \omega t} dt =  \frac{1}{b - i \omega} + \frac{1}{a + i \omega} = \frac{b + i \omega}{b^{2} +  \omega^{2} } + \frac{a - i \omega}{a^{2} +  \omega^{2}} \\
= [ \frac{b}{b^{2} +  \omega^{2} } + \frac{a }{a^{2} +  \omega^{2}} ]  +  i \omega [  \frac{1}{b^{2} +  \omega^{2}} - \frac{1}{a^{2} +  \omega^{2} } ]
\end{eqnarray*}
\begin{equation} \end{equation}

We can see that the real part of $G_0(\omega)$ given by $\frac{b}{b^{2} +  \omega^{2} } + \frac{a }{a^{2} +  \omega^{2}} $ \textbf{does not have zeros} for any  finite real value of $\omega$ and hence $G_0(\omega)$ does not have zeros for any finite value of $\omega$. \\

Given that the inverse Fourier Transform of Riemann Xi function  $\xi(\frac{1}{2}+ \sigma + i \omega)=E_{p\omega}(\omega)$ given by $E_p(t)$ is expressed as an \textbf{infinite summation of two-sided decaying exponential functions }in previous subsection, we could investigate if $E_{p\omega}(\omega)$ also does not have zeros for any  finite real value of $\omega$.





\subsection{\label{sec:level2} \textbf{Step 4: On the zeros of a related function $G(\omega)$ } \protect\\  \lowercase{} }



\textbf{Statement 1}: Let us assume that Riemann's Xi function $\xi(\frac{1}{2} + \sigma + i \omega)= E_{p\omega}(\omega)$ has a zero at $\omega = \omega_{0}$ where $\omega_{0}$ is real and finite and $0 < |\sigma| < \frac{1}{2}$, corresponding to the critical strip excluding the critical line.  We will prove that this assumption leads to a \textbf{contradiction}. \\


Let us consider $0 < \sigma < \frac{1}{2}$ at first. Let us consider \textbf{a toy example} with a new function  $g(t) = E_p(t) e^{-\sigma t}  u(-t) + E_p(t) e^{\sigma t}  u(t)  $ where $g(t)$ is a real function of variable $t$ and $u(t)$ is Heaviside unit step function. We can see that $g(t) h(t) = E_p(t)$ where $h(t)=[ e^{ \sigma t} u(-t) + e^{ -  \sigma t} u(t) ]$ . \\

In \textbf{~\ref{sec:Appendix_D_5}}, we will show that the Fourier transform of the \textbf{even function} $g_{even}(t)=\frac{1}{2} [g(t)+g(-t) ] $ given by $G_{even}(\omega) =  G_R(\omega)$  must have \textbf{at least one zero} at $\omega = \omega_{1} \neq 0$ to satisfy Statement 1, where $\omega_1$ is real and finite.\\


As an \textbf{example}, consider $E_p(t)= e^{b t} u(t) + e^{-a t} u(-t)$ where $a, b > \sigma >0$ are real and $h(t) = [ e^{\sigma t} u(-t) + e^{-\sigma t} u(t) ] $. We see that $g(t)= e^{ (b-\sigma) t} u(-t) + e^{ -(a - \sigma) t} u(t)$. The real part of Fourier transform of $g(t)$ is given by $G_R(\omega)=\frac{(b-\sigma)}{(b-\sigma)^{2} + \omega^{2} } + \frac{(a - \sigma)}{(a - \sigma)^{2} +  \omega^{2}}$
\textbf{does not} have any zeros for real and finite $\omega$. The Fourier transform of $h(t)$ is given by $H(\omega)=\frac{2\sigma}{(\sigma)^{2} + \omega^{2} }$ also \textbf{does not} have any zeros for real and finite $\omega$.\\

Because $g(t) h(t) = E_p(t)$ corresponds to \textbf{convolution} of the respective Fourier transforms (\href{https://www.ocf.berkeley.edu/~araman/files/math_z/g_h_E_plots.png}{plot}), therefore real part of Fourier transform of $E_p(t)$ given by $Re[E_{p\omega}(\omega)]$ \textbf{cannot} have zeros for real and finite $\omega$, which \textbf{contradicts} Statement 1. Therefore $G_R(\omega)$  must have \textbf{at least one zero} at $\omega = \omega_{1} \neq 0$ to satisfy Statement 1, where $\omega_1$ is real and finite.\\


Similarly, in Section~\ref{sec:Section_2_1}, we consider a \textbf{modified} \textbf{even symmetric} function $g(t) = f(t) e^{-\sigma t}  u(-t) + f(t) e^{3 \sigma t}  u(t)  $ for $|t_0| \leq \infty$ where $f(t)= e^{-\sigma t_0} E_p(t - t_0) + e^{\sigma t_0} E_p(t + t_0) $ and $h(t)=[ e^{ \sigma t} u(-t) + e^{ - 3 \sigma t} u(t) ]$ where $g(t) h(t) = f(t)$  and show that Fourier transform of the \textbf{even function} $g(t) $ given by $G_R(\omega)$  must have \textbf{at least one zero} at $\omega = \omega_{2}(t_0) \neq 0$, for \textbf{every value} of $t_0$, to satisfy Statement 1, where $\omega_{2}(t_0)$ is real and finite. (~\ref{sec:Appendix_D_6}). \\

If there is more than one solution for $\omega_{2}(t_0)$, these different solutions can remain distinct. This is shown by an example video simulation in  \href{https://www.ocf.berkeley.edu/~araman/files/math_z/demo_w2_t0_v1.mp4}{ link}. In Section~\ref{sec:Section_A_2}, it is shown that $\omega_2(t_0)$ is a well defined continuous function, which is \textbf{at least} differentiable twice.\\


\subsection{\label{sec:level2} \textbf{Step 5: On the zeros of the function $G_{R}(\omega)$  } \protect\\  \lowercase{} }


In Section~\ref{sec:Section_2_1}, we compute the Fourier transform of the even function $g(t)$ given by $G_{R}(\omega)$. We require $G_{R}(\omega) =0$ for $\omega=\omega_2(t_0)$ for every value of $t_0$, to satisfy \textbf{Statement 1}. In general, $\omega_2(t_0) \neq \omega_0$. \\

It is shown that  $R(t_0) = G_R(\omega_2(t_0), t_0)$ is an \textbf{odd} function of variable $t_0$ as follows. 

\begin{eqnarray*}\label{sec_1_2_eq_1}   
R(t_0) =  e^{ 2 \sigma t_0} [ \cos{ (\omega_2(t_0) t_0)} \int_{-\infty}^{t_0}    E_0(\tau)  e^{ - 2 \sigma \tau}  \cos{ ( \omega_2(t_0) \tau)} d\tau + \sin{ (\omega_2(t_0) t_0)}  \int_{-\infty}^{t_0}  E_0(\tau)  e^{ - 2 \sigma \tau} \sin{ (\omega_2(t_0) \tau)} d\tau ] 
\end{eqnarray*}
\begin{equation} \end{equation}

Using Taylor series representation of $E_p(t)=  \displaystyle\sum\limits_{n,k,r,p}  c_{nkrp} e^{b_{krp} t} e^{-\sigma t}$,  we use the fact that $E_0(t)=E_0(-t)$,  we can write as follows.

\begin{eqnarray*}\label{sec_1_2_eq_2}   
R(t_0) =   \displaystyle\sum\limits_{n,k,r,p}  c_{nkrp}   \frac{ (b_{krp}-2\sigma) e^{ (b_{krp}) t_0}}{(\omega_2^{2}(t_0) + (b_{krp}-2\sigma)^{2})} 
\end{eqnarray*}
\begin{equation} \end{equation}

We see that there is a \textbf{one to one correspondence} between the integral representation in Eq.~\ref{sec_1_2_eq_1} and Taylor series representation in Eq.~\ref{sec_1_2_eq_2}. Given that it is easier to show integral convergence, we use only the integral representation in subsequent steps.

\subsection{\label{sec:level2} \textbf{Step 5.1: Method 1: Asymptotic fall off rate argument  } \protect\\  \lowercase{} }

In Section~\ref{sec:Section_A_1}, we consider the asymptotic case and show that $\lim_{t_2 \to \infty}  \omega_{2}(t_2) =  \omega_z $  and derive as follows.


\begin{eqnarray*}\label{sec_1_2_eq_5}  
\lim_{t_0 \to \infty}  \cos{ (\omega_z  t_0)} \int_{-\infty}^{t_0}    E_0(\tau)  e^{ - 2 \sigma \tau}  \cos{ ( \omega_z  \tau)} d\tau + \sin{ (\omega_z  t_0)}  \int_{-\infty}^{t_0}  E_0(\tau)  e^{ - 2 \sigma \tau} \sin{ (\omega_z  \tau)} d\tau   = 0 \\
\lim_{t_0 \to \infty}  \cos{ (\omega_z  t_0)} \int_{-\infty}^{t_0}    E_0(\tau)  e^{ - 2 \sigma \tau}  \sin{ ( \omega_z  \tau)} d\tau - \sin{ (\omega_z  t_0)}  \int_{-\infty}^{t_0}  E_0(\tau)  e^{ - 2 \sigma \tau} \cos{ (\omega_z  \tau)} d\tau   = 0 \\
\int_{-\infty}^{\infty}     E_{0}( t)  e^{ -2 \sigma t}  e^{ i( \omega_z t)} dt = 0
\end{eqnarray*}
\begin{equation} \end{equation}

We started with \textbf{Statement 1} that the Fourier Transform of the function $E_p(t) = E_0(t) e^{-\sigma t} $ has a zero at $\omega = \omega_{0}$ which means that $\int_{-\infty}^{\infty}    E_0(\tau) e^{- \sigma \tau} e^{-i \omega_0 \tau} d\tau = 0$ and we derived the result that $\int_{-\infty}^{\infty}    E_0(\tau) e^{-2 \sigma \tau} e^{-i  \omega_z \tau} d\tau = 0$.\\

We repeat above steps N times till $2^{N} \sigma > \frac{1}{2}$ and get the result $\int_{-\infty}^{\infty}    E_0(\tau) e^{-2^{N} \sigma \tau} e^{-i \omega_{zN} \tau} d\tau = 0$. In each iteration $n$, we use $h(t)=  e^{ 2^{n} \sigma t} u(-t) + e^{ - 3*2^{n} \sigma t} u(t) $. We know that  the Fourier Transform of $E_{0}(t) e^{-2^{N} \sigma t}$  \textbf{does not} have a real zero for $2^{N} \sigma > \frac{1}{2}$, corresponding to $Re[s] > 1$ and we show a \textbf{contradiction} of  \textbf{Statement 1} that the Fourier Transform of the function $E_p(t) = E_0(t) e^{-\sigma t} $ has a zero at $\omega = \omega_{0}$.\\


\subsection{\label{sec:level2} \textbf{Step 6: Method 2: First 2 derivatives of $R(t_0)$  } \protect\\  \lowercase{} }

In Section~\ref{sec:Section_A_2}, we show that  $\omega_2(t_0)$, $R(t_0)$ are \textbf{at least} differentiable twice. 
In Section~\ref{sec:Section_2_2}, we derive the first 2 derivatives of $R(t_0)$ at $t_0=0$ as follows, where \\ $e_0=E_0(0),\omega_{20} = [\omega_{2}(t_0)]_{t_0=0} $. $ m_0 = \int_{-\infty}^{0}    E_0(\tau) e^{-2 \sigma \tau} \cos{ (\omega_{20} \tau)} d\tau $, $ n_0 = \int_{-\infty}^{0}    E_0(\tau) e^{-2 \sigma \tau} \sin{ (\omega_{20} \tau)} d\tau$,\\ $ m_2 = -\omega_{22} \int_{-\infty}^{0}  \tau  E_0(\tau) e^{-2 \sigma \tau} \sin{ (\omega_{20} \tau)} d\tau $. % and $ n_2 = \omega_{22}\int_{-\infty}^{0}    \tau E_0(\tau) e^{-2 \sigma \tau} \cos{ (\omega_{20} \tau)} d\tau$. 

\begin{eqnarray*}\label{sec_1_2_eq_3}   
[ R(t_0) ]_{t_0=0} = m_0 \\
(\frac{dR(t_0)}{dt_0})_{t_0=0} =   e_0  + n_0  \omega_{20} + 2 \sigma m_0 \\
(\frac{d^2R(t_0)}{dt_0^2})_{t_0=0} =   m_2 + \sigma e_0  +2 \sigma n_0  \omega_{20}   + 2 \sigma^{2} m_0 - m_0 \frac{\omega_{20}^2}{2} 
\end{eqnarray*}
\begin{equation} \end{equation}

Given that  $R(t_0) = G_R(\omega_2(t_0), t_0)$ is an \textbf{odd} function of variable $t_0$, we get $m_0=0$ and $m_2 + \sigma e_0  +2 \sigma n_0  \omega_{20}  = 0$.



\subsection{\label{sec:level2} \textbf{Step 7: Next Step  } \protect\\  \lowercase{} }

In Section~\ref{sec:Section_2_3}, we replace $E_p(t)$  by $E_{pp}(t)= e^{\sigma t_2} E_p(t+t_2) + e^{-\sigma t_2} E_p(t-t_2)$, for $|t_2| \leq \infty$ and derive as follows.  %then replace $t_2$ by $t_2$ and 


\begin{eqnarray*}\label{sec_1_2_eq_4}  
m_{0}^{'}(t_2) = R'(t_2) + R'(-t_2) = 0 \\ 
R'(t_2) =   e^{ 2 \sigma t_2} [ \cos{ (\omega_{20}(t_2) t_2)} \int_{-\infty}^{t_2}    E_0(\tau)  e^{ - 2 \sigma \tau}  \cos{ ( \omega_{20}(t_2) \tau)} d\tau + \sin{ (\omega_{20}(t_2) t_2)}  \int_{-\infty}^{t_2}  E_0(\tau)  e^{ - 2 \sigma \tau} \sin{ (\omega_{20}(t_2) \tau)} d\tau ] \\
A(t_2) = m_2^{'}(t_2) + \sigma e_0^{'}(t_2)  + 2 \sigma n_0^{'}(t_2)  \omega_{2}(t_2) = 0 \\
e_0^{'}(t_2) =  E_0(t_2) + E_0(-t_2) \\
n_{0}^{'}(t_2) = n_{0p}(t_2) + n_{0p}(-t_2)  \\
n_{0p}(t_2) =   e^{ 2 \sigma t_2} [ \cos{ (\omega_{2}(t_2) t_2)} \int_{-\infty}^{t_2}    E_0(\tau)  e^{ - 2 \sigma \tau}  \sin{ ( \omega_{2}(t_2) \tau)} d\tau - \sin{ (\omega_{2}(t_2) t_2)}  \int_{-\infty}^{t_2}  E_0(\tau)  e^{ - 2 \sigma \tau} \cos{ (\omega_{2}(t_2) \tau)} d\tau ]  \\
m_2^{'}(t_2) =  m_{2p}(t_2) + m_{2p}(-t_2)  \\
m_{2p}(t_2) =  -  \frac{1}{2} \frac{d^2 \omega_{2}(t_2)}{dt_2^2}  e^{ 2 \sigma t_2} [ \cos{ (\omega_{2}(t_2) t_2)} \int_{-\infty}^{t_2} (\tau - t_2)   E_0(\tau)  e^{ - 2 \sigma \tau}  \sin{ ( \omega_{2}(t_2) \tau)} d\tau \\- \sin{ (\omega_{2}(t_2) t_2)}  \int_{-\infty}^{t_2} (\tau - t_2)   E_0(\tau)  e^{ - 2 \sigma \tau} \cos{ (\omega_{2}(t_2) \tau)} d\tau ] 
\end{eqnarray*}
\begin{equation} \end{equation}


\subsection{\label{sec:level2} \textbf{Step 8: Asymptotic Case and Final result } \protect\\  \lowercase{} }

In Section~\ref{sec:Section_2_4}, we consider the asymptotic case and show that $\lim_{t_2 \to \infty}  \omega_{2}(t_2) =  \omega_z $ and derive as follows.

\begin{eqnarray*}\label{sec_1_2_eq_5}  
\lim_{t_2 \to \infty} A(t_2) = \lim_{t_2 \to \infty} 2 \sigma  \omega_z n_0^{'}(t_2) =  0 \\
\lim_{t_2 \to \infty} n_0^{'}(t_2) = 0 \\
\lim_{t_2 \to \infty} m_0^{'}(t_2)  = 0 \\
\int_{-\infty}^{\infty}     E_{0}( t)  e^{ -2 \sigma t}  e^{ i( \omega_z t)} dt = 0
\end{eqnarray*}
\begin{equation} \end{equation}


We started with \textbf{Statement 1} that the Fourier Transform of the function $E_p(t) = E_0(t) e^{-\sigma t} $ has a zero at $\omega = \omega_{0}$ which means that $\int_{-\infty}^{\infty}    E_0(\tau) e^{- \sigma \tau} e^{-i \omega_0 \tau} d\tau = 0$ and we derived the result that $\int_{-\infty}^{\infty}    E_0(\tau) e^{-2 \sigma \tau} e^{-i  \omega_z \tau} d\tau = 0$.\\

We repeat above steps N times till $(2^{N+1} \sigma) > \frac{1}{2}$ and get the result $\int_{-\infty}^{\infty}    E_0(\tau) e^{-(2^{N+1} \sigma) \tau} e^{-i \omega_{(zN)} \tau} d\tau = 0$. In each iteration $n$, we use $h(t)=  e^{ (2^{N+1} \sigma) t} u(-t) + e^{ - 3*(2^{N+1} \sigma) t} u(t) $. We know that  the Fourier Transform of $E_{0}(t) e^{-(2^{N+1} \sigma) t}$  \textbf{does not} have a real zero for $(2^{N+1} \sigma) > \frac{1}{2}$, corresponding to $Re[s] > 1$ and we show a \textbf{contradiction} of  \textbf{Statement 1} that the Fourier Transform of the function $E_p(t) = E_0(t) e^{-\sigma t} $ has a zero at $\omega = \omega_{0}$.\\




\clearpage
\section{\label{sec:Section_2} An Approach towards Riemann's Hypothesis: Method 1  \protect\\  \lowercase{} }

%\textbf{Formulate New Theorem here:}\\
\textbf{Theorem 1}:   Riemann's Xi function $\xi(\frac{1}{2} + \sigma + i \omega)= E_{p\omega}(\omega)$ does not have zeros for any real value of $-\infty < \omega < \infty$, for $0 < |\sigma| < \frac{1}{2}$, corresponding to the critical strip excluding the critical line, given that $E_0(t) = E_0(-t)$ is an even function of variable $t$, where $E_p(t) =  \frac{1}{2 \pi} \int_{-\infty}^{\infty} E_{p\omega}(\omega) e^{i \omega t} d\omega $, $E_p(t)= E_0(t) e^{-\sigma t}$ and $E_{0}(t) =   2 \sum_{n=1}^{\infty}  [ 2 \pi^{2} n^{4} e^{4t}    - 3 \pi n^{2}   e^{2t} ]  e^{- \pi n^{2} e^{2t}} e^{\frac{t}{2}} $.\\


\textbf{Proof}: We assume that Riemann Hypothesis is false and prove its truth using proof by contradiction.\\
%The proof of this theorem is shown in subsections below.\\

\textbf{Statement 1}: Let us assume that Riemann's Xi function $\xi(\frac{1}{2} + \sigma + i \omega)= E_{p\omega}(\omega)$ has a zero at $\omega = \omega_{0}$ where $\omega_{0}$ is real and finite and $0 < |\sigma| < \frac{1}{2}$, corresponding to the critical strip excluding the critical line.  We will prove that this assumption leads to a \textbf{contradiction}. \\

We will prove it for $0 < \sigma < \frac{1}{2}$ first and then use the property $\xi(\frac{1}{2} + \sigma + i \omega) = \xi(\frac{1}{2} - \sigma - i \omega)$ to show the result for $-\frac{1}{2} < \sigma < 0$ and hence show the result for  $0 < |\sigma| < \frac{1}{2}$.\\

We know that $\omega_{0} \neq 0$, because $\zeta(s)$ has no zeros on the real axis between 0 and 1, when $s=\frac{1}{2}+\sigma + i \omega$ is real, $\omega=0$ and $0 < |\sigma| < \frac{1}{2}$. $^{\citep{ECT}}$ This is shown in detail in first two paragraphs in ~\ref{sec:appendix_C_1}.

%The inverse Fourier Transform of the function $ E_{p\omega}(\omega)$ is given by $E_p(t) = E_0(t) e^{-\sigma t} = \frac{1}{2 \pi} \int_{-\infty}^{\infty} E_{p\omega}(\omega) e^{i \omega t} d\omega $. We see that $E_0(t) = 2 \sum_{n=1}^{\infty}  [ 2 \pi^{2} n^{4} e^{4t}    - 3 \pi n^{2}   e^{2t} ]  e^{- \pi n^{2} e^{2t}} e^{\frac{t}{2}} > 0 $ for all $0 \leq t < \infty$. Given that $E_0(t)=E_0(-t)$, we see that $E_0(t) > 0$ and  $E_p(t) = E_0(t) e^{-\sigma t} > 0 $ for all $-\infty < t < \infty$.\\

%Given that $E_{0\omega}(\omega)$ is an entire function and finite for all $\omega$, we see that $E_0(t)=0$  at $t=\pm \infty$, because if $E_0(t) \neq 0$  at $t=\pm \infty$, then its Fourier transform $E_{0\omega}(\omega)$ will not be finite.
%Hence $E_p(t)= E_0(t) e^{-\sigma t} = 0$ at $t=\pm \infty$ and we showed that $E_p(t) > 0 $ for all $-\infty < t < \infty$. Hence $E_{p\omega}(\omega) = \int_{-\infty}^{\infty} E_p(t) e^{-i \omega t} dt$, evaluated at $\omega=0$ \textbf{cannot} be zero. Hence $E_{p\omega}(\omega)$ \textbf{does not have a zero} at $\omega=0$ and hence $\omega_{0} \neq 0$.\\




%\clearpage
\subsection{\label{sec:Section_2_1} \textbf{On a related function $G(\omega)$ } \protect\\  \lowercase{} }

%Let us consider the single sided function $f(t) =  e^{\sigma t_0} E_p(t + t_0) $ where $t_0$ is a constant delay and we can see that the Fourier Transform of this function $F(\omega)= e^{\sigma t_0} E_{p\omega}(\omega) e^{i \omega t_0}$ also has a zero at $\omega = \omega_{0}$.\\

%Let us consider a new function  $g(t) = g_{-}(t)  u(-t) + g_{+}(t)  u(t)  $ where $g(t)$ is a real and asymmetric function of variable $t$ and $u(t)$ is Heaviside unit step function and $g_{-}(t) = f(t) e^{-\sigma t}$ and $g_{+}(t) = f(t) e^{\sigma t}$ . We can see that $g_1(t)h(t)= f_1(t)$ where $h(t)= [ e^{ \sigma t} u(-t) + e^{ - \sigma t} u(t) ] $. \\


%Similarly, we can compute the Fourier transform of the function $g_{odd}(t)= \frac{1}{2} [ g(t) - g(-t) ]$ given by $i G_{I}(\omega)$. We require $G_{I}(\omega) =0$ for $\omega=\omega_2(t_0)$ for \textbf{every value} of $t_0$, to satisfy \textbf{Statement 1}. In general, $\omega_2(t_0) \neq \omega_0$. This is shown in ~\ref{sec:Appendix_H_1}. \\

%First we compute the Fourier transform of the function $g(t)$ given by $G(\omega)= G_{R}(\omega) + i G_{I}(\omega) $. Let us define $E_q(t)= E_p(-t)$. We can see that $E_p(t - t_0) = E_q(-t + t_0)$ and $E_p(t + t_0) = E_q(-t - t_0)$.
%Substituting $t = -t$ in the second integral below, we have\\



Let us form a new function $f(t) = e^{-\sigma t_0} E_p(t - t_0) +  e^{\sigma t_0} E_p(t + t_0) = [  E_0(t + t_0) +  E_0(t - t_0)]  e^{-\sigma t} = E_{0n}(t) e^{-\sigma t} $, where $|t_0| \leq \infty$, $E_{0n}(t) = E_{0n}(-t) = E_0(t + t_0) +  E_0(t - t_0)$. Its Fourier Transform given by $F(\omega)=  E_{p\omega}(\omega) [ e^{-\sigma t_0} e^{-i \omega t_0} +  e^{\sigma t_0} e^{i \omega t_0} ]$ also has a zero at $\omega = \omega_{0}$.\\

Let us consider a real and \textbf{even symmetric} function  $g(t) = g(-t) = g_{-}(t)  u(-t) + g_{+}(t)  u(t)  $ where $u(t)$ is Heaviside unit step function and $g_{-}(t) = f(t) e^{-\sigma t}$ and $g_{+}(t) =  g_{-}(-t) = f(-t) e^{\sigma t} = f(t) e^{3 \sigma t}$, because $f(t)= E_{0n}(t) e^{-\sigma t} $, $f(-t) e^{\sigma t}= E_{0n}(t) e^{2 \sigma t}$, $f(t) e^{3 \sigma t} = E_{0n}(t) e^{2 \sigma t}$ and $E_{0n}(t) = E_{0n}(-t) $. We see that $g(t) = E_{0n}(t) e^{-2 \sigma t} u(-t) + E_{0n}(t) e^{2 \sigma t} u(t)  $. We can see that $g(t) h(t) = f(t)$ where $h(t)=[ e^{ \sigma t} u(-t) + e^{ - 3 \sigma t} u(t) ]$. \\

%Let us consider a new function  $g(t) = f(t)  u(-t) + f(-t)  u(t)  $ where $g(t)$ is an even function of variable $t$ and $u(t)$ is Heaviside unit step function. We will show that $g(t) [ u(-t) + e^{ - 2 \sigma t} u(t) ] = f(t)$. \\


We can see that $g(t)$ is a real $L^{1}$ integrable function , its Fourier transform $G(\omega)$ is finite for $|\omega| < \infty$ and goes to zero as $\omega \to \pm \infty$, as per \textbf{Riemann-Lebesgue Lemma}[\href{https://en.wikipedia.org/wiki/Riemann-Lebesgue\_lemma}{   Riemann Lebesgue Lemma}]. This is explained in detail in ~\ref{sec:appendix_C_1}. \\

If we take the Fourier transform of the equation $g(t)  h(t) = f(t)$ where $h(t) = [ e^{\sigma t} u(-t) + e^{-3 \sigma t} u(t) ] $, we get $\frac{1}{2\pi} [ G(\omega) \ast H(\omega)] = F(\omega)$ where $\ast$ denotes convolution operation given by $F(\omega) = \frac{1}{2\pi} \int_{-\infty}^{\infty} G(\omega') H(\omega - \omega') d\omega'$ and $H(\omega)=  [ \frac{1}{  \sigma - i \omega} +  \frac{1}{  3 \sigma + i \omega}   ]  = [ \frac{\sigma}{(\sigma^{2} + \omega^{2})} + \frac{3 \sigma}{(9 \sigma^{2} + \omega^{2})} ] + i \omega [ \frac{1}{(\sigma^{2} - \omega^{2})} - \frac{1}{(9 \sigma^{2} + \omega^{2})}  ] $ is the Fourier transform of the function $h(t)$. \\


For \textbf{every value} of $t_0$, we require the Fourier transform of the function $f(t)$ given by $F(\omega)$ to have a zero at $\omega = \omega_{0}$. This implies that the Fourier transform of the \textbf{even} function $g(t) $  given by $G(\omega)=G_R(\omega)$ must have\textbf{ at least one real zero} at $\omega = \omega_{2}(t_0)$ for \textbf{every value} of $t_0$. Because the real part of $H(\omega)$ given by $H_{R}(\omega) = \frac{\sigma}{(\sigma^{2} + \omega^{2})} + \frac{3 \sigma}{(9 \sigma^{2} + \omega^{2})} $ does not have real zeros, if $G_{R}(\omega)$ does not have real zeros, then $F_{R}(\omega)= G_{R}(\omega) \ast H_{R}(\omega)$ obtained by the convolution of $H_{R}(\omega)$ and $G_{R}(\omega)$, cannot possibly have real zeros, which goes against \textbf{Statement 1}.\\

This is explained in detail in ~\ref{sec:Appendix_D_6}.\\




\textbf{Next Step}\\

Let us compute the Fourier transform of the function $g(t)$ given by $G(\omega)$.

\begin{eqnarray*}\label{sec_1_eq_3}    
g(t) = g_{-}(t) u(-t) + g_{+}(t) u(t) = g_{-}(t) u(-t) + g_{-}(-t) u(t)  \\
g(t) = [ e^{-\sigma t_0} E_p(t - t_0) +  e^{ \sigma t_0} E_p(t + t_0)]  e^{-\sigma t} u(-t) + [  e^{-\sigma t_0} E_p(-t - t_0) +  e^{ \sigma t_0} E_p(-t + t_0)] e^{  \sigma t} u(t) \\
G(\omega) = \int_{-\infty}^{\infty} g(t) e^{-i \omega t} dt = \int_{-\infty}^{0} [ e^{-\sigma t_0} E_p(t - t_0) +  e^{ \sigma t_0} E_p(t + t_0)]  e^{-\sigma t}  e^{-i \omega t} dt \\
+ \int_{0}^{\infty} [  e^{-\sigma t_0} E_p(-t - t_0) +  e^{ \sigma t_0} E_p(-t + t_0)] e^{  \sigma t}  e^{-i \omega t} dt  
\end{eqnarray*}
\begin{equation} \end{equation}

In the second integral in above equation ,we can substitute $t=-t$ and we get

\begin{eqnarray*}\label{sec_1_eq_3}    
G(\omega) = \int_{-\infty}^{0} [ e^{-\sigma t_0} E_p(t - t_0) +  e^{ \sigma t_0} E_p(t + t_0)]  e^{-\sigma t}  e^{-i \omega t} dt 
+ \int_{-\infty}^{0} [  e^{-\sigma t_0} E_p(t - t_0) +  e^{ \sigma t_0} E_p(t + t_0)] e^{ - \sigma t}  e^{i \omega t} dt \\
G(\omega) = 2 \int_{-\infty}^{0} [ e^{-\sigma t_0} E_p(t - t_0) +  e^{ \sigma t_0} E_p(t + t_0)]  e^{-\sigma t}  \cos{ \omega t} dt = G_R(\omega) + i G_I(\omega) = G_R(\omega)
\end{eqnarray*}
\begin{equation} \end{equation}


Using the substitutions $t - t_0 = \tau, dt = d\tau$ and $t + t_0 =\tau, dt = d\tau$, we can write the above equation as follows. We use $E_p(\tau) = E_0(\tau) e^{ - \sigma \tau} $.

\begin{eqnarray*}\label{sec_1_eq_4}    
G_R(\omega) = G_R(\omega, t_0) =  G_2(\omega, t_0) + G_2(\omega, -t_0)  \\
G_2(\omega, t_0) = 2 e^{ \sigma t_0} e^{ \sigma t_0} [ \cos{ (\omega t_0)} \int_{-\infty}^{t_0}    E_p(\tau) e^{ - \sigma \tau}  \cos{ ( \omega \tau)} d\tau + \sin{ (\omega t_0)}  \int_{-\infty}^{t_0}  E_p(\tau) e^{ - \sigma \tau}  \sin{ (\omega \tau)} d\tau ] \\
G_2(\omega, t_0) = 2 e^{ 2 \sigma t_0} [ \cos{ (\omega t_0)} \int_{-\infty}^{t_0}    E_0(\tau) e^{ - 2 \sigma \tau}  \cos{ ( \omega \tau)} d\tau + \sin{ (\omega t_0)}  \int_{-\infty}^{t_0}  E_0(\tau) e^{ - 2 \sigma \tau}  \sin{ (\omega \tau)} d\tau ] 
\end{eqnarray*}
\begin{equation} \end{equation}
 

We require $G(\omega) = G_R(\omega)  =0$ for $\omega=\omega_2(t_0)$ for \textbf{every value} of $t_0$, to satisfy \textbf{Statement 1}. Hence we can see that $R(t_0) = \frac{1}{2}  G_2(\omega_2(t_0), t_0)$ is an \textbf{odd function} of variable $t_0$. 

\begin{eqnarray*}\label{sec_1_eq_7}   
 G(\omega_2(t_0), t_0) = G_2(\omega_2(t_0), t_0) + G_2(\omega_2(t_0), -t_0) = 0 \\
R(t_0) = \frac{1}{2} G_2(\omega_2(t_0), t_0)  \\
R(t_0) =  e^{2 \sigma t_0} [ \cos{ (\omega_2(t_0) t_0)} \int_{-\infty}^{t_0}    E_0(\tau)  e^{ - 2 \sigma \tau}  \cos{ ( \omega_2(t_0) \tau)} d\tau + \sin{ (\omega_2(t_0) t_0)}  \int_{-\infty}^{t_0}  E_0(\tau)  e^{ - 2 \sigma \tau} \sin{ (\omega_2(t_0) \tau)} d\tau ] \\
S(t_0) = R(t_0) + R(-t_0) = 0
\end{eqnarray*}
\begin{equation} \end{equation}

We see that $f(t) = e^{-\sigma t_0} E_p(t - t_0) +  e^{\sigma t_0} E_p(t + t_0)$ is \textbf{unchanged} by the substitution $t_0=-t_0$ and hence $\omega_2(t_0)$ is an \textbf{even} function of variable $t_0$. 

%In Section~\ref{sec:Section_3_2}, it is shown that $\omega_2(t_0)$ and $R(t_0)$ are well defined continuous functions, which are \textbf{at least} differentiable twice.\\
%~\ref{sec:Appendix_D_7}

\clearpage
\subsection{\label{sec:Section_A_1} \textbf{ Method 1: Asymptotic Fall off rate argument.} \protect\\  \lowercase{} }

This method \textbf{does not} require differentability of $\omega_2(t_0)$ and is \textbf{independent} of Method 2 in Section~\ref{sec:Section_A_2}.\\

In Section~\ref{sec:Section_3_1}, we show that $\lim_{t_0 \to \infty} g(t)$ is an \textbf{analytic} function, with the \textbf{magnitude} of the step discontinuity at $t=0$ \textbf{decreasing to zero}, and its Fourier transform is an analytic function with \textbf{isolated zeros} and each isolated zero has a single value, as $\lim_{t_0 \to \infty}$. \\ %Hence $\lim_{t_0 \to \infty} \omega_{2}(t_0) = \omega_z \neq 0$ which is a constant. \\

In Section~\ref{sec:Section_A_1_1}, we show that $\lim_{t_0 \to \infty} \omega_{2}(t_0) = \omega_z $ is a constant and we \textbf{rule  out} the pathological case of $\omega_{2}(t_0)$ which is discontinuous everywhere and/or ill-defined. It is shown that the integrals $I_1(t_0) =  \int_{-\infty}^{t_0}     E_{0}( \tau)  e^{ -2 \sigma \tau}  \cos{ (\omega_2(t_0) \tau)} d\tau$ and $I_2(t_0) =  \int_{-\infty}^{t_0}  E_{0}( \tau)  e^{ -2 \sigma \tau}  \sin{ (\omega_2(t_0) \tau)} d\tau$ in Eq.~\ref{sec_a_1_eq_1} \textbf{converge} as $\lim_{t_0 \to \infty}$.
\\% Hence $\lim_{t_0 \to \infty} \omega_{2}(t_0) = \omega_z $ is a constant. \\

As  $\lim_{t_0 \to \infty}$, we can compute $S(t_0)$ in Eq.~\ref{sec_1_eq_7} as follows. The expression for $R(-t_0)$ goes to zero as $\lim_{t_0 \to \infty}$, due to the term $e^{-2 \sigma t_0}$. In the equation for $R(t_0)$, the term $\lim_{t_0 \to \infty} e^{2 \sigma t_0} =  \infty$. Hence we require \\$\lim_{t_0 \to \infty} U(t_0)=  \lim_{t_0 \to \infty} \cos{ (\omega_z  t_0)} \int_{-\infty}^{t_0}    E_0(\tau)  e^{ - 2 \sigma \tau}  \cos{ ( \omega_z  \tau)} d\tau + \sin{ (\omega_z  t_0)}  \int_{-\infty}^{t_0}  E_0(\tau)  e^{ - 2 \sigma \tau} \sin{ (\omega_z  \tau)} d\tau   = 0$. \\We use $\lim_{t_0 \to \infty} \omega_{2}(t_0) = \omega_z $ and write as follows.

\begin{eqnarray*}\label{sec_a_1_eq_1}   
\lim_{t_0 \to \infty}  S(t_0)  =  \lim_{t_0 \to \infty} e^{2 \sigma t_0} [ \cos{ (\omega_2(t_0) t_0)} \int_{-\infty}^{t_0}    E_0(\tau)  e^{ - 2 \sigma \tau}  \cos{ ( \omega_2(t_0) \tau)} d\tau + \sin{ (\omega_2(t_0) t_0)}  \int_{-\infty}^{t_0}  E_0(\tau)  e^{ - 2 \sigma \tau} \sin{ (\omega_2(t_0) \tau)} d\tau ] = 0 \\
\lim_{t_0 \to \infty} U(t_0)= \lim_{t_0 \to \infty}  \cos{ (\omega_z  t_0)} \int_{-\infty}^{t_0}    E_0(\tau)  e^{ - 2 \sigma \tau}  \cos{ ( \omega_z  \tau)} d\tau + \sin{ (\omega_z  t_0)}  \int_{-\infty}^{t_0}  E_0(\tau)  e^{ - 2 \sigma \tau} \sin{ (\omega_z  \tau)} d\tau   = 0
\end{eqnarray*}
\begin{equation} \end{equation}

We define $I_1(t_0) =  \int_{-\infty}^{t_0}     E_{0}( \tau)  e^{ -2 \sigma \tau}  \cos{ (\omega_2(t_0) \tau)} d\tau$ and $I_2(t_0) =  \int_{-\infty}^{t_0}  E_{0}( \tau)  e^{ -2 \sigma \tau}  \sin{ (\omega_2(t_0) \tau)} d\tau$ in Eq.~\ref{sec_a_1_eq_1}  and note that $\lim_{t_0 \to \infty} I_1(t_0)$ and $\lim_{t_0 \to \infty} I_2(t_0)$ tend to a constant, which is finite and determinate, given that $\lim_{t_0 \to \infty} \omega_{2}(t_0) = \omega_z$ \textbf{(Statement A)}. We require $\lim_{t_0 \to \infty} U(t_0)= 0$ and we consider following 2 cases.\\

$\bullet$ \textbf{Case 1:} We consider the general case and require the terms $I_1(t_0)$ and $I_2(t_0)$ have an \textbf{asymptotic fall-off }rate of $e^{-K t_0}$, as  $\lim_{t_0 \to \infty}$, where $K > 2 \sigma$, to satisfy the equation $U(t_0) = 0$. \\

Given the \textbf{asymptotic fall-off }rate of $e^{-K t_0}$, where $K > 2 \sigma$, we see that $\lim_{t_0 \to \infty} I_1(t_0) = 0$ and $\lim_{t_0 \to \infty} I_2(t_0) = 0$.\\

$\bullet$ \textbf{Case 2:} We consider the specific case $\lim_{t_0 \to \infty} I_1(t_0) = V(t_0) \sin{ (\omega_z  t_0)}$ and 
$\lim_{t_0 \to \infty} I_2(t_0) = -V(t_0) \cos{ (\omega_z  t_0)}$, which will satisfy the equation $\lim_{t_0 \to \infty} U(t_0)= 0$. Given Statement A, we require $\lim_{t_0 \to \infty} V(t_0)$ to have an \textbf{asymptotic fall-off }rate of $e^{-K t_0}$, as  $\lim_{t_0 \to \infty}$, where $K > 0$, to satisfy the equation  $U(t_0) = 0$. \\

\textbf{If} $K \leq 0$, we see that $\lim_{t_0 \to \infty} I_1(t_0) = V(t_0) \sin{ (\omega_z  t_0)}$ and $\lim_{t_0 \to \infty} I_2(t_0) = -V(t_0) \cos{ (\omega_z  t_0)}$ will be oscillatory due to the terms $ \sin{ (\omega_z  t_0)}$ and $ \cos{ (\omega_z  t_0)}$ which is \textbf{not} valid, given Statement A.  \\

Given the \textbf{asymptotic fall-off }rate of $e^{-K t_0}$, where $K > 0$, we see that $\lim_{t_0 \to \infty} I_1(t_0) = 0$ and $\lim_{t_0 \to \infty} I_2(t_0) = 0$.\\

In both cases, we arrive at the \textbf{result} that $\int_{-\infty}^{\infty}     E_0(\tau) e^{-2 \sigma \tau} e^{-i  \omega_z \tau} d\tau = 0$. \\


We started with \textbf{Statement 1} that the Fourier Transform of the function $E_p(t) = E_0(t) e^{-\sigma t} $ has a zero at $\omega = \omega_{0}$ which means that $\int_{-\infty}^{\infty}    E_0(\tau) e^{- \sigma \tau} e^{-i \omega_0 \tau} d\tau = 0$ and we derived the result that $\int_{-\infty}^{\infty}    E_0(\tau) e^{-2 \sigma \tau} e^{-i  \omega_z \tau} d\tau = 0$.\\

Now we can repeat the steps in Section 2, starting with the new result that $\int_{-\infty}^{\infty}    E_0(\tau) e^{-2 \sigma \tau} e^{-i \omega_z \tau} d\tau = 0$ and $\sigma$ replaced by $2 \sigma$ and derive the next result that $\int_{-\infty}^{\infty}    E_0(\tau) e^{-4 \sigma \tau} e^{-i \omega_{(z1)} \tau} d\tau = 0$.\\

We can repeat above steps N times till $(2^{N+1} \sigma) > \frac{1}{2}$ and get the result $\int_{-\infty}^{\infty}    E_0(\tau) e^{-(2^{N+1} \sigma) \tau} e^{-i \omega_{(zN)} \tau} d\tau = 0$. In each iteration $n$, we use $h(t)=  e^{ (2^{N+1} \sigma) t} u(-t) + e^{ - 3*(2^{N+1} \sigma) t} u(t) $, $\omega_2(t_0)$ replaced by $\omega_{2n}(t_0)$ and $\omega_z$ replaced by $\omega_{(zn)}$. We know that  the Fourier Transform of $E_{0}(t) e^{-(2^{N+1} \sigma) t} =    \sum_{n=1}^{\infty}  [ 4 \pi^{2} n^{4} e^{4t}    - 6 \pi n^{2}   e^{2t} ]  e^{- \pi n^{2} e^{2t}} e^{\frac{t}{2}} e^{-(2^{N+1} \sigma) t}$ given by $E_{p\omega N}(\omega)=\xi(\frac{1}{2}+ 2^N \sigma + i \omega)$ \textbf{does not} have a real zero for $(2^{N+1} \sigma) > \frac{1}{2}$ , corresponding to $Re[s] > 1$. \\% for  $0 < |\sigma| < \frac{1}{2}$ Here we use the well known fact that $E_0(t)=E_0(-t)$. \\

We have shown this result for $0 < \sigma < \frac{1}{2}$ and then use the property $\xi(\frac{1}{2} + \sigma + i \omega) = \xi(\frac{1}{2} - \sigma - i \omega)$ to show the result for $-\frac{1}{2} < \sigma < 0$. Hence we have produced a \textbf{contradiction} of  \textbf{Statement 1} that the Fourier Transform of the function $E_p(t) = E_0(t) e^{-\sigma t} $ has a zero at $\omega = \omega_{0}$ for  $0 < |\sigma| < \frac{1}{2}$.\\

Previous version of this result is in ~\ref{Appendix_D_7}.

\subsection{\label{sec:Section_A_1_1} \textbf{ Integral convergence} \protect\\  \lowercase{} }

In this section, we show that $\lim_{t_0 \to \infty} \omega_{2}(t_0)$ equals a well defined constant and that the integrals \\$I_1(t_0) =  \int_{-\infty}^{t_0}     E_{0}( \tau)  e^{ -2 \sigma \tau}  \cos{ (\omega_2(t_0) \tau)} d\tau$ and $I_2(t_0) =  \int_{-\infty}^{t_0}  E_{0}( \tau)  e^{ -2 \sigma \tau}  \sin{ (\omega_2(t_0) \tau)} d\tau$ in Eq.~\ref{sec_a_1_eq_1} \textbf{converge} as $\lim_{t_0 \to \infty}$. \\
%\textbf{or} if $\lim_{t_0 \to \infty} \omega_{2}(t_0)$ has discontinuities, then

We take Eq.~\ref{sec_1_eq_7} and write it as follows using $E_{0}^{'}(t)  = E_0(t+t_0) + E_0(t-t_0) $. If we substitute $t+t_0=\tau$ and $t-t_0=\tau$, we get Eq.~\ref{sec_1_eq_7} copied below.

\begin{eqnarray*}\label{sec_a_1_1_eq_1}   
S(t_0) = R(t_0) + R(-t_0) = \int_{-\infty}^{0}    E_0^{'}(\tau)  e^{ - 2 \sigma \tau}  \cos{ ( \omega_2(t_0) \tau)} d\tau  = 0 \\
R(t_0) =  e^{2 \sigma t_0} [ \cos{ (\omega_2(t_0) t_0)} \int_{-\infty}^{t_0}    E_0(\tau)  e^{ - 2 \sigma \tau}  \cos{ ( \omega_2(t_0) \tau)} d\tau + \sin{ (\omega_2(t_0) t_0)}  \int_{-\infty}^{t_0}  E_0(\tau)  e^{ - 2 \sigma \tau} \sin{ (\omega_2(t_0) \tau)} d\tau ] 
\end{eqnarray*}
\begin{equation} \end{equation}

We consider the \textbf{pathological }case where $\omega_2(t_0)$ is \textbf{discontinuous everywhere} and/or ill-defined (\textbf{Statement 2}). Then $ S(t_0) =  \int_{-\infty}^{0}    E_0^{'}(\tau)  e^{ - 2 \sigma \tau}  \cos{ (\omega_2(t_0) \tau)} d\tau$ is ill-defined everywhere as a result (\textbf{Statement 3}) and we can show that this pathological case  \textbf{does not} apply to $\omega_2(t_0)$. \\

We see that the integral $S(t_0)=0$ in Eq.~\ref{sec_a_1_1_eq_1} \textbf{converges} for $|t_0|\leq \infty$ and this result is derived from \textbf{Statement 1} (Riemann's Xi function has a zero in the critical strip excluding the critical line). This contradicts Statement 2 and 3.\\

Given that $\lim_{t_0 \to \infty} R(-t_0)=0$, we have $\lim_{t_0 \to \infty} R(t_0)= \lim_{t_0 \to \infty} S(t_0) = 0$.   \\

%\textbf{If} $\lim_{t_0 \to \infty} \omega_2(t_0)$ in Eq.~\ref{sec_a_1_1_eq_1} is ill-defined as a result and  $\lim_{t_0 \to \infty}  S(t_0) =  \lim_{t_0 \to \infty}  \int_{-\infty}^{0}    E_0^{'}(\tau)  e^{ - 2 \sigma \tau}  \cos{ (\omega_2(t_0) \tau)} d\tau$ is ill-defined  as a result (\textbf{Statement 3}), \\
%we can show that this pathological case  \textbf{does not} apply to $\omega_2(t_0)$. \\

%This implies that \textbf{either} $\lim_{t_0 \to \infty} \omega_{2}(t_0)$ equals  a constant \textbf{or} if $\lim_{t_0 \to \infty} \omega_{2}(t_0)$ has discontinuities, then the integral $I_1(t_0) =  \int_{-\infty}^{t_0}     E_{0}( \tau)  e^{ -2 \sigma \tau}  \cos{ (\omega_2(t_0) \tau)} d\tau$  in Eq.~\ref{sec_a_1_eq_1} \textbf{converges} as $\lim_{t_0 \to \infty} $.\\


%We can use a similar argument for the term $I_2(t_0)$ in Eq.~\ref{sec_a_1_eq_2} and show that \\ $I_2(t_0) =  \int_{-\infty}^{t_0}  E_{0}( \tau)  e^{ -2 \sigma \tau}  \sin{ (\omega_2(t_0) \tau)} d\tau$ converges  as $\lim_{t_0 \to \infty}$ as follows.
%\textbf{or} $\lim_{t_0 \to \infty} \omega_{2}(t_0)$ equals a constant, 

%\begin{eqnarray*}\label{sec_a_1_1_eq_2}   
%S_1(t_0) = R_1(t_0) + R_1(-t_0) = \int_{-\infty}^{0}    E_0^{'}(\tau)  e^{ - 2 \sigma \tau}  \sin{ ( \omega_2(t_0) \tau)} d\tau =  0 \\
%R_1(t_0) =  e^{2 \sigma t_0} [ \cos{ (\omega_2(t_0) t_0)} \int_{-\infty}^{t_0}    E_0(\tau)  e^{ - 2 \sigma \tau}  \sin{ ( \omega_2(t_0) \tau)} d\tau - \sin{ (\omega_2(t_0) t_0)}  \int_{-\infty}^{t_0}  E_0(\tau)  e^{ - 2 \sigma \tau} \cos{ (\omega_2(t_0) \tau)} d\tau ] 
%\end{eqnarray*}
%\begin{equation} \end{equation}

%$\bullet$ \textbf{If} Statement 1 and 2 are true, \textbf{then} we get the result that the integral  \\$  S(t_0)= \int_{-\infty}^{0}    E_0^{'}(\tau)  e^{ - 2 \sigma \tau}  \cos{ ( \omega_2(t_0) \tau)} d\tau$  \textbf{converges} as $\lim_{t_0 \to \infty}$. Given that $\lim_{t_0 \to \infty} R(-t_0)=0$, we have \\$\lim_{t_0 \to \infty} R(t_0)= \lim_{t_0 \to \infty} S(t_0) = 0$. \\

%Hence the results derived in Section~\ref{sec:Section_A_1} are \textbf{valid}, with $\omega_z$ used as a shorthand notation for $\lim_{t_0 \to \infty} \omega_{2}(t_0)$ and the integrals $I_1(t_0) =  \int_{-\infty}^{t_0}     E_{0}( \tau)  e^{ -2 \sigma \tau}  \cos{ (\omega_2(t_0) \tau)} d\tau$ and $I_2(t_0) =  \int_{-\infty}^{t_0}  E_{0}( \tau)  e^{ -2 \sigma \tau}  \sin{ (\omega_2(t_0) \tau)} d\tau$ in Eq.~\ref{sec_a_1_eq_1} \textbf{converge} as $\lim_{t_0 \to \infty}$.  \\


$\bullet$ We can \textbf{rule out} the \textbf{pathological }case where $\omega_2(t_0)$ is \textbf{discontinuous everywhere} and/or ill-defined, as follows. \textbf{If} \textbf{Statements 1, 2 and 3} were true, \textbf{then} the result that the integral in Eq.~\ref{sec_a_1_1_eq_1} converges, suggests one of the following: \\

a) Statement 1  is true and above result \textbf{ contradicts} Statement 2 and 3 and hence we can \textbf{rule out}  pathological case for $\omega_2(t_0)$ \textbf{or} \\

b) Statements 2 and 3 are true and \textbf{Statement 1 is false} and we complete the proof of theorem 1 at this point. We \textbf{do not} require to show that $\omega_2(t_0)$ is \textbf{not} pathological, for this case. \\


$\bullet$ Let us consider the \textbf{pathological }case where $\omega_2(t_0)$ is \textbf{discontinuous everywhere} and/or ill-defined. In Eq.~\ref{sec_a_1_eq_1} copied below, we see that $ \cos{ (\omega_2(t_0)  t_0)} $ and $ \sin{ (\omega_2(t_0)  t_0)} $ are ill-defined and the integrals are also ill-defined functions.

\begin{eqnarray*}\label{sec_a_2_eq_3}   
\lim_{t_0 \to \infty}  \cos{ (\omega_2(t_0)  t_0)} \int_{-\infty}^{t_0}    E_0(\tau)  e^{ - 2 \sigma \tau}  \cos{ ( \omega_2(t_0)  \tau)} d\tau + \sin{ (\omega_2(t_0)  t_0)}  \int_{-\infty}^{t_0}  E_0(\tau)  e^{ - 2 \sigma \tau} \sin{ (\omega_2(t_0)  \tau)} d\tau   = 0
\end{eqnarray*}
\begin{equation} \end{equation}

Hence  we see that (ill-defined function) * (ill-defined function) + (ill-defined function) * (ill-defined function) = 0, 
as $\lim_{t_0 \to \infty} $, and this \textbf{does not} make sense. Therefore the assumption that $\omega_2(t_0)$ is \textbf{discontinuous everywhere} and/or ill-defined is \textbf{false}, if \textbf{Statement 1} is true.\\

$\bullet$ Let us consider the case where $\omega_2(t_0)$ is well defined but \textbf{first derivative} $\frac{d\omega_2(t_0)}{dt_0} $ is \textbf{discontinuous everywhere} and/or ill-defined. We take the first derivative of Eq.~\ref{sec_a_1_1_eq_1}.

\begin{eqnarray*}\label{sec_a_2_eq_4} 
\frac{dS(t_0)}{dt_0} = - \frac{d\omega_2(t_0)}{dt_0} \int_{-\infty}^{0}  \tau  E_0^{'}(\tau)  e^{ - 2 \sigma \tau}  \sin{ ( \omega_2(t_0) \tau)} d\tau +   \int_{-\infty}^{0}   \frac{d E_0^{'}(\tau)}{dt_0}  e^{ - 2 \sigma \tau}  \cos{ ( \omega_2(t_0) \tau)} d\tau  = 0   
\end{eqnarray*}
\begin{equation} \end{equation}

Hence  we see that (ill-defined function) * (ill-defined function) +  (ill-defined function) = 0, 
for all $|t_0| \leq \infty$, and this \textbf{does not} make sense. Therefore the assumption that $\frac{d\omega_2(t_0)}{dt_0}$ is \textbf{discontinuous everywhere} and/or ill-defined is \textbf{false}, if \textbf{Statement 1} is true.\\

$\bullet$ \textbf{If }Statement 1 is  true,\textbf{ then} we have shown that  \textbf{Statement 2 is false} and hence $\omega_{2}(t_0)$ is a well-defined function and $\lim_{t_0 \to \infty} \omega_{2}(t_0) = \omega_z$ is a well defined constant (also shown in Section~\ref{sec:Section_3_1}). Hence the results derived in Section~\ref{sec:Section_A_1} are \textbf{valid}, with constant $\omega_z$. Hence the integrals  $I_1(t_0), I_2(t_0)$ in Eq.~\ref{sec_a_1_eq_1} converge.\\



%$\bullet$ Let us consider the case $\omega_2(t_0)$ has is \textbf{discontinuous everywhere}, which produces Dirac delta functions everywhere in $\frac{d\omega_2(t_0)}{dt_0}$. In Eq.~\ref{sec_a_1_1_eq_3}, we take the first derivative of $S_1(t_0)$ in Eq.~\ref{sec_a_1_1_eq_2} and  the left hand side has $\frac{dS_1(t_0)}{dt_0}$ which is \textbf{discontinuous} everywhere, while the right hand side term $\frac{d\omega_2(t_0)}{dt_0}$ has \textbf{Dirac delta} functions everywhere, while  integrals in the right hand side are continuous functions, which is\textbf{ not} possible, for a general non-zero integrals.\\ 

%If $C(t_0)=0$ for all $|t_0|\leq \infty$, then $\frac{dM(t_0)}{dt_0} =0$ for all $|t_0|\leq \infty$ and $M(t_0)$ is a constant, which means  $\omega_2(t_0)$ is a constant, in which case the assumption that $\omega_2(t_0)$ is \textbf{discontinuous everywhere} is \textbf{false}.\\
%We can use arguments used in previous paragraphs and show that this case is \textbf{not} possible.\\

%\begin{eqnarray*}\label{sec_a_1_1_eq_3}   
%\frac{dS_1(t_0)}{dt_0} = \frac{d\omega_2(t_0)}{dt_0} \int_{-\infty}^{0}  \tau  E_0^{'}(\tau)  e^{ - 2 \sigma \tau}  \cos{ ( \omega_2(t_0) \tau)} d\tau + \int_{-\infty}^{0}    \frac{d E_0^{'}(\tau)}{dt_0}  e^{ - 2 \sigma \tau}  \sin{ ( \omega_2(t_0) \tau)} d\tau = 0
%\end{eqnarray*}
%\begin{equation} \end{equation}




\section{\label{sec:Section_A_2} \textbf{ Method 2: $\omega_2(t_0)$, $R(t_0)$ are \textbf{at least} differentiable twice. } \protect\\  \lowercase{} }

In this section, which is applicable for the \textbf{non-pathological}, well defined case of $\omega_2(t_0)$ and $\frac{d\omega_2(t_0)}{dt_0}$, it is shown that $\omega_2(t_0)$, $R(t_0)$ and $M(t_0), N(t_0)$ are well defined continuous functions, which are \textbf{at least} differentiable twice. This method is \textbf{independent} of Method 1 in Section~\ref{sec:Section_A_1}.\\


In ~\ref{sec:Appendix_D_6}, $\omega_2(t_0)$ is shown to be \textbf{finite} for all  $|t_0| \leq \infty$. This means there are \textbf{no} Dirac delta functions present in $\omega_2(t_0)$. \\

There is a well known equation describing derivatives of Dirac delta function  $t^{2r} \delta^{2r}(t) = (-1)^{2r} (!(2r)) \delta(t) = (!(2r)) \delta(t)$  \href{https://mathworld.wolfram.com/DeltaFunction.html}{(Eq. 17 in link)}.\\

We take the first 2 derivatives of $S(t_0)$, \textbf{even if} we \textbf{assume} that the first 2 derivatives contains Dirac delta functions and we show that the \textbf{assumption} that $\frac{d\omega_2(t_0)}{dt_0}$ or $\frac{d^2\omega_2(t_0)}{dt_0^2} $ has a Dirac delta function is\textbf{ false}. \\

We take the first derivative of $S(t_0)$ in  Eq.~\ref{sec_a_1_1_eq_1} as follows where $E_{0}^{'}(t)  = E_0(t+t_0) + E_0(t-t_0) $.

\begin{eqnarray*}\label{sec_a_2_eq_0}   
S(t_0)  = \int_{-\infty}^{0}    E_0^{'}(\tau)  e^{ - 2 \sigma \tau}  \cos{ ( \omega_2(t_0) \tau)} d\tau  = 0 \\
\frac{dS(t_0)}{dt_0} =  - \frac{d\omega_2(t_0)}{dt_0}  \int_{-\infty}^{0} \tau   E_0^{'}(\tau)  e^{ - 2 \sigma \tau}  \sin{ ( \omega_2(t_0) \tau)} d\tau +   \int_{-\infty}^{0}    \frac{d E_0^{'}(\tau)}{dt_0}  e^{ - 2 \sigma \tau}  \cos{ ( \omega_2(t_0) \tau)} d\tau = 0 \\
\frac{d\omega_2(t_0)}{dt_0} P(t_0) =  Q(t_0), \quad
P(t_0) = \int_{-\infty}^{0} \tau   E_0^{'}(\tau)  e^{ - 2 \sigma \tau}  \sin{ ( \omega_2(t_0) \tau)} d\tau, \quad Q(t_0)= \int_{-\infty}^{0}    \frac{d E_0^{'}(\tau)}{dt_0}  e^{ - 2 \sigma \tau}  \cos{ ( \omega_2(t_0) \tau)} d\tau 
\end{eqnarray*}
\begin{equation} \end{equation}

$\bullet$ Let us consider the case $\omega_2(t_0)$ has a \textbf{step discontinuity} at $t_0=\pm t_A$ of magnitude $A_0$ and continuous everywhere else. In this case, $\frac{d\omega_2(t_0)}{dt_0} = A_0 (\delta(t-t_A) - \delta(t+t_A) ) + B(t_0) $ has a\textbf{ Dirac delta} function at $t_0= \pm t_A$ given that $\omega_2(t_0)$ has even symmetry and $B(t_0)$ does not have Dirac delta function components. We see that both integrals $P(t_0), Q(t_0)$ in Eq.~\ref{sec_a_2_eq_0} are \textbf{continuous} functions, because integral of a rectangular function with step discontinuity is a triangular function which is continuous.\\

 It is possible that the Dirac delta function at $t=t_A$ in $\frac{d\omega_2(t_0)}{dt_0}$ is cancelled if $P(t_A)=0$. We see that the term $\frac{d\omega_2(t_0)}{dt_0}$ \textbf{does not} have any other step discontinuity other than the Dirac delta function at $t=t_A$, given that we \textbf{require} $\frac{d\omega_2(t_0)}{dt_0} P(t_0) =  Q(t_0)$ (\textbf{Result A}). \\

We take $M(t_0)$ and its first derivative in Eq.~\ref{app_F_4_eq_1}  as follows. 

\begin{eqnarray*}\label{sec_a_2_eq_1}   
M(t_0) =  \int_{-\infty}^{0}    E_0(\tau)e^{-2 \sigma \tau}  \cos{ (\omega_2(t_0) \tau)} d\tau \\
\frac{dM(t_0)}{dt_0} =  - \frac{d\omega_2(t_0)}{dt_0}  \int_{-\infty}^{0}  \tau  E_0(\tau)e^{-2 \sigma \tau}  \sin{ (\omega_2(t_0) \tau)} d\tau  =  - \frac{d\omega_2(t_0)}{dt_0} C(t_0) \\
\end{eqnarray*}
\begin{equation} \end{equation}




$\bullet$  Let us consider the case $\omega_2(t_0)$ has a \textbf{step discontinuity} at $t_0=\pm t_A$ of magnitude $A_0$. In this case, $\frac{d\omega_2(t_0)}{dt_0} = A_0 (\delta(t-t_A) - \delta(t+t_A) ) + B(t_0) $ has a\textbf{ Dirac delta} function at $t_0= \pm t_A$ given that $\omega_2(t_0)$ has even symmetry and $B(t_0)$ does not have Dirac delta function components. We see that $M(t_0)$ is a \textbf{continuous} function whose first derivative  $\frac{dM(t_0)}{dt_0}$ has a\textbf{ step discontinuity} at $t_0=\pm t_A$, because $M(t_0)$ is obtained by \textbf{integrating} terms containing $\omega_2(t_0)$. We see that $C(t_0)$ is also a  \textbf{continuous} function  for the same reason.\\

In Eq.~\ref{sec_a_2_eq_1}, we see that, at $t_0 = t_A$, the left hand side of the equation $\frac{dM(t_0)}{dt_0}$ has a \textbf{step discontinuity} at $t_0=t_A$, while  the terms on the right hand side $C(t_0)$ is continuous, and $\frac{d\omega_2(t_0)}{dt_0}$ has a Dirac delta function at $t_0=\pm t_A$. Hence the right hand side is \textbf{either} a continuous function if $C(t_A)=0$\textbf{ or} has a Dirac delta at $t_0 = t_A$. This is \textbf{not} possible. Hence we infer that $\omega_2(t_0)$ \textbf{does not} have a step discontinuity at $t_0=\pm t_A$. \\

For example, $\frac{dM(t_0)}{dt_0}$ has a left limit value of $M_0$ and a \textbf{different} right limit value of $M_1 \neq M_0$ at $t_0 = t_A$. In the right hand side of Eq.~\ref{sec_a_2_eq_1}, if $C(t_A)=0$, then the Dirac delta function term contribution vanishes at $t_0 = t_A$ and left limit value and right limit value are the \textbf{same} at $t_0 = t_A$ because $C(t_0)$ is a a continuous function and $\frac{d\omega_2(t_0)}{dt_0}$ is continuous in the vicinity of $t_0 = t_A$ (\textbf{Result A}), besides the Dirac delta function at $t_0 = t_A$.\\

$\bullet$ Let us consider the case $\omega_2(t_0)$ is a continuous function but $\frac{d\omega_2(t_0)}{dt_0}$ has a \textbf{step discontinuity} at $t_0= \pm t_A$ of magnitude $A_1$. In Eq.~\ref{sec_a_2_eq_0}, we see that $P(t_0), Q(t_0)$ are continuous functions, while $\frac{d\omega_2(t_0)}{dt_0}$ has a \textbf{step discontinuity} at $t_0= \pm t_A$. This is clearly \textbf{not} possible. Hence we infer that $\frac{d\omega_2(t_0)}{dt_0}$ \textbf{does not} have a step discontinuity at $t_0=\pm t_A$. \\

%In Eq.~\ref{sec_a_2_eq_1}, we see that $\frac{dM(t_0)}{dt_0}$ on the left hand side, is a continuous function, because it is obtained by integrating a continuous $\omega_2(t_0)$. On the right hand side, $C(t_0)$ is a continuous function, while $\frac{d\omega_2(t_0)}{dt_0}$ has a \textbf{step discontinuity} at $t_0= \pm t_A$. This is clearly \textbf{not} possible. Hence we infer that $\frac{d\omega_2(t_0)}{dt_0}$ \textbf{does not} have a step discontinuity at $t_0=\pm t_A$. \\

%Now we take the second derivative of $M(t_0)$ as follows. 

%\begin{eqnarray*}\label{sec_a_2_eq_2}   
%\frac{d^{2}M(t_0)}{dt_0^{2}} = - \frac{d^2\omega_2(t_0)}{dt_0^2}  \int_{-\infty}^{0}  \tau  E_0(\tau)e^{-2 \sigma \tau}  \sin{ (\omega_2(t_0) \tau)} d\tau - (\frac{d\omega_2(t_0)}{dt_0})^{2}  \int_{-\infty}^{0}  \tau^2  E_0(\tau)e^{-2 \sigma \tau}  \cos{ (\omega_2(t_0) \tau)} d\tau  \\
%\frac{d^{2}M(t_0)}{dt_0^{2}} + (\frac{d\omega_2(t_0)}{dt_0})^{2}  \int_{-\infty}^{0}  \tau^2  E_0(\tau)e^{-2 \sigma \tau}  \cos{ (\omega_2(t_0) \tau)} d\tau  = - \frac{d^2\omega_2(t_0)}{dt_0^2}  \int_{-\infty}^{0}  \tau  E_0(\tau)e^{-2 \sigma \tau}  \sin{ (\omega_2(t_0) \tau)} d\tau 
%\end{eqnarray*}
%\begin{equation} \end{equation}

%In Eq.~\ref{sec_a_2_eq_2}, using arguments in previous paras, we see that the terms on the left hand side of the equation $\frac{d^{2}M(t_0)}{dt_0^{2}}$ and $(\frac{d\omega_2(t_0)}{dt_0})^{2}$ have \textbf{ step discontinuity} at $t_0=\pm t_A$.
%On the other hand, the integral on the right hand side is continuous, and $\frac{d^2\omega_2(t_0)}{dt_0^2}$ has a Dirac delta function. This is not possible, given similar arguments used earlier. Hence we infer that $\frac{d\omega_2(t_0)}{dt_0}$ \textbf{does not} have a step discontinuity at $t_0=\pm t_A$. \\

The above arguments apply to the case of one or more \textbf{isolated }\textbf{step discontinuitities} in $\omega_2(t_0)$ and $\frac{d\omega_2(t_0)}{dt_0}$.\\

Hence we have shown that $\omega_2(t_0)$, $R(t_0)$ and $M(t_0), N(t_0)$ are well defined continuous functions, which are \textbf{at least} differentiable twice.\\


%which produces Dirac delta functions everywhere in $\frac{d\omega_2(t_0)}{dt_0}
% In Eq.~\ref{sec_a_2_eq_1}, the left hand side has $\frac{dM(t_0)}{dt_0}$ which is \textbf{discontinuous} everywhere, while the right hand side term $\frac{d\omega_2(t_0)}{dt_0}$ has \textbf{Dirac delta} functions everywhere, while $C(t_0)$ is a continuous function, which is\textbf{ not} possible, for a general non-zero  $C(t_0)$.\\ 

%If $C(t_0)=0$ for all $|t_0|\leq \infty$, then $\frac{dM(t_0)}{dt_0} =0$ for all $|t_0|\leq \infty$ and $M(t_0)$ is a constant, which means  $\omega_2(t_0)$ is a constant, in which case the assumption that $\omega_2(t_0)$ is \textbf{discontinuous everywhere} is \textbf{false}.\\
%We can use arguments used in previous paragraphs and show that this case is \textbf{not} possible.\\



%$\bullet$ We can show that $g(t) = E_{0n}(t) e^{-2 \sigma t} u(-t) + E_{0n}(t) e^{2 \sigma t} u(t)$ has an \textbf{exponential} fall-off rate as $|t| \to \infty$ where $E_{0n}(t) = E_0(t + t_0) +  E_0(t - t_0)$ and $E_0(t) = 2 \sum_{n=1}^{\infty}  [ 2 \pi^{2} n^{4} e^{4t}    - 3 \pi n^{2}   e^{2t} ]  e^{- \pi n^{2} e^{2t}} e^{\frac{t}{2}}  $. We see that $g(t)$ goes to zero as $|t| \to \infty$ with its order of decay greater than $e^{\frac{3t}{2}}$, for $0 < \sigma < \frac{1}{2}$, for every value of $t_0$. \\

%In ~\ref{sec:appendix_C_3} and ~\ref{sec:appendix_C_5}, it is shown that the Fourier transform of an analytic function has exponential fall-off rate. Hence, for every value of $t_0$, the Fourier transform of $g(t)$ given by $G_R(\omega, t_0)$ in Eq.~\ref{sec_1_eq_7} is an \textbf{analytic} function which is infinitely differentiable. Given that $\omega_2(t_0)$ is the location of the zeros in $G_R(\omega, t_0)$, $\omega_2(t_0)$ is also an analytic function, which is differentiable \textbf{at least} twice.\\
%

%Hence  $R(t_0)$, $M(t_0)$ and $N(t_0)$ are also analytic functions which are differentiable \textbf{at least} twice, because they contain $E_0(\tau) e^{ -2 \sigma \tau}, \cos(\omega_2(t_0) \tau), \sin(\omega_2(t_0) \tau)$ terms  which are analytic functions.\\


%$\bullet$ There is another \textbf{independent} reason why $\omega_2(t_0)$ is an analytic function, which is differentiable \textbf{at least} twice. In ~\ref{sec:Appendix_D_6}, $\omega_2(t_0)$ is shown to be \textbf{finite} for all  $|t_0| \leq \infty$. This means there are no Dirac delta functions present in $\omega_2(t_0)$. \\

%The equation for $S(t_0)$ in Eq.~\ref{sec_1_eq_7} has terms containing $ \cos(\omega_2(t_0) t_0), \sin(\omega_2(t_0) t_0)$ outside the integrals. \textbf{If} $\omega_2(t_0)$ has a \textbf{step} discontinuity at $t_0= t_A$ (similar to a rectangular function), \textbf{then} $S(t_0)$ will also have a \textbf{step} discontinuity at $t_0= t_A$, which \textbf{contradicts} the result in Eq.~\ref{sec_1_eq_7}, which requires that $S(t_0) = R(t_0) + R(-t_0) = 0$ for all $|t_0|\leq \infty$. This implies that $\omega_2(t_0)$  is a \textbf{continuous} function which is differentiable at least once. \\

%Now we take the first derivative of $S(t_0)$ and use the arguments in above para, with $\omega_2(t_0)$ replaced by $\frac{d\omega_2(t_0)}{dt_0}$ and we can show that $\frac{d\omega_2(t_0)}{dt_0}$  is a \textbf{continuous} function which is differentiable at least once and hence $\omega_2(t_0)$  is a \textbf{continuous} function which is differentiable at least twice. \\

%The equation for $S(t_0)$ in Eq.~\ref{sec_1_eq_7} is obtained by \textbf{integrating} terms containing $ \cos(\omega_2(t_0) \tau), \sin(\omega_2(t_0) \tau)$. \textbf{If} $\omega_2(t_0)$ has a \textbf{step} discontinuity at $t_0= t_A$ (rectangular function), \textbf{then} $S(t_0)$ will have a triangular shape, whose \textbf{first derivative} given by $\frac{d S(t_0)}{dt_0}$ has a \textbf{step} discontinuity at $t_0= t_A$, which \textbf{contradicts} the result in Eq.~\ref{sec_1_eq_7}, which implies that $S(t_0) = R(t_0) + R(-t_0) = 0$ and $\frac{d S(t_0)}{dt_0} = 0$ for all $|t_0|\leq \infty$.\\



%We see from Section~\ref{sec:Section_3_1} that $\lim_{t_0 \to \infty} \omega_{2}(t_0) = \omega_z$ is a constant which is also finite and that as $t_0 \to \infty$, $ \omega_{2}(t_0)$ becomes an \textbf{analytic} function which is infinitely differentiable. This means, \textbf{if we assume} that $\omega_{2}(t_0)$ is a Weierstrass type of function \href{https://en.wikipedia.org/wiki/Weierstrass_function}{(link)} which is differentiable nowhere (\textbf{Statement A}), this can happen\textbf{ only for} finite $t_0$. As $t_0 \to \infty$, $ \omega_{2}(t_0)$ approaches a finite constant $ \omega_z$, which \textbf{does not} jump up and down like a Weierstrass type of function. \\

%Hence, as $t_0 \to \infty$, $ \omega_{2}(t_0)$ becomes a continuous function which is differentiable\textbf{ at least} twice.
%The results in following sections hold for the region $t_0 \to \infty$, where $ \omega_{2}(t_0)$ becomes an analytic function.

\subsection{\label{sec:Section_2_2} \textbf{ First 2 derivatives of $R(t_0)$  } \protect\\  \lowercase{} }


%In this section, we assume that $\omega_2(t_0)$ is analytic in $|t_0| \leq \infty$ and hence can be expanded using Taylor series around $t_0=0$, given by $\omega_2(t_0) = w_{20}  + w_{22} t_0^{2} +... = \displaystyle\sum_{k=0}^{\infty} w_{2(2k)} t_0^{2k}$. [ We can also derive the results for $R(t_0)$ using integration by parts, \textbf{without} using Taylor series for $\omega_2(t_0)$ as shown in ~\ref{sec:appendix_F} ].\\



%If $\omega_2(t_0)$ is a Weierstrass type of function which is differentiable nowhere (\textbf{Statement A}), then its first 2 derivatives are not finite. Given that derivatives of Dirac delta functions are defined by a well known equation \href{https://mathworld.wolfram.com/DeltaFunction.html}{(Eq. 17 in link)}, we can proceed with equations below, \textbf{even if} derivatives are not finite and derive further results. In ~\ref{sec:Appendix_D_7}, we show that Statement 1 implies that \textbf{Statement A is false} and that $\omega_2(t_0)$ is a continuous function, which is differentiable at least twice.\\

In ~\ref{sec:appendix_F}, we derive the first 2 derivatives of $R(t_0)$ at $t_0=0$ as follows, where $m_0= M(0), m_2=[ \frac{d^2M(t_0)}{dt_0^2}]_{t_0=0}$ and $n_0= N(0), n_2=[ \frac{d^2N(t_0)}{dt_0^2}]_{t_0=0}$ and $ M(t_0) = \int_{-\infty}^{0}    E_0(\tau) e^{-2 \sigma \tau} \cos{ (\omega_2(t_0) \tau)} d\tau $ and $ N(t_0) = \int_{-\infty}^{0}    E_0(\tau) e^{-2 \sigma \tau} \sin{ (\omega_2(t_0) \tau)} d\tau $, $e_0= [E_0(t)]_{t_0=0}$ and $[\omega_{2}(t_0)]_{t_0=0}=\omega_{20}$. 

% $ m_0 = \int_{-\infty}^{0}    E_0(\tau) e^{-2 \sigma \tau} \cos{ (\omega_{20} \tau)} d\tau $, $ n_0 = \int_{-\infty}^{0}    E_0(\tau) e^{-2 \sigma \tau} \sin{ (\omega_{20} \tau)} d\tau$,\\ $ m_2 = -\omega_{22} \int_{-\infty}^{0}  \tau  E_0(\tau) e^{-2 \sigma \tau} \sin{ (\omega_{20} \tau)} d\tau $ and $ n_2 = \omega_{22}\int_{-\infty}^{0}    \tau E_0(\tau) e^{-2 \sigma \tau} \cos{ (\omega_{20} \tau)} d\tau$.

%We get $m_1=n_1=0$ because $\omega_2(t_0)$ is an \textbf{even} function of variable $t_0$.
%using Taylor series for $\omega_2(t_0)$ and we can write Eq.~\ref{sec_1_eq_7} as follows,

\begin{eqnarray*}\label{sec_2_2_eq_1} 
[ R(t_0) ]_{t_0=0} = m_0 \\
(\frac{dR(t_0)}{dt_0})_{t_0=0} =   e_0  + n_0  \omega_{20} + 2 \sigma m_0 \\
(\frac{d^2R(t_0)}{dt_0^2})_{t_0=0} =   m_2 + \sigma e_0  +2 \sigma n_0  \omega_{20}   + 2 \sigma^{2} m_0 - m_0 \frac{\omega_{20}^2}{2} 
\end{eqnarray*}
\begin{equation} \end{equation}

The equations for $m_0, m_2, n_0$  are described in ~\ref{sec:Appendix_F_4}. Given that $R(t_0)$ is an \textbf{odd function} of variable $t_0$, we get 

\begin{eqnarray*}\label{sec_2_2_eq_2}   
m_{0} = 0 \\
m_2 + \sigma e_0  +2 \sigma n_0  \omega_{20}   + 2 \sigma^{2} m_0 - m_0 \frac{\omega_{20}^2}{2} = 0, \quad m_2 + \sigma e_0  +2 \sigma n_0  \omega_{20} = 0 \\
m_0 =  \int_{-\infty}^{0}    E_0(\tau)e^{-2 \sigma \tau}  \cos{ (\omega_{20} \tau)} d\tau, \quad n_0 =  \int_{-\infty}^{0}    E_0(\tau)e^{-2 \sigma \tau}  \sin{ (\omega_{20} \tau)} d\tau \\
m_2 = - \omega_{22} \int_{-\infty}^{0}  \tau  E_0(\tau)e^{-2 \sigma \tau}  \sin{ (\omega_{20} \tau)}) d\tau , \quad e_0 = E_0(0)
\end{eqnarray*}
\begin{equation} \end{equation}

\subsection{\label{sec:Section_2_3} \textbf{Next Step } \protect\\  \lowercase{} }



If we replace $E_p(t)$ in above section by $E_{pp}(t)= e^{\sigma t_2} E_p(t+t_2) + e^{-\sigma t_2} E_p(t-t_2)= [E_0(t+t_2) + E_0(t-t_2) ] e^{-\sigma t}= E_{0}^{'}(t) e^{-\sigma t}$, for $|t_2| \leq \infty$, where $E_{0}^{'}(t)  = E_0(t+t_2) + E_0(t-t_2) $, the location of the zeros in Fourier transform of $g(t, t_0, t_2)$ are represented by $\omega_{2}^{'}(t_2, t_0)$  and using method in the above section, we can get results similar to Eq.~\ref{sec_2_2_eq_2} with $E_0(t)$ replaced by  $E_{0}^{'}(t)$ and $\omega_{20}$ replaced by $\omega_{20}^{'}(t_2)$ and other variables replaced with their \textbf{primed} versions as follows. We use $\omega_2^{'}(t_2, t_0) = w_{20}^{'}(t_2) + w_{22}^{'}(t_2) t_0^{2} +...$


\begin{eqnarray*}\label{sec_2_2_eq_3}   
m_{0}^{'}(t_2) = \int_{-\infty}^{0}    E_0^{'}(\tau)e^{-2 \sigma \tau}  \cos{ (\omega_{20}^{'}(t_2) \tau)} d\tau = 0 \\
m_2^{'}(t_2) + \sigma e_0^{'}(t_2)  + 2 \sigma n_0^{'}(t_2)  \omega_{20}^{'}(t_2) = 0 \\
n_0^{'}(t_2) =  \int_{-\infty}^{0}    E_0^{'}(\tau)e^{-2 \sigma \tau}  \sin{ (\omega_{20}^{'}(t_2) \tau)} d\tau \\
m_2^{'}(t_2) = - \omega_{22}^{'}(t_2) \int_{-\infty}^{0}  \tau  E_0^{'}(\tau)e^{-2 \sigma \tau}  \sin{ (\omega_{20}^{'}(t_2) \tau)}) d\tau , \quad e_0^{'}(t_2) = E_0^{'}(0) = E_0(t_2) + E_0(-t_2) 
\end{eqnarray*}
\begin{equation} \end{equation}

We use $E_{0}^{'}(t)  = E_0(t+t_2) + E_0(t-t_2) $ in Eq.~\ref{sec_2_2_eq_3} and then substitute $t+t_2=\tau$ for the first term and $t-t_2=\tau$ for the second term and get $m_{0}^{'}(t_2)$ as follows.

\begin{eqnarray*}\label{sec_2_2_eq_4}   
m_{0}^{'}(t_2) =   e^{ 2 \sigma t_2} [ \cos{ (\omega_{20}^{'}(t_2) t_2)} \int_{-\infty}^{t_2}    E_0(\tau)  e^{ - 2 \sigma \tau}  \cos{ ( \omega_{20}^{'}(t_2) \tau)} d\tau + \sin{ (\omega_{20}^{'}(t_2) t_2)}  \int_{-\infty}^{t_2}  E_0(\tau)  e^{ - 2 \sigma \tau} \sin{ (\omega_{20}^{'}(t_2) \tau)} d\tau ] \\
+  e^{ -2 \sigma t_2} [ \cos{ (\omega_{20}^{'}(t_2) t_2)} \int_{-\infty}^{-t_2}    E_0(\tau)  e^{ - 2 \sigma \tau}  \cos{ ( \omega_{20}^{'}(t_2) \tau)} d\tau - \sin{ (\omega_{20}^{'}(t_2) t_2)}  \int_{-\infty}^{-t_2}  E_0(\tau)  e^{ - 2 \sigma \tau} \sin{ (\omega_{20}^{'}(t_2) \tau)} d\tau ]  = 0 \\
m_{0}^{'}(t_2) = R'(t_2) + R'(-t_2) = 0 \\
R'(t_2) =  e^{ 2 \sigma t_2} [ \cos{ (\omega_{20}^{'}(t_2) t_2)} \int_{-\infty}^{t_2}    E_0(\tau)  e^{ - 2 \sigma \tau}  \cos{ ( \omega_{20}^{'}(t_2) \tau)} d\tau + \sin{ (\omega_{20}^{'}(t_2) t_2)}  \int_{-\infty}^{t_2}  E_0(\tau)  e^{ - 2 \sigma \tau} \sin{ (\omega_{20}^{'}(t_2) \tau)} d\tau ] 
\end{eqnarray*}
\begin{equation} \end{equation}

We compare Eq.~\ref{sec_2_2_eq_4} with Eq.~\ref{sec_1_eq_7} and see that $R(t_0)$ and $R'(t_2)$ are similar equations, with $t_0, \omega_{2}(t_0)$ replaced by $t_2, \omega_{20}^{'}(t_2)$  and hence both equations \textbf{must have at least one} common solution with $\omega_{2}(t_0) = \omega_{20}^{'}(t_2)$. Hence we replace $\omega_{20}^{'}(t_2)$ in Eq.~\ref{sec_2_2_eq_3} with $\omega_{2}(t_2)$ and use $E_{0}^{'}(t)  = E_0(t+t_2) + E_0(t-t_2) $ and write as follows.


\begin{eqnarray*}\label{sec_2_2_eq_5}   
n_{0}^{'}(t_2) = n_{0p}(t_2) + n_{0p}(-t_2)  \\
n_{0p}(t_2) =   e^{ 2 \sigma t_2} [ \cos{ (\omega_{2}(t_2) t_2)} \int_{-\infty}^{t_2}    E_0(\tau)  e^{ - 2 \sigma \tau}  \sin{ ( \omega_{2}(t_2) \tau)} d\tau - \sin{ (\omega_{2}(t_2) t_2)}  \int_{-\infty}^{t_2}  E_0(\tau)  e^{ - 2 \sigma \tau} \cos{ (\omega_{2}(t_2) \tau)} d\tau ]  \\
m_2^{'}(t_2) =  m_{2p}(t_2) + m_{2p}(-t_2)  \\
m_{2p}(t_2) =  - \frac{1}{2} \frac{d^2 \omega_{2}(t_2)}{dt_2^2}  e^{ 2 \sigma t_2} [ \cos{ (\omega_{2}(t_2) t_2)} \int_{-\infty}^{t_2} (\tau - t_2)   E_0(\tau)  e^{ - 2 \sigma \tau}  \sin{ ( \omega_{2}(t_2) \tau)} d\tau \\- \sin{ (\omega_{2}(t_2) t_2)}  \int_{-\infty}^{t_2} (\tau - t_2)   E_0(\tau)  e^{ - 2 \sigma \tau} \cos{ (\omega_{2}(t_2) \tau)} d\tau ] \\
e_0^{'}(t_2) =  E_0(t_2) + E_0(-t_2) \\
A(t_2) = m_2^{'}(t_2) + \sigma e_0^{'}(t_2)  + 2 \sigma n_0^{'}(t_2)  \omega_{2}(t_2) = 0 \\
\end{eqnarray*}
\begin{equation} \end{equation}

The term $\frac{d^2 \omega_{2}(t_0)}{dt_0^2}$ in Eq.~\ref{sec_2_2_eq_5} is obtained as follows. We see that $f^{'}(t)=  e^{\sigma t_0} E_{pp}(t+t_0) + e^{-\sigma t_0} E_{pp}(t-t_0)$ remains the \textbf{same}, when we \textbf{interchange} $t_0$ with $t_2$, where $E_{pp}(t)= e^{\sigma t_2} E_p(t+t_2) + e^{-\sigma t_2} E_p(t-t_2)$. Because the Fourier transform of $f^{'}(t)$ given by $F^{'}(\omega)=E_{pp\omega}(\omega)(e^{\sigma t_0} e^{i \omega t_0} + e^{-\sigma t_0} e^{-i \omega t_0}  ) = E_{p\omega}(\omega) (e^{\sigma t_2} e^{i \omega t_2} + e^{-\sigma t_2} e^{-i \omega t_2} ) (e^{\sigma t_0} e^{i \omega t_0} + e^{-\sigma t_0} e^{-i \omega t_0}  )  $ remains the \textbf{same}, when we \textbf{interchange} $t_0$ with $t_2$.\\

Hence $\omega_{2}(t_2, t_0) = \omega_{2}(t_0, t_2)$. The second derivative is given by $\frac{d^2 \omega_{2}(t_2, t_0)}{dt_0^2} = \frac{d^2 \omega_{2}(t_0, t_2)}{dt_2^2} $. In Eq.~\ref{app_F_4_eq_1}, we computed $\omega_{22}$ by evaluating  $\frac{1}{2} \frac{d^2 \omega_{2}(t_0)}{dt_0^2}$ at $t_0=0$ to obtain $m_2$.  Similarly, we compute $\omega_{22}^{'}(t_2)$ in Eq.~\ref{sec_2_2_eq_3} , by evaluating the term $\frac{1}{2}  \frac{d^2 \omega_{2}(t_2, t_0)}{dt_0^2}$ at $t_0=0$, hence this is the \textbf{same} as evaluating $\frac{1}{2} \frac{d^2 \omega_{2}(t_0, t_2)}{dt_2^2}$ at $t_0=0$ which equals $\frac{1}{2} \frac{d^2 \omega_{2}(t_2)}{dt_2^2}$. in Eq.~\ref{sec_2_2_eq_5}. \\
%Then we replace $t_2$ with $t_2$ and get the term $\frac{d^2 \omega_{2}(t_2)}{dt_2^2}$ in Eq.~\ref{sec_2_2_eq_5}.\\

%Similarly, we can write the equation for $m_1^{'}(t_2)$ as follows. In Eq.~\ref{app_F_4_eq_1}, we see that \\ $m_1 = [\frac{d\omega_{2}(t_0)}{dt_0}]_{t_0=0} = - \omega_{21} \int_{-\infty}^{0}  \tau  E_0(\tau)e^{-2 \sigma \tau}  \sin{ (\omega_{20} \tau)}   d\tau = 0$ because $\frac{d\omega_{2}(t_0)}{dt_0}$ is an odd function.

%\begin{eqnarray*}\label{sec_2_2_eq_6}   
%m_1^{'}(t_2) =  m_{1p}(t_2) + m_{1p}(-t_2) = 0 \\
%m_{1p}(t_2) =  -  \frac{d\omega_{2}(t_2)}{dt_2}  e^{ 2 \sigma t_2} [ \cos{ (\omega_{2}(t_2) t_2)} \int_{-\infty}^{t_2} (\tau - t_2)   E_0(\tau)  e^{ - 2 \sigma \tau}  \sin{ ( \omega_{2}(t_2) \tau)} d\tau \\- \sin{ (\omega_{2}(t_2) t_2)}  \int_{-\infty}^{t_2} (\tau - t_2)   E_0(\tau)  e^{ - 2 \sigma \tau} \cos{ (\omega_{2}(t_2) \tau)} d\tau ] 
%\end{eqnarray*}
%\begin{equation} \end{equation}




%\clearpage
\subsection{\label{sec:Section_2_4} \textbf{ Asymptotic Case and Final result} \protect\\  \lowercase{} }



In Section~\ref{sec:Section_3_1}, we show that $\lim_{t_2 \to \infty} g(t)$ is an \textbf{analytic} function, with the \textbf{magnitude} of the step discontinuity at $t=0$ \textbf{decreasing to zero}, and its Fourier transform is an analytic function with \textbf{isolated zeros}, as $\lim_{t_2 \to \infty}$. Hence $\lim_{t_2 \to \infty} \omega_{2}(t_2) = \omega_z \neq 0$ which is a constant and $\lim_{t_2 \to \infty} \frac{d^2 \omega_{2}(t_2)}{dt_2^2} = 0$. Hence $\lim_{t_2 \to \infty} m_2^{'}(t_2) = 0$. We see that $\lim_{t_2 \to \infty} e_0^{'}(t_2)  = 0$ and  $\lim_{t_2 \to \infty} n_{0p}(-t_2), m_{2p}(-t_2) = 0$ and we write Eq.~\ref{sec_2_2_eq_5} as follows given $\sigma,\omega_z \neq 0 $. \\


\begin{eqnarray*}\label{sec_2_2_eq_6}   
\lim_{t_2 \to \infty} A(t_2) = \lim_{t_2 \to \infty} 2 \sigma  \omega_z n_0^{'}(t_2) =  0 \\
\lim_{t_2 \to \infty} n_0^{'}(t_2) = \lim_{t_2 \to \infty}  e^{ 2 \sigma t_2} [ \cos{ (\omega_{2}(t_2) t_2)} \int_{-\infty}^{t_2}    E_0(\tau)  e^{ - 2 \sigma \tau}  \sin{ ( \omega_{2}(t_2) \tau)} d\tau - \sin{ (\omega_{2}(t_2) t_2)}  \int_{-\infty}^{t_2}  E_0(\tau)  e^{ - 2 \sigma \tau} \cos{ (\omega_{2}(t_2) \tau)} d\tau ]   = 0 \\
\lim_{t_2 \to \infty}  [ \cos{ (\omega_{2}(t_2) t_2)} \int_{-\infty}^{t_2}    E_0(\tau)  e^{ - 2 \sigma \tau}  \sin{ ( \omega_{2}(t_2) \tau)} d\tau - \sin{ (\omega_{2}(t_2) t_2)}  \int_{-\infty}^{t_2}  E_0(\tau)  e^{ - 2 \sigma \tau} \cos{ (\omega_{2}(t_2) \tau)} d\tau ]  = 0
\end{eqnarray*}
\begin{equation} \end{equation}

Similarly, we can write Eq.~\ref{sec_2_2_eq_4} in the asymptotic case $\lim_{t_2 \to \infty}$ as follows.

\begin{eqnarray*}\label{sec_2_2_eq_7}   
\lim_{t_2 \to \infty} m_0^{'}(t_2) = \lim_{t_2 \to \infty}  e^{ 2 \sigma t_2} [ \cos{ (\omega_{2}(t_2) t_2)} \int_{-\infty}^{t_2}    E_0(\tau)  e^{ - 2 \sigma \tau}  \cos{ ( \omega_{2}(t_2) \tau)} d\tau \\+ \sin{ (\omega_{2}(t_2) t_2)}  \int_{-\infty}^{t_2}  E_0(\tau)  e^{ - 2 \sigma \tau} \sin{ (\omega_{2}(t_2) \tau)} d\tau ]   = 0 
\end{eqnarray*}
\begin{equation} \end{equation}


If we write $I_1(t_2) =  \int_{-\infty}^{t_2}     E_{0}( \tau)  e^{ -2 \sigma \tau}  \cos{ (\omega_z \tau)} d\tau$ and $I_2(t_2) =  \int_{-\infty}^{t_2}  E_{0}( \tau)  e^{ -2 \sigma \tau}  \sin{ (\omega_z \tau)} d\tau$, and $\lim_{t_2 \to \infty} (\omega_2(t_2) = \omega_z $ we can write
Eq.~\ref{sec_2_2_eq_6}  and Eq.~\ref{sec_2_2_eq_7} as follows.

\begin{eqnarray*}\label{sec_1_1_eq_4}   
\lim_{t_2 \to \infty}  \cos{ ( \omega_z t_2)}  I_2(t_2) - \lim_{t_2 \to \infty}  \sin{ ( \omega_z t_2)}  I_1(t_2)  = 0 \\
\lim_{t_2 \to \infty}  \cos{ ( \omega_z t_2)}  I_1(t_2) + \lim_{t_2 \to \infty}  \sin{ ( \omega_z t_2)}  I_2(t_2) = 0 \\
\lim_{t_2 \to \infty} \frac{I_2(t_2)}{I_1(t_2)} = \lim_{t_2 \to \infty} \frac{ \sin{ ( \omega_z t_2)}}{ \cos{ ( \omega_z t_2)}} = \lim_{t_2 \to \infty} -\frac{I_1(t_2)}{I_2(t_2)} 
\end{eqnarray*}
\begin{equation} \end{equation}  

For the general case of $\lim_{t_2 \to \infty} \frac{ \sin{ ( \omega_z t_2)}}{ \cos{ ( \omega_z t_2)}} \neq 0, \pm \infty$, we get $\lim_{t_2 \to \infty} (I_1(t_2)^{2} + I_2(t_2)^{2}) = 0$. This implies that $\lim_{t_2 \to \infty} I_1(t_2)= \lim_{t_2 \to \infty}I_2(t_2) = 0$ and $\int_{-\infty}^{\infty}     E_0(\tau) e^{-2 \sigma \tau} e^{-i  \omega_z \tau} d\tau = 0$. \\

We started with \textbf{Statement 1} that the Fourier Transform of the function $E_p(t) = E_0(t) e^{-\sigma t} $ has a zero at $\omega = \omega_{0}$ which means that $\int_{-\infty}^{\infty}    E_0(\tau) e^{- \sigma \tau} e^{-i \omega_0 \tau} d\tau = 0$ and we derived the result that $\int_{-\infty}^{\infty}    E_0(\tau) e^{-2 \sigma \tau} e^{-i  \omega_z \tau} d\tau = 0$.\\

Now we can repeat the steps in Section 2, starting with the new result that $\int_{-\infty}^{\infty}    E_0(\tau) e^{-2 \sigma \tau} e^{-i \omega_z \tau} d\tau = 0$ and $\sigma$ replaced by $2 \sigma$ and derive the next result that $\int_{-\infty}^{\infty}    E_0(\tau) e^{-4 \sigma \tau} e^{-i \omega_{(z1)} \tau} d\tau = 0$.\\

We can repeat above steps N times till $(2^{N+1} \sigma) > \frac{1}{2}$ and get the result $\int_{-\infty}^{\infty}    E_0(\tau) e^{-(2^{N+1} \sigma) \tau} e^{-i \omega_{(zN)} \tau} d\tau = 0$. In each iteration $n$, we use $h(t)=  e^{ (2^{N+1} \sigma) t} u(-t) + e^{ - 3*(2^{N+1} \sigma) t} u(t) $, $\omega_2(t_2)$ replaced by $\omega_{2n}(t_2)$ and $\omega_z$ replaced by $\omega_{(zn)}$. We know that  the Fourier Transform of $E_{0}(t) e^{-(2^{N+1} \sigma) t} =    \sum_{n=1}^{\infty}  [ 4 \pi^{2} n^{4} e^{4t}    - 6 \pi n^{2}   e^{2t} ]  e^{- \pi n^{2} e^{2t}} e^{\frac{t}{2}} e^{-(2^{N+1} \sigma) t}$ given by $E_{p\omega N}(\omega)=\xi(\frac{1}{2}+ 2^N \sigma + i \omega)$ \textbf{does not} have a real zero for $(2^{N+1} \sigma) > \frac{1}{2}$ , corresponding to $Re[s] > 1$. \\% for  $0 < |\sigma| < \frac{1}{2}$ Here we use the well known fact that $E_0(t)=E_0(-t)$. \\

We have shown this result for $0 < \sigma < \frac{1}{2}$ and then use the property $\xi(\frac{1}{2} + \sigma + i \omega) = \xi(\frac{1}{2} - \sigma - i \omega)$ to show the result for $-\frac{1}{2} < \sigma < 0$. Hence we have produced a \textbf{contradiction} of  \textbf{Statement 1} that the Fourier Transform of the function $E_p(t) = E_0(t) e^{-\sigma t} $ has a zero at $\omega = \omega_{0}$ for  $0 < |\sigma| < \frac{1}{2}$.


\clearpage
\subsection{\label{sec:Section_3_1} \textbf{ Analytic Functions and Isolated Zeros.  } \protect\\  \lowercase{} }
%$\lim_{t_0 \to \infty} \omega_{2}(t_0) = \omega_z$ becomes a continuous function.

In this section, we show that $\lim_{t_0 \to \infty} g(t)$ is an analytic function, with the magnitude of the step discontinuity at $t=0$ decreasing to zero, and its Fourier transform is an analytic function with isolated zeros, as $\lim_{t_0 \to \infty}$. Hence $\lim_{t_0 \to \infty} \omega_{2}(t_0) = \omega_z \neq 0$ which is a constant and $\lim_{t_0 \to \infty} \frac{d^2 \omega_{2}(t_0)}{dt_0^2} = 0$. \\

We see that $g(t) = E_0^{'}(t) e^{-2 \sigma t} u(-t) + E_0^{'}(t) e^{2 \sigma t} u(t)  $ where $E_0^{'}(t) = E_0^{'}(-t)  =  E_0(t + t_0) +  E_0(t - t_0)$ and its first derivative has a \textbf{step} discontinuity at $t=0$ with magnitude $\Delta_d = 4 \sigma E_0^{'}(0) = 4 \sigma (E_0(t_0) + E_0(-t_0))$. As $\lim_{t_0 \to \infty}$, $\Delta_d \to 0$ because $E_0(t_0)$ and $E_0(-t_0)$ decrease to zero as  $\lim_{t_0 \to \infty}$ and hence $\lim_{t_0 \to \infty} g(t) =\lim_{t_0 \to \infty} [  E_0^{'}(t) e^{-2 \sigma t}  + E_0^{'}(t) e^{2 \sigma t} ]$  is an \textbf{analytic} function.\\

We use a \textbf{scale factor} and get $g_s(t) = g(t)  e^{-2 \sigma t_0}$, so that $\lim_{t_0 \to \infty} g_s(t)$ remains \textbf{finite} for all $|t| \leq \infty$. This scale factor \textbf{does not} affect the location of zeros in the Fourier transform of $g(t)$ and $g_s(t)$. Hence $\lim_{t_0 \to \infty} g_s(t) =\lim_{t_0 \to \infty} E_0^{'}(t) [ e^{-2 \sigma t}  +  e^{2 \sigma t} ] e^{-2 \sigma t_0} = E_0(t + t_0) e^{-2 \sigma (t+t_0)} + E_0(t - t_0) e^{2 \sigma (t-t_0)} + E_0(t - t_0) e^{-2 \sigma (t+t_0)} + E_0(t + t_0) e^{2 \sigma (t-t_0)} $.\\

The Fourier transform of $g_s(t)$ is given by $G_s(\omega)$ and $\lim_{t_0 \to \infty} G_s(\omega) = E_{0\omega}(\omega - i 2 \sigma) e^{i \omega t_0} + E_{0\omega}(\omega + i 2 \sigma) e^{-i \omega t_0} +  E_{0\omega}(\omega - i 2 \sigma) e^{-i \omega t_0}  e^{-4 \sigma t_0} + E_{0\omega}(\omega + i 2 \sigma) e^{i \omega t_0}  e^{-4 \sigma t_0} $. As  $\lim_{t_0 \to \infty}$, the last two terms in $\lim_{t_0 \to \infty} G_s(\omega)$ go to zero.\\

Hence $\lim_{t_0 \to \infty} G_s(\omega) = E_{0\omega}(\omega - i 2 \sigma) e^{i \omega t_0} + E_{0\omega}(\omega + i 2 \sigma) e^{-i \omega t_0}$ is an \textbf{analytic function} for all $|\omega| \leq \infty$ because it is derived from the \textbf{entire function} $\xi(s)$ and we know that $E_{0\omega}(\omega) = \xi(\frac{1}{2} + i \omega)$. The same statement holds for $\lim_{t_0 \to \infty} G(\omega)$ which differs only by a scale factor $ e^{-2 \sigma t_0}$.\\

We use the well known result that \textbf{analytic} functions have \textbf{isolated zeros}.\href{https://proofwiki.org/wiki/Zeroes_of_Analytic_Function_are_Isolated}{(link)} Hence $\lim_{t_0 \to \infty} G_s(\omega)$ and $\lim_{t_0 \to \infty} G(\omega)$ have \textbf{isolated zeros} at $\omega=\omega_2(t_0) = \omega_z$ and the \textbf{second derivative}  given by $\lim_{t_0 \to \infty} \frac{d^2 \omega_{2}(t_0)}{dt_0^2} = 0$.

\subsubsection{\label{sec:Section_3_1} \textbf{  Isolated Zeros are single valued. } \protect\\  \lowercase{} }

We consider $g_s(t) = g(t)  e^{-2 \sigma t_0}$ and see that $\lim_{t_0 \to \infty} g_s(t) = E_0(t + t_0) e^{-2 \sigma (t+t_0)} + E_0(t - t_0) e^{2 \sigma (t-t_0)} + E_0(t - t_0) e^{-2 \sigma (t+t_0)} + E_0(t + t_0) e^{2 \sigma (t-t_0)} $ is an \textbf{analytic} function, whose Fourier transform is given by $\lim_{t_0 \to \infty} G_s(\omega) = E_{0\omega}(\omega - i 2 \sigma) e^{i \omega t_0} + E_{0\omega}(\omega + i 2 \sigma) e^{-i \omega t_0}$ which is derived from the \textbf{entire} function $\xi(s)$ where $E_{0\omega}(\omega) = \xi(\frac{1}{2} + i \omega)$.\\

We know that \textbf{analytic} functions have \textbf{isolated zeros} \href{https://proofwiki.org/wiki/Zeroes_of_Analytic_Function_are_Isolated}{(link)} and each isolated zero has a \textbf{single value}.  For example, the analytic function $E_{0\omega}(\omega) = \xi(\frac{1}{2} + i \omega)$ corresponding to the\textbf{ critical line}, is well known to have isolated zeros and each isolated zero  has a \textbf{single value}. Hence we can expect the analytic function $\lim_{t_0 \to \infty} G_s(\omega) $ to  have \textbf{isolated zeros} and each isolated zero to have a \textbf{single value}. Hence $\lim_{t_0 \to \infty}  \omega_2(t_0)$ is \textbf{not }a pathological function with multiple values or ill-defined, but $\lim_{t_0 \to \infty}  \omega_2(t_0) = \omega_z$  is a well defined constant.

%=\lim_{t_0 \to \infty} [  E_0^{'}(t) e^{-2 \sigma t}  + E_0^{'}(t) e^{2 \sigma t} ] e^{-2 \sigma t_0} = \lim_{t_0 \to \infty} E_1(t + t_0) + E_2(t - t_0) $ where $E_0^{'}(t) = E_0(t + t_0) +  E_0(t - t_0)$ and $E_1(t)=E_0(t)  e^{-2 \sigma t} and $E_2(t)=E_0(t)  e^{2 \sigma t}$. Hence $\lim_{t_0 \to \infty} g_s(t)$ 

%the step discontinuity at $t=0$ decreases to zero and we can write $\lim_{t_0 \to \infty} g(t) =\lim_{t_0 \to \infty} [  E_0^{'}(t) e^{-2 \sigma t}  + E_0^{'}(t) e^{2 \sigma t} ] $.

%\clearpage




 
\begin{thebibliography}{15}

\bibitem{BR} Bernhard Riemann, On the Number of Prime Numbers less than a Given Quantity.(Ueber die Anzahl der Primzahlen untereiner gegebenen Grosse.) Monatsberichte der Berliner Akademie,November 1859. \href{https://www.claymath.org/sites/default/files/ezeta.pdf}{(Link to Riemann's 1859 paper)}


\bibitem{GH} Hardy, G.H., Littlewood, J.E. The zeros of Riemann's zeta-function on the critical line. Mathematische Zeitschrift volume  10, pp.283 to 317 (1921).
%G. H. Hardy, Sur les zeros de la fonction ζ(s),Comp. Rend. Acad. Sci.158(1914) 
 
\bibitem{ECT} E. C. Titchmarsh, The Theory of the Riemann Zeta Function. (1986) pp.254 to 255 

\bibitem{FWE} Fern Ellison and William J. Ellison, Prime Numbers  (1985).pp147 to 152

\bibitem{JBC}  J. Brian Conrey, The Riemann Hypothesis  (2003). \href{https://www.ams.org/notices/200303/fea-conrey-web.pdf}{(Link to Brian Conrey's 2003 article)}

\bibitem{WK1}  Mathworld article on Hurwitz Zeta functions. \href{https://mathworld.wolfram.com/HurwitzZetaFunction.html}{(Link)}

\bibitem{WK2}  Wikipedia article on Dirichlet L-functions. \href{https://en.wikipedia.org/wiki/Dirichlet_L-function#Functional_equation}{(Link)}



\end{thebibliography}




%\clearpage
%\section{\label{sec:appendix_A} Appendix A \protect\\  \lowercase{} }
\appendix\section{\label{sec:appendix_A} \textbf{Derivation of $E_p(t)$} \protect\\  \lowercase{} }
%\section{appendix_A}\label{sec:appendix_A}

Let us start with Riemann's Xi Function $\xi(s)$ evaluated at $s = \frac{1}{2} + i \omega$ given by $\xi(\frac{1}{2} + i \omega)= E_{0\omega}(\omega)$. Its inverse Fourier Transform is given by $ E_0(t)=  \frac{1}{2 \pi} \int_{-\infty}^{\infty} E_{0\omega}(\omega) e^{i\omega t} d\omega = 2 \sum_{n=1}^{\infty}  [ 2 \pi^{2} n^{4} e^{4t}    - 3 \pi n^{2}   e^{2t} ]  e^{- \pi n^{2} e^{2t}} e^{\frac{t}{2}}$ \href{https://www.ams.org/notices/200303/fea-conrey-web.pdf#page=5}{(link)}. This is re-derived in ~\ref{sec:appendix_H}.\\


We will show in this section that the inverse Fourier Transform of the function  $\xi(\frac{1}{2} + \sigma + i \omega) =  E_{p\omega}(\omega)$,  is given by $E_{p}(t) =   E_{0}(t) e^{-\sigma t}$  where $0 \leq |\sigma| < \frac{1}{2}$ is real.

\begin{eqnarray*}\label{sec_app_A_1_eq_1}
\xi(\frac{1}{2} + \sigma + i \omega) = \xi(\frac{1}{2} + i (\omega - i \sigma) ) = E_{p\omega}(\omega) = E_{0\omega}(\omega - i \sigma) \\
E_{p}(t) =   \frac{1}{2 \pi} \int_{-\infty}^{\infty} E_{p\omega}(\omega) e^{i\omega t} d\omega =   \frac{1}{2 \pi} \int_{-\infty}^{\infty} E_{0\omega}(\omega - i \sigma) e^{i\omega t} d\omega 
\end{eqnarray*}
\begin{equation} \end{equation}

 We substitute $\omega' = \omega - i \sigma$ in Eq.~\ref{sec_app_A_1_eq_1} as follows.
 
\begin{eqnarray*}\label{sec_app_A_1_eq_1_1}
E_{p}(t) =    e^{-\sigma t}  \frac{1}{2 \pi} \int_{-\infty  - i \sigma}^{\infty  - i \sigma} E_{0\omega}(\omega') e^{i\omega' t} d\omega'
\end{eqnarray*}
\begin{equation} \end{equation}

We can evaluate the above integral in the complex plane using contour integration, substituting $\omega' = z = x + i y$ and we use a rectangular contour comprised of $C_1$ along the line $ x = [-\infty, \infty]$, $C_2$ along the line $y = [\infty, \infty-i\sigma]$, $C_3$ along the line $x = [ \infty-i\sigma, -\infty-i\sigma]$ and then $C_4$ along the line $y = [-\infty-i\sigma, -\infty]$. We can see that $E_{0\omega}(z)=\xi(\frac{1}{2}+ i z)$ has no singularities in the region bounded by the contour because $\xi(\frac{1}{2}+ i z)$ is an entire function in the Z-plane.\\

%Given that $\xi(\frac{1}{2} + \sigma + i \omega)= E_{p\omega}(\omega)$ is an entire function in the whole of s-plane, it is finite for $|\omega| \leq \infty$ and also for $\omega=0$. Hence $\int_{-\infty}^{\infty} E_p(t) dt$ is finite. 
In \textbf{~\ref{sec:appendix_C_1}}, we show that $\int_{-\infty}^{\infty} |E_p(t)| dt$ is finite and $E_p(t)= E_0(t) e^{-\sigma t} $ is an  absolutely integrable function, for $0 \leq |\sigma| < \frac{1}{2}$.\\

We use the fact that $E_{0\omega}(z) = \xi(\frac{1}{2}+ i z) = \xi(\frac{1}{2} -y + i x) =  \int_{-\infty}^{\infty} E_0(t) e^{-i z t} dt  =  \int_{-\infty}^{\infty} E_0(t) e^{y t} e^{-i x t} dt $, \textbf{goes to zero} as $x \to \pm \infty$ when $-\sigma \leq y \leq 0$, as per Riemann-Lebesgue Lemma \href{https://en.wikipedia.org/wiki/Riemann%E2%80%93Lebesgue_lemma}{(link)}, because $E_0(t) e^{y t}$ is a absolutely integrable function  in the interval $-\infty \leq t \leq \infty$. Hence the integral in Eq.~\ref{sec_app_A_1_eq_1_1} \textbf{vanishes }along  the contours $C_2$ and $C_4$. Using Cauchy's Integral theroem, we can write Eq.~\ref{sec_app_A_1_eq_1_1} as follows.

\begin{eqnarray*}\label{sec_app_A_1_eq_2}
E_{p}(t) =    e^{-\sigma t}  \frac{1}{2 \pi} \int_{-\infty}^{\infty} E_{0\omega}(\omega') e^{i\omega' t} d\omega' \\
E_{p}(t) =    E_{0}(t) e^{-\sigma t} = 2 \sum_{n=1}^{\infty}  [ 2 \pi^{2} n^{4} e^{4t}    - 3 \pi n^{2}   e^{2t} ]  e^{- \pi n^{2} e^{2t}} e^{\frac{t}{2}} e^{-\sigma t} 
\end{eqnarray*}
\begin{equation} \end{equation}

Thus we have arrived at the desired result $E_{p}(t) =   E_{0}(t) e^{-\sigma t}$. \textbf{Alternate} derivation is in ~\ref{sec:appendix_H_1}.









\section{\label{sec:appendix_H} \textbf{Derivation of entire function $\xi(s)$ } \protect\\  \lowercase{} }


%Let us start with Riemann's Xi Function $\xi(\frac{1}{2} + i \omega)= E_{0\omega}(\omega)$. Its inverse Fourier Transform is given by $ E_0(t)=  \frac{1}{2 \pi} \int_{-\infty}^{\infty} E_{0\omega}(\omega) e^{i\omega t} d\omega $.\\

In this section, we will re-derive Riemann's Xi function $\xi(s)$ and the inverse Fourier Transform of  $\xi(\frac{1}{2} + i \omega)= E_{0\omega}(\omega)$ and show the result $E_{0}(t) =   2 \sum_{n=1}^{\infty}  [ 2 \pi^{2} n^{4} e^{4t}    - 3 \pi n^{2}   e^{2t} ]  e^{- \pi n^{2} e^{2t}} e^{\frac{t}{2}} $. \\
 
We will use the steps in Ellison's book "Prime Numbers" pages 151-152 and re-derive the steps below$^{\citep{FWE}}$ \href{https://www.ocf.berkeley.edu/~araman/files/math_z/Ellison_p147-152.pdf#page=5}{(link)}. We start with the gamma function $\Gamma(s) = \int_{0}^{\infty} y^{s-1} e^{-y} dy$ and substitute $y = \pi n^{2} x$ and derive as follows. %We define $s= \frac{1}{2} + \sigma + i \omega$.

\begin{eqnarray*}\label{sec:App_H_eq_1}   
\Gamma(\frac{s}{2}) = \int_{0}^{\infty} y^{\frac{s}{2}-1} e^{-y} dy \\
\Gamma(\frac{s}{2})  (\pi n^{2})^{-\frac{s}{2}} = \int_{0}^{\infty} x^{\frac{s}{2}-1} e^{-\pi n^{2} x} dx
\end{eqnarray*}
\begin{equation} \end{equation}
 
%For $\sigma > 1$, 
For real part of $s$ greater than 1, we can do a  summation of both sides of above equation for all positive integers $n$ and obtain
as follows. We note that $\zeta(s)= \sum_{n=1}^{\infty} \frac{1}{n^{s}} $. 
 
\begin{eqnarray*}\label{sec:App_H_eq_2}   
\Gamma(\frac{s}{2}) \pi^{-\frac{s}{2}} \zeta(s) = \sum_{n=1}^{\infty} \int_{0}^{\infty} x^{\frac{s}{2}-1} e^{-\pi n^{2} x} dx
\end{eqnarray*}
\begin{equation} \end{equation}
 
For real part of $s$ ($\sigma'$) greater than 1, we can use theorem of dominated convergence and interchange the order of summation and integration as follows.  We use the fact that  $w(x) =  \displaystyle\sum\limits_{n=1}^{\infty} e^{- \pi n^{2}x }$ and \\
$ \displaystyle\sum_{n=1}^{\infty} \int_{0}^{\infty}| x^{\frac{s}{2}-1} e^{-\pi n^{2} x} | dx = \Gamma(\frac{\sigma'}{2}) \pi^{-\frac{\sigma'}{2}} \zeta(\sigma')$.

 \begin{eqnarray*} \label{sec:App_H_eq_3}   
\Gamma(\frac{s}{2}) \pi^{-\frac{s}{2}} \zeta(s) = \int_{0}^{\infty} x^{\frac{s}{2}-1} w(x)  dx
\end{eqnarray*}
\begin{equation} \end{equation}

For real part of $s$ less than or equal to 1, $\zeta(s)$ \textbf{diverges}. Hence we do the following. In Eq.~\ref{sec:App_H_eq_3}, first we consider real part of $s$ greater than 1 and we divide the range of integration into two parts: $(0,1]$ and $[1, \infty)$ and make the substitution $x \rightarrow \frac{1}{x}$ in the first interval $(0,1]$. We use \textbf{the well known theorem} $ 1 + 2 w(x) = \frac{1}{\sqrt{x}} (1 + 2 w(\frac{1}{x}) )$, where $x > 0$ is real.$^{\citep{FWE}}$


\begin{eqnarray*} \label{sec:App_H_eq_3_a_1}   
\Gamma(\frac{s}{2}) \pi^{-\frac{s}{2}} \zeta(s) = \int_{1}^{\infty} x^{\frac{s}{2}-1} w(x)  dx +  \int_{1}^{\infty} \frac{x^{-(\frac{s}{2}-1)}}{x^2} \frac{(1 + 2 w(x)) \sqrt{x} - 1)}{2}  dx
 \end{eqnarray*}
\begin{equation} \end{equation}

Hence we can simplify Eq.~\ref{sec:App_H_eq_3_a_1} as follows.

\begin{eqnarray*}  \label{sec:App_H_eq_3_a_2}   
\Gamma(\frac{s}{2}) \pi^{-\frac{s}{2}} \zeta(s) = \frac{1}{s (s-1)} + \int_{1}^{\infty} x^{\frac{s}{2}-1} w(x)  dx + \int_{1}^{\infty} x^{\frac{-(s+1)}{2}} w(x)  dx
\end{eqnarray*}
\begin{equation}  \end{equation}


We multiply above equation by $\frac{1}{2} s (s-1)$ and get

\begin{eqnarray*} \label{sec:App_H_eq_3_1} 
\xi(s) = \frac{1}{2} s (s-1) \Gamma(\frac{s}{2}) \pi^{-\frac{s}{2}} \zeta(s) =  \frac{1}{2} [ 1 +  s (s-1)  \int_{1}^{\infty} ( x^{\frac{s}{2}} + x^{\frac{1-s}{2}}) w(x)  \frac{dx}{x}  ]
\end{eqnarray*}
\begin{equation}  \end{equation}

We see that $\xi(s)$ is an entire function, for all values of  $Re[s]$ in the complex plane and hence we get an analytic continuation of $\xi(s)$ over the entire complex plane. We see that $\xi(s) = \xi(1-s)$  $^{\citep{FWE}}$.\\
%the analytic continuation of $\zeta(s)$s

\subsection{\label{sec:appendix_H_1} \textbf{Derivation of $E_p(t)$ and $E_0(t)$ } \protect\\  \lowercase{} }


Given that $w(x) = \sum\limits_{n=1}^{\infty} e^{- \pi n^{2}x }$, we substitute $x= e^{2t}, \frac{dx}{x}= 2 dt$ in Eq.~\ref{sec:App_H_eq_3_1} and evaluate at  $s= \frac{1}{2} + \sigma + i \omega$ as follows.

\begin{equation}  \label{sec:App_H_eq_4} 
\xi(\frac{1}{2}  + \sigma + i \omega) =   \frac{1}{2} [ 1 +  2 (\frac{1}{2} + \sigma + i \omega) (-\frac{1}{2} + \sigma + i \omega)  \int_{0}^{\infty}  \sum\limits_{n=1}^{\infty} e^{- \pi n^{2} e^{2t} }  ( e^{\frac{t}{2}} e^{\sigma t} e^{ i \omega t} +  e^{\frac{t}{2}} e^{-\sigma t} e^{ -i \omega t} )  dt   ]
%\begin{eqnarray*} \end{eqnarray*}
\end{equation}

We can substitute $t=-t$ in the first term in above integral and simplify above equation as follows.

\begin{eqnarray*}\label{sec:App_H_eq_5} 
\xi(\frac{1}{2}  + \sigma + i \omega) =   \frac{1}{2}  +   (- \frac{1}{4} +  \sigma^{2} -  \omega^{2} + i \omega (2 \sigma)  )  [ \int_{-\infty}^{0}  \sum\limits_{n=1}^{\infty} e^{- \pi n^{2} e^{-2t} }  e^{\frac{-t}{2}} e^{-\sigma t} e^{ -i \omega t} dt \\
+ \int_{0}^{\infty}  \sum\limits_{n=1}^{\infty} e^{- \pi n^{2} e^{2t} }   e^{\frac{t}{2}} e^{-\sigma t} e^{ -i \omega t}   dt  ] 
\end{eqnarray*}
\begin{equation} \end{equation}

We can write this as follows.

\begin{equation}  \label{sec:App_H_eq_6} 
\xi(\frac{1}{2}  + \sigma + i \omega) =   \frac{1}{2}  +   (- \frac{1}{4} +  \sigma^{2} -  \omega^{2} + i \omega (2 \sigma)  )   \int_{-\infty}^{\infty} [ \sum\limits_{n=1}^{\infty} e^{- \pi n^{2} e^{-2t} }  e^{\frac{-t}{2}} u(-t) + \sum\limits_{n=1}^{\infty} e^{- \pi n^{2} e^{2t} }   e^{\frac{t}{2}}   u(t) ]  e^{-\sigma t} e^{ -i \omega t} dt 
%\begin{eqnarray*} \end{eqnarray*}
\end{equation}

We define $A(t) =  [ \sum\limits_{n=1}^{\infty} e^{- \pi n^{2} e^{-2t} }  e^{\frac{-t}{2}} u(-t) + \sum\limits_{n=1}^{\infty} e^{- \pi n^{2} e^{2t} }   e^{\frac{t}{2}}   u(t) ]  e^{-\sigma t}  $ and get the \textbf{inverse Fourier transform} of $\xi(\frac{1}{2} + \sigma + i \omega)$ in above equation given by $E_p(t)$ as follows. We use dirac delta function  $ \delta(t) $.


\begin{eqnarray*}\label{sec:App_H_eq_10} 
E_p(t) =  \frac{1}{2} \delta(t) + (- \frac{1}{4} +  \sigma^{2}) A(t) + 2 \sigma \frac{dA(t)}{dt}  + \frac{d^{2}A(t)}{dt^{2}}  \\
A(t) = [ \sum\limits_{n=1}^{\infty} e^{- \pi n^{2} e^{-2t} }  e^{\frac{-t}{2}} u(-t) + \sum\limits_{n=1}^{\infty} e^{- \pi n^{2} e^{2t} }   e^{\frac{t}{2}}   u(t) ]  e^{-\sigma t}  \\
\frac{dA(t)}{dt} = \sum\limits_{n=1}^{\infty} e^{- \pi n^{2} e^{-2t} } e^{\frac{-t}{2}}  e^{-\sigma t} [ -\frac{1}{2} - \sigma + 2 \pi n^{2} e^{-2t} ] u(-t) + \sum\limits_{n=1}^{\infty} e^{- \pi n^{2} e^{2t} } e^{\frac{t}{2}}  e^{-\sigma t} [ \frac{1}{2} - \sigma - 2 \pi n^{2} e^{2t} ] u(t) \\
\frac{d^{2}A(t)}{dt^{2}}  = \sum\limits_{n=1}^{\infty} e^{- \pi n^{2} e^{-2t} } e^{\frac{-t}{2}} e^{-\sigma t} [ - 4 \pi n^{2} e^{-2t} + (-\frac{1}{2} - \sigma + 2 \pi n^{2} e^{-2t} )^{2} ] u(-t) \\+  \sum\limits_{n=1}^{\infty} e^{- \pi n^{2} e^{2t} } e^{\frac{t}{2}} e^{-\sigma t} [ - 4 \pi n^{2} e^{2t} + (\frac{1}{2} - \sigma - 2 \pi n^{2} e^{2t} )^{2} ] u(t) + \delta(t) [\sum\limits_{n=1}^{\infty} e^{- \pi n^{2}} (1 - 4 \pi n^{2} ) ] 
\end{eqnarray*}
\begin{equation} \end{equation}

We can simplify above equation as follows.

\begin{eqnarray*}\label{sec:App_H_eq_11} 
\frac{d^{2}A(t)}{dt^{2}}  =   \sum\limits_{n=1}^{\infty} e^{- \pi n^{2} e^{-2t} } e^{\frac{-t}{2}}  e^{-\sigma t}  [ \frac{1}{4} + \sigma^2 +\sigma +   4 \pi^{2} n^{4} e^{-4t}  - 6 \pi n^{2} e^{-2t} - 4 \sigma \pi n^{2} e^{-2t}  ] u(-t) \\
\sum\limits_{n=1}^{\infty} e^{- \pi n^{2} e^{2t} } e^{\frac{t}{2}}  e^{-\sigma t}  [ \frac{1}{4} + \sigma^2 -\sigma +   4 \pi^{2} n^{4} e^{4t}  - 6 \pi n^{2} e^{2t} + 4 \sigma \pi n^{2} e^{2t}  ] u(t) + \delta(t) [\sum\limits_{n=1}^{\infty} e^{- \pi n^{2}} (1 - 4 \pi n^{2} ) ] 
\end{eqnarray*}
\begin{equation} \end{equation}



We use the fact that  $F(x) = 1 + 2 w(x) = \frac{1}{\sqrt{x}} (1 + 2 w(\frac{1}{x}) )$, where $w(x) = \sum\limits_{n=1}^{\infty} e^{- \pi n^{2}x }$ and $x > 0$ is real$^{\citep{FWE}}$, and we take the first derivative of $F(x)$ and evaluate it at $x=1$. We see that $\sum\limits_{n=1}^{\infty} e^{- \pi n^{2}} (1 - 4 \pi n^{2} )  = -\frac{1}{2}$ (~\ref{sec:appendix_H_2}) and hence \textbf{dirac delta terms cancel each other} in equation below. 

\begin{eqnarray*}\label{sec:App_H_eq_12} 
E_p(t) =  \frac{1}{2} \delta(t) + (- \frac{1}{4} +  \sigma^{2}) A(t) + 2 \sigma \frac{dA(t)}{dt}  + \frac{d^{2}A(t)}{dt^{2}}  \\
E_p(t) =  \sum\limits_{n=1}^{\infty} e^{- \pi n^{2} e^{2t} } e^{\frac{t}{2}}  e^{-\sigma t}  [ -\frac{1}{4} + \sigma^2  + 2 \sigma( \frac{1}{2} - \sigma - 2 \pi n^{2} e^{2t} ) + \frac{1}{4} + \sigma^2 -\sigma +   4 \pi^{2} n^{4} e^{4t}  - 6 \pi n^{2} e^{2t} + 4 \sigma \pi n^{2} e^{2t}  ] u(t) \\
+ \sum\limits_{n=1}^{\infty} e^{- \pi n^{2} e^{-2t} } e^{\frac{-t}{2}}  e^{-\sigma t}  [ -\frac{1}{4} + \sigma^2  + 2 \sigma( -\frac{1}{2} - \sigma + 2 \pi n^{2} e^{-2t} ) \\+ \frac{1}{4} + \sigma^2 +\sigma +   4 \pi^{2} n^{4} e^{-4t}  - 6 \pi n^{2} e^{-2t} - 4 \sigma \pi n^{2} e^{-2t}  ] u(-t) 
\end{eqnarray*}
\begin{equation} \end{equation}

We can simplify above equation as follows. 

\begin{eqnarray*}\label{sec:App_H_eq_13} 
E_p(t) =  [ E_0(-t) u(-t) + E_0(t) u(t) ] e^{-\sigma t} \\
E_0(t) =  2 \sum\limits_{n=1}^{\infty} [  2 \pi^{2} n^{4} e^{4t}  - 3 \pi n^{2} e^{2t}  ]  e^{- \pi n^{2} e^{2t} } e^{\frac{t}{2}} 
\end{eqnarray*}
\begin{equation} \end{equation}

We use the fact that $E_{0}(t)=E_{0}(-t)$  because $\xi(s) = \xi(1-s)$ and hence $\xi(\frac{1}{2} + i \omega)=\xi(\frac{1}{2} - i \omega)$ when evaluated at the critical line $s = \frac{1}{2} + i \omega$. This means  $\xi(\frac{1}{2} + i \omega) = E_{0\omega}(\omega) = E_{0\omega}(-\omega)$ and  $E_0(t)=E_0(-t)$ and we arrive at the desired result for $E_p(t)$ as follows. 

\begin{eqnarray*}\label{sec:App_H_eq_14} 
E_0(t) =  2 \sum\limits_{n=1}^{\infty} [  2 \pi^{2} n^{4} e^{4t}  - 3 \pi n^{2} e^{2t}  ]  e^{- \pi n^{2} e^{2t} } e^{\frac{t}{2}} \\
E_p(t) =  E_0(t) e^{-\sigma t} =  2 \sum\limits_{n=1}^{\infty} [  2 \pi^{2} n^{4} e^{4t}  - 3 \pi n^{2} e^{2t}  ]  e^{- \pi n^{2} e^{2t} } e^{\frac{t}{2}} e^{-\sigma t} 
\end{eqnarray*}
\begin{equation} \end{equation}


\subsection{\label{sec:appendix_H_2} \textbf{Derivation of  $\sum\limits_{n=1}^{\infty} e^{- \pi n^{2}} (1 - 4 \pi n^{2} )  = -\frac{1}{2}$  } \protect\\  \lowercase{} }

In this section, we derive $\sum\limits_{n=1}^{\infty} e^{- \pi n^{2}} (1 - 4 \pi n^{2} )  = -\frac{1}{2}$. We use the fact that  $F(x) = 1 + 2 w(x) = \frac{1}{\sqrt{x}} (1 + 2 w(\frac{1}{x}) )$, where $w(x) = \sum\limits_{n=1}^{\infty} e^{- \pi n^{2}x }$ and $x > 0$ is real$^{\citep{FWE}}$, and we take the first derivative of $F(x)$ and evaluate it at $x=1$.

\begin{eqnarray*}\label{sec:App_H_2_eq_1} 
F(x) = 1 + 2 w(x) = \frac{1}{\sqrt{x}} (1 + 2 w(\frac{1}{x}) ) \\
F(x) = 1 + 2 \sum\limits_{n=1}^{\infty} e^{- \pi n^{2}x } = \frac{1}{\sqrt{x}} (1 + 2 \sum\limits_{n=1}^{\infty} e^{- \pi n^{2}\frac{1}{x} } ) \\
\frac{dF(x)}{dx} = 2 \sum\limits_{n=1}^{\infty} (- \pi n^{2}) e^{- \pi n^{2}x } = \frac{1}{\sqrt{x}}  \sum\limits_{n=1}^{\infty} (2 \pi n^{2}) e^{- \pi n^{2}\frac{1}{x} } (\frac{1}{x^2})  + (1 + 2 \sum\limits_{n=1}^{\infty} e^{- \pi n^{2}\frac{1}{x} } )  (\frac{-1}{2}) \frac{1}{x^{\frac{3}{2}}}
\end{eqnarray*}
\begin{equation} \end{equation}

We evaluate the above equation at $x=1$ and we simplify as follows.


\begin{eqnarray*}\label{sec:App_H_2_eq_2} 
[ \frac{dF(x)}{dx} ]_{x=1} = 2 \sum\limits_{n=1}^{\infty} (- \pi n^{2}) e^{- \pi n^{2} } =  \sum\limits_{n=1}^{\infty} (2 \pi n^{2}) e^{- \pi n^{2} }   + (1 + 2 \sum\limits_{n=1}^{\infty} e^{- \pi n^{2} } )  (\frac{-1}{2})  \\
\sum\limits_{n=1}^{\infty} e^{- \pi n^{2}} (1 - 4 \pi n^{2} )  = -\frac{1}{2}
\end{eqnarray*}
\begin{equation} \end{equation}





\section{\label{sec:appendix_I} Properties of Fourier Transforms Part 1 \protect\\  \lowercase{} }

In this section, some well-known properties of Fourier transforms are re-derived.


\subsection{\label{sec:appendix_I_1} \textbf{ Convolution Theorem: Multiplication of g(t) and h(t) corresponds to convolution in Fourier transform domain} \protect\\  \lowercase{} }

We start with the Fourier transform equation $F(\omega)=  \int_{-\infty}^{\infty} f(t) e^{-i \omega t} dt$ where $f(t)=g(t) h(t)$
and show that $F(\omega)=  \frac{1}{2 \pi}  [ G(\omega) \ast H(\omega)]  = \frac{1}{2 \pi}  \int_{-\infty}^{\infty} G(\omega') H(\omega - \omega') d\omega'$ obtained by the \textbf{convolution} of the functions $G(\omega)$ and $H(\omega)$ which correspond to the Fourier transforms of $g(t)$ and $h(t)$ respectively.\\

\begin{equation} \label{sec_C_1_eq_0}   
F(\omega)=  \int_{-\infty}^{\infty} f(t) e^{-i \omega t} dt = \int_{-\infty}^{\infty} g(t) h(t) e^{-i\omega t} dt 
\end{equation}

We use the inverse Fourier transform equation  $g(t)=  \frac{1}{2 \pi} \int_{-\infty}^{\infty} G(\omega') e^{i \omega' t} d\omega' $ and we interchange the order of integration in equations below using Fubini's theorem \href{https://en.m.wikipedia.org/wiki/Convolution_theorem}{(link)}.

\begin{eqnarray*}\label{sec_C_1_eq_1}   
F(\omega)=  \frac{1}{2 \pi}  \int_{-\infty}^{\infty} [  \int_{-\infty}^{\infty} G(\omega') e^{i \omega' t} d\omega' ] h(t) e^{-i\omega t} dt \\
F(\omega)=  \frac{1}{2 \pi}  \int_{-\infty}^{\infty} G(\omega')  [ \int_{-\infty}^{\infty}  e^{i \omega' t}   h(t) e^{-i\omega t} dt ] d\omega' \\
F(\omega)=  \frac{1}{2 \pi}  \int_{-\infty}^{\infty} G(\omega')  [ \int_{-\infty}^{\infty}   h(t) e^{-i(\omega- \omega') t} dt ] d\omega'
\end{eqnarray*}
\begin{equation} \end{equation}

We substitute $\int_{-\infty}^{\infty}   h(t) e^{-i(\omega- \omega') t} dt  = H(\omega - \omega')$ in Eq.~\ref{sec_C_1_eq_1} and arrive at the convolution theorem.

\begin{equation} \label{sec_C_1_eq_2}   
F(\omega) = \frac{1}{2 \pi}  \int_{-\infty}^{\infty} G(\omega') H(\omega - \omega') d\omega' = \int_{-\infty}^{\infty} g(t) h(t) e^{-i\omega t} dt
\end{equation}




\subsection{\label{sec:appendix_I_2} \textbf{Fourier transform of Real g(t)} \protect\\  \lowercase{} }


In this section, we show that the Fourier transform of a real function $g(t)$, given by $G(\omega) =  G_R(\omega) + i G_I(\omega)$ has the properties given by $ G_R(-\omega) = G_R(\omega) $ and  $ G_I(-\omega) = -G_I(\omega)$.

\begin{eqnarray*}\label{sec_C_2_eq_1}   
G(\omega)=  \int_{-\infty}^{\infty} g(t) e^{-i \omega t} dt = G_R(\omega) + i G_I(\omega)\\
G_R(\omega)=  \int_{-\infty}^{\infty} g(t) \cos{(\omega t) } dt = G_R(-\omega) \\
G_I(\omega)=  - \int_{-\infty}^{\infty} g(t) \sin{(\omega t) } dt = - G_I(-\omega) 
\end{eqnarray*}
\begin{equation} \end{equation}


\subsection{\label{sec:appendix_I_3} \textbf{Even part of g(t) corresponds to real part of Fourier transform $G(\omega)$} \protect\\  \lowercase{} }


In this section, we show that the \textbf{even part} of real function $g(t)$, given by $g_{even}(t)=\frac{1}{2} [g(t)+g(-t) ] $, corresponds to \textbf{real part} of its Fourier transform $G(\omega)$. We use the fact that $ G_R(-\omega) = G_R(\omega) $ and  $ G_I(-\omega) = -G_I(\omega) $ for a real function $g(t)$.


\begin{eqnarray*}\label{sec_C_3_eq_1}   
G(\omega)=  \int_{-\infty}^{\infty} g(t) e^{-i \omega t} dt = G_R(\omega) + i G_I(\omega)\\
 \int_{-\infty}^{\infty} g_{even}(t) e^{-i \omega t} dt = \int_{-\infty}^{\infty} \frac{1}{2} [g(t)+g(-t) ] e^{-i\omega t} dt =\frac{1}{2}  [ G(\omega) + G(-\omega)  ] = G_R(\omega) 
\end{eqnarray*}
\begin{equation} \end{equation}



\subsection{\label{sec:appendix_I_4} \textbf{Odd part of g(t) corresponds to imaginary part of Fourier transform $G(\omega)$} \protect\\  \lowercase{} }

In this section, we show that the \textbf{odd part} of real function $g(t)$, given by $g_{odd}(t)=\frac{1}{2} [g(t)-g(-t) ] $, corresponds to \textbf{imaginary part} of its Fourier transform $G(\omega)$. We use the fact that $ G_R(-\omega) = G_R(\omega) $ and  $ G_I(-\omega) = -G_I(\omega) $ for a real function $g(t)$.


\begin{eqnarray*}\label{sec_C_4_eq_1}   
G(\omega)=  \int_{-\infty}^{\infty} g(t) e^{-i \omega t} dt = G_R(\omega) + i G_I(\omega)\\
 \int_{-\infty}^{\infty} g_{odd}(t) e^{-i \omega t} dt = \int_{-\infty}^{\infty} \frac{1}{2} [g(t)-g(-t) ] e^{-i\omega t} dt =\frac{1}{2}  [ G(\omega) - G(-\omega)  ] = i G_I(\omega) 
\end{eqnarray*}
\begin{equation} \end{equation}








\section{\label{sec:appendix_C} Properties of Fourier Transforms Part 2 \protect\\  \lowercase{} }

\subsection{\label{sec:appendix_C_1} \textbf{ $E_p(t), h(t), g(t)$ are absolutely integrable functions and their Fourier Transforms are finite. } \protect\\  \lowercase{} }

The inverse Fourier Transform of the function $ E_{p\omega}(\omega)=\xi(\frac{1}{2}+ \sigma + i \omega)$ is given by $E_p(t) = E_0(t) e^{-\sigma t} = \frac{1}{2 \pi} \int_{-\infty}^{\infty} E_{p\omega}(\omega) e^{i \omega t} d\omega $. We see that $E_0(t) = 2 \sum_{n=1}^{\infty}  [ 2 \pi^{2} n^{4} e^{4t}    - 3 \pi n^{2}   e^{2t} ]  e^{- \pi n^{2} e^{2t}} e^{\frac{t}{2}} > 0 $ for all $0 \leq t < \infty$. Given that $E_0(t)=E_0(-t)$, we see that $E_0(t) > 0$ and  $E_p(t) = E_0(t) e^{-\sigma t} > 0 $ for all $-\infty < t < \infty$.\\

As $t \to \infty$, $E_p(t)$  goes to zero, due to the term $e^{- \pi n^{2} e^{2t} }$.  As $t \to -\infty$, $E_p(t)$ goes to zero, because for every value of $n$, the term $e^{- \pi n^{2} e^{2t} } e^{\frac{5t}{2}} e^{-\sigma t}$ goes to zero, for $0 \leq |\sigma| < \frac{1}{2}$. Hence $E_p(t)= E_0(t) e^{-\sigma t} = 0$ at $t=\pm \infty$ and we showed that $E_p(t) > 0 $ for all $-\infty < t < \infty$. Hence $E_{p\omega}(\omega) = \int_{-\infty}^{\infty} E_p(t) e^{-i \omega t} dt$, evaluated at $\omega=0$ \textbf{cannot} be zero. Hence $E_{p\omega}(\omega)$ \textbf{does not have a zero} at $\omega=0$ and hence $\omega_{0} \neq 0$.\\

%Given that $E_{0\omega}(\omega)=\xi(\frac{1}{2} + i \omega)$ is an entire function and finite for all $\omega$, we see that $E_0(t)=0$  at $t=\pm \infty$, because if $E_0(t) \neq 0$  at $t=\pm \infty$, then its Fourier transform $E_{0\omega}(\omega)$ will not be finite.


%The inverse Fourier Transform of the function $ E_{p\omega}(\omega) = \xi(\frac{1}{2} + \sigma + i \omega)$ is given by $E_p(t) = E_0(t) e^{-\sigma t}$. We see that $E_0(t) = 2 \sum_{n=1}^{\infty}  [ 2 \pi^{2} n^{4} e^{4t}    - 3 \pi n^{2}   e^{2t} ]  e^{- \pi n^{2} e^{2t}} e^{\frac{t}{2}} > 0 $ for all $0 \leq t < \infty$. Given that $E_0(t)=E_0(-t)$, we see that $E_0(t) > 0$ and  $E_p(t) = E_0(t) e^{-\sigma t} > 0 $ for all $-\infty < t < \infty$. We see that $E_p(t)=0$ at $t=\pm \infty$ and hence $E_p(t) \geq 0$ for all $|t| \leq \infty$.\\

Given that $\xi(\frac{1}{2} + \sigma + i \omega)= E_{p\omega}(\omega)$ is an entire function in the whole of s-plane, it is finite for $|\omega| \leq \infty$ and also for $\omega=0$. Hence $\int_{-\infty}^{\infty} E_p(t) dt$ is finite. We see that $E_p(t) \geq 0$ for all $|t| \leq \infty$. Hence we can write $\int_{-\infty}^{\infty} |E_p(t)| dt$ is finite and $E_p(t)$ is an  absolutely \textbf{integrable function} and its Fourier transform $ E_{p\omega}(\omega)$ goes to zero as $\omega \to \pm \infty$, as per Riemann Lebesgue Lemma \href{https://en.wikipedia.org/wiki/Riemann-Lebesgue\_lemma}{(link)}.\\




Let us consider a new function  $g(t) = E_p(t) e^{-\sigma t}  u(-t) + E_p(t) e^{\sigma t}  u(t)  $ where $g(t)$ is a real function of variable $t$ and $u(t)$ is Heaviside unit step function and $0 < \sigma < \frac{1}{2}$. We can see that $g(t) h(t) = E_p(t)$ where $h(t)=  e^{ \sigma t} u(-t) + e^{ - \sigma t} u(t) $. \\

We can see that $h(t)=  e^{ \sigma t} u(-t) + e^{ - \sigma t} u(t) $ is an absolutely \textbf{integrable function} because $\int_{-\infty}^{\infty} |h(t)| dt = \int_{-\infty}^{\infty} h(t) dt =  [ \int_{-\infty}^{\infty} h(t) e^{-i \omega t} dt ]_{\omega=0} = [ \frac{1}{\sigma - i \omega} +  \frac{1}{\sigma + i \omega} ]_{\omega=0} = \frac{2}{\sigma}$, is finite for $0 < \sigma < \frac{1}{2}$ and its Fourier transform $ H(\omega)$ goes to zero as $\omega \to \pm \infty$, as per Riemann Lebesgue Lemma \href{https://en.wikipedia.org/wiki/Riemann-Lebesgue\_lemma}{(link)}.\\

It is shown in ~\ref{sec:appendix_C_5} that $E_0(t)$ and $E_0(t) e^{-2 \sigma t}$ have fall-off rates \textbf{at least} $\frac{1}{t^2}$ as $|t| \to \infty$ and hence are absolutely \textbf{integrable} functions and the integrals  $\int_{-\infty}^{\infty} |E_0(t)| dt < \infty$ and  $\int_{-\infty}^{\infty} |E_0(t) e^{-2 \sigma t}| dt < \infty$. Hence  $g(t) = E_0(t) e^{-2 \sigma t} u(-t) + E_0(t) u(t)$ is an  absolutely \textbf{integrable function} and $\int_{-\infty}^{\infty} |g(t)| dt = \int_{-\infty}^{\infty} g(t) dt$ is finite and its Fourier transform $ G(\omega)$ goes to zero as $\omega \to \pm \infty$, as per Riemann Lebesgue Lemma \href{https://en.wikipedia.org/wiki/Riemann-Lebesgue\_lemma}{(link)}.\\
 
 %exponential fall-off rate $e^{-B|t|}$, as $|t| \to \infty$, where $B>0$ is real,
 
%We can see that $g(t) = E_p(t) e^{-\sigma t}  u(-t) + E_p(t) e^{\sigma t}  u(t) \geq 0 $ for all $|t| \leq \infty$ because $E_p(t) \geq 0$ for all $|t| \leq \infty$. Given that $E_{p}(t) = E_0(t) e^{-\sigma t} =  [ E_0(t)  u(-t) +   E_0(-t)  u(t) ]    e^{-\sigma t} $ where $E_0(t) = 2 \sum_{n=1}^{\infty}  [ 2 \pi^{2} n^{4} e^{4t}    - 3 \pi n^{2}   e^{2t} ]  e^{- \pi n^{2} e^{2t}} e^{\frac{t}{2}}  $ , we see that $g(t)$ goes to zero as $t \to -\infty$ with its order of decay greater than $e^{\frac{3t}{2}}$ and $g(t)$ goes to zero as $t \to \infty$ with its order of decay greater than $e^{-\frac{5t}{2}}$, for $0 < \sigma < \frac{1}{2}$. 


\subsection{\label{sec:appendix_C_2} \textbf{ Convolution integral convergence  } \protect\\  \lowercase{} }

Let us consider $h(t)=  e^{ \sigma t} u(-t) + e^{ - \sigma t} u(t) $  whose \textbf{first derivative is discontinuous} at $t=0$. The second derivative of $h(t)$ given by $h_2(t)$ has a Dirac delta function $A_0 \delta(t)$ where $A_0=-2 \sigma$ and its Fourier transform $H_2(\omega)$ has a constant term $A_0$, corresponding to the Dirac delta function.\\

This means $h(t)$ is obtained by integrating $h_2(t)$ twice and its Fourier transform $H(\omega)$ has a term $-\frac{A_0}{\omega^2}$  \href{https://web.stanford.edu/class/ee102/lectures/fourtran#page=15}{(link)} and has a \textbf{fall off rate} of $\frac{1}{\omega^2}$ as $|\omega| \to \infty$ and  $\int_{-\infty}^{\infty} H(\omega) d\omega $ converges.\\

We see that $E_{p}(t) = E_0(t) e^{-\sigma t} $ where $E_0(t) = 2 \sum_{n=1}^{\infty}  [ 2 \pi^{2} n^{4} e^{4t}    - 3 \pi n^{2}   e^{2t} ]  e^{- \pi n^{2} e^{2t}} e^{\frac{t}{2}}  $.\\

Let us consider a new function  $g(t) = E_p(t) e^{-\sigma t}  u(-t) + E_p(t) e^{\sigma t}  u(t)  $ where $g(t)$ is a real function of variable $t$ and $u(t)$ is Heaviside unit step function and $0 < \sigma < \frac{1}{2}$. We can see that $g(t) h(t) = E_p(t)$ where $h(t)=  e^{ \sigma t} u(-t) + e^{ - \sigma t} u(t) $. \\

We can see that $G(\omega), H(\omega)$ have \textbf{fall-off rate }of $\frac{1}{\omega^2}$ as $|\omega| \to \infty$ because the \textbf{first derivatives }of $g(t), h(t)$ are \textbf{discontinuous} at $t=0$. Also, $h(t), g(t)$ are absolutely integrable functions and their Fourier Transforms are finite as shown in ~\ref{sec:appendix_C_1}. Hence the convolution integral below converges to a finite value for  $|\omega| \leq \infty$.

\begin{equation} \label{sec_C_1_eq_2}   
F(\omega) = \frac{1}{2 \pi}  \int_{-\infty}^{\infty} G(\omega') H(\omega - \omega') d\omega' = \frac{1}{2 \pi}  [ G(\omega) \ast H(\omega)]
\end{equation}

\subsection{\label{sec:appendix_C_3} \textbf{ Fall off rate of Fourier Transform of functions } \protect\\  \lowercase{} }

Let us consider a real Fourier transformable function $P(t)=P_{+}(t) u(t) + P_{-}(t)u(-t)$ whose \textbf{$(N-1)^{th}$ derivative is discontinuous} at $t=0$. The $(N)^{th}$ derivative of $P(t)$ given by $P_N(t)$ has a Dirac delta function $A_0 \delta(t)$ where $A_0 = [\frac{d^{N-1}P_{+}(t)}{dt^{N-1}} - \frac{d^{N-1}P_{-}(t)}{dt^{N-1}} ]_{t=0}$ and its Fourier transform $P_N(\omega)$ has a constant term $A_0$, corresponding to the Dirac delta function.\\

This means $P(t)$ is obtained by integrating $P_N(t)$, N times and its Fourier transform $P(\omega)$ has a term $ \frac{A_0}{(i \omega)^N}$ \href{https://web.stanford.edu/class/ee102/lectures/fourtran#page=15}{(link)} and has a \textbf{fall off rate} of $\frac{1}{\omega^N}$ as $|\omega| \to \infty$.\\% and  $\int_{-\infty}^{\infty} P(\omega) d\omega $ converges for $N>1$.\\

We have shown that if the \textbf{$(N-1)^{th}$ derivative} of the function $P(t)$ is \textbf{discontinuous} at $t=0$ then its Fourier transform $P(\omega)$ has a \textbf{fall-off rate} of $\frac{1}{\omega^N}$ as $|\omega| \to \infty$ . \\

In Section~\ref{sec:Section_1_1}, we showed that $E_0(t)$ is an analytic function which is infinitely differentiable which produces no discontinuities in $|t| \leq \infty$. Hence its Fourier transform $E_{0\omega}(\omega)$ has a fall-off rate faster than $\frac{1}{\omega^M}$ as $M \to \infty$, as $|\omega| \to \infty$ and it should have a fall-off rate \textbf{at least }of the order of $\omega^A e^{-B|\omega|}$ as $|\omega| \to \infty$, where $A, B>0$ are  real. 

%\subsection{\label{sec:appendix_C_4} \textbf{ Fall off rate of $E_0(t)$ and $E_0(t) e^{-2 \sigma t}$ } \protect\\  \lowercase{} }

%Given that Riemann's Xi function $\xi(s)$ is an entire function in the s-plane, we see that $E_{0\omega}(\omega)= \xi(\frac{1}{2} + i \omega)$ is an analytic function which is infinitely differentiable which produces no discontinuities for all $|\omega| \leq \infty$ .
%Hence its inverse Fourier transform $E_0(t)$ has a fall-off rate faster than $\frac{1}{t^M}$ as $M \to \infty$, as $|t| \to \infty$ and it should have a fall-off rate \textbf{at least} of the order of $e^{-A|t|}$ as $|t| \to \infty$, where $A>0$ is real.(~\ref{sec:appendix_C_3}) \\

%Hence $E_0(t)$ and $E_0(t) e^{-2 \sigma t}$ are absolutely \textbf{integrable} functions and the integrals  $\int_{-\infty}^{\infty} |E_0(t)| dt < \infty$ and  $\int_{-\infty}^{\infty} |E_0(t) e^{-2 \sigma t}| dt < \infty$.

\subsection{\label{sec:appendix_C_5} \textbf{ Payley-Weiner theorem and Fall off rate of analytic functions. } \protect\\  \lowercase{} }

We know that Payley-Weiner theorem relates analytic functions and exponential decay rate of their Fourier transforms \href{http://cosweb1.fau.edu/~jmirelesjames/ODE_course/Complex_Analysis.pdf#page=140}{(link)}. Using similar arguments, we will show that the functions $E_0(t), E_p(t)$ and $x(t)=E_0(t) e^{-2 \sigma t}$ and $\frac{d^{2r}x(t)}{dt^{2r}}$ have  fall-off rates \textbf{at least} $\frac{1}{t^2}$ as $|t| \to \infty$ for $0 < \sigma < \frac{1}{2}$.\\
%exponential fall-off rate $e^{-B|t|}$, as $|t| \to \infty$, where $B>0$ is real and $0 < \sigma < \frac{1}{2}$. \\

We know that the order of Riemann's Xi function $\xi(\frac{1}{2} + i \omega) = E_{0\omega}(\omega) = \Xi(\omega)$ is given by $O(\omega^A e^{-\frac{|\omega|\pi}{4}})$ where $A$ is a constant$^{\citep{ECT}}$ \href{https://www.ocf.berkeley.edu/~araman/files/math_z/titchmarsh_p2.png}{(link)}. Hence both $E_{0\omega}(\omega)$ and $E_{p\omega}(\omega)= \xi(\frac{1}{2} + \sigma + i \omega) = E_{0\omega}(\omega - i \sigma)$ have \textbf{exponential fall-off} rate $O(\omega^A e^{-\frac{|\omega|\pi}{4}})$ as $|\omega| \to \infty$ and they are absolutely integrable and Fourier transformable, given that they are derived from an entire function $\xi(s)$.\\
%($\sigma' = \frac{1}{2} + \sigma$) 

Given that $\xi(s)$ is an entire function in the s-plane, we see that $E_{0\omega}(\omega)$ and $E_{p\omega}(\omega)$ are \textbf{analytic} functions which are infinitely differentiable which produce no discontinuities for all $|\omega| \leq \infty$ and $0 < \sigma < \frac{1}{2}$. Hence their respective \textbf{inverse Fourier transforms} $E_0(t), E_p(t)$ have fall-off rates faster than $\frac{1}{t^M}$ as $M \to \infty$, as $|t| \to \infty$ (~\ref{sec:appendix_C_3})  and hence it should have a fall-off rate \textbf{at least} $\frac{1}{t^2}$ as $|t| \to \infty$.\\
%$ 0 < \sigma' < 1$
%$t^A e^{-B|t|}$ as $|t| \to \infty$, where $B>0$ and $A$ are real.\\

We can use similar arguments to show that $x(t)=E_0(t) e^{-2 \sigma t}$ and $\frac{d^{2r}x(t)}{dt^{2r}}$  have fall-off rates \textbf{at least} $\frac{1}{t^2}$ as $|t| \to \infty$, because their Fourier transforms are \textbf{analytic} functions for all $|\omega| \leq \infty$ with  \textbf{exponential fall-off} rate $O( \omega^A e^{-\frac{|\omega|\pi}{4}})$ as $|\omega| \to \infty$. %and $g(t) = E_0(t) e^{-2 \sigma t} u(-t) + E_0(t) u(t)$
% \textbf{exponential} fall-off rate $e^{-B|t|}$, as $|t| \to \infty$, where $B>0$ is real
















\clearpage
\section{\label{sec:appendix_F}  First 2 derivatives of $R(t_0)$ \protect\\  \lowercase{} }

In this section, we derive the first 2 derivatives of $R(t_0)$. We use the result in Section~\ref{sec:Section_A_2} that $\omega_2(t_0)$ is \textbf{at least} differentiable twice. \\
%assume that $\omega_2(t_0)$ is\textbf{ not} analytic and  we can derive the equation for $R(t_0)$ in Eq.~\ref{sec_2_2_eq_1},  using integration by parts, \textbf{without} using Taylor series for $\cos{ (\omega_2(t_0) t_0)}, \sin{ (\omega_2(t_0) t_0)}$



We expand a few terms in $R(t_0)$  which are analytic functions, using Taylor series as follows. We use $E_{0}(t)=E_{0}(-t)$, $E_0(t) e^{-2 \sigma t} = [e_0 + e_2 \frac{t^{2}}{!2} + e_4 \frac{t^{4}}{!4} + ... ] [  1 -  2 \sigma t +  2\sigma^{2} t^{2}+...] = e_0  -  2 \sigma e_0 t + t^{2}  (\frac{e_2}{!2} +  2 e_0 \sigma^{2}) + ... $.\\

We use  $(f_{c}(t))_{-\infty} = [\int    E_0(t) e^{-2 \sigma t}  \cos{ ( \omega_2(t_0) t)} dt ]_{t=-\infty} -K_{1c} = - M(t_0) -K_{1c}$ and \\$(f_{s}(t))_{-\infty} =  [\int    E_0(t) e^{-2 \sigma t}  \sin{ ( \omega_2(t_0) t)} dt ]_{t=-\infty}  -K_{1s}  = -N(t_0)  -K_{1s}$ as derived in ~\ref{sec:Appendix_F_7}  and split each integral in Eq.~\ref{sec_1_eq_7} copied below, into two integrals evaluated at upper and lower limits. $M(t_0), N(t_0)$ are defined in ~\ref{sec:Appendix_F_4}. Integration constants $K_{1c},K_{1s}$ get \textbf{cancelled} at upper and lower limits of the integrals.% and hence \textbf{omitted} in computations below. 

\begin{eqnarray*}\label{app_F_eq_1}   
R(t_0) =  e^{ 2 \sigma t_0} [ \cos{ (\omega_2(t_0) t_0)} \int_{-\infty}^{t_0}    E_0(t) e^{-2 \sigma t}  \cos{ ( \omega_2(t_0) t)} dt + \sin{ (\omega_2(t_0) t_0)}  \int_{-\infty}^{t_0}  E_0(t) e^{-2 \sigma t}  \sin{ (\omega_2(t_0) t)} dt ]  \\
R(t_0) =   e^{ 2 \sigma t_0} [  \cos{ (\omega_2(t_0) t_0)} 
  \int_{-\infty}^{t_0} [  e_0  -  2 \sigma e_0 t + t^{2}  (\frac{e_2}{!2} +  2 e_0 \sigma^{2}) + ... ]  \cos{ ( \omega_2(t_0) t)} dt \\
 +   \sin{ (\omega_2(t_0) t_0)} 
 \int_{-\infty}^{t_0} [ e_0  -  2 \sigma e_0 t + t^{2}  (\frac{e_2}{!2} +  2 e_0 \sigma^{2}) + ... ] \sin{ ( \omega_2(t_0) t)}  dt ] \\
 R(t_0) =   e^{ 2 \sigma t_0} [  \cos{ (\omega_2(t_0) t_0)} 
  [ \int   (  e_0  -  2 \sigma e_0 t + t^{2}  (\frac{e_2}{!2} +  2 e_0 \sigma^{2}) + ... )  \cos{ ( \omega_2(t_0) t)} dt ]_{t=t_0} \\
 +   \sin{ (\omega_2(t_0) t_0)} 
[ \int    ( e_0  -  2 \sigma e_0 t + t^{2}  (\frac{e_2}{!2} +  2 e_0 \sigma^{2}) + ... ) \sin{ ( \omega_2(t_0) t)}  dt ]_{t=t_0} ] \\+ e^{ 2 \sigma t_0} ( (M(t_0)+K_{1c})  \cos{ (\omega_2(t_0) t_0)} + (N(t_0)+K_{1s}) \sin{ (\omega_2(t_0) t_0)})
\end{eqnarray*}
\begin{equation} \end{equation}

Using\textbf{ repeated} integration by parts, for the first two terms $t^{0}, t^{1}$ in the two integrals in above equation, this can be simplified as follows. For the \textbf{first} integral $I_1(t_0)= \int_{-\infty}^{t_0}    E_0(t) e^{-2 \sigma t}  \cos{ ( \omega_2(t_0) t)} dt$, we use $u= \cos{ ( \omega_2(t_0) t)}, dv = t^{r} dt, v = \frac{t^{(r+1)}}{(r+1)}, du =  -  \omega_2(t_0)\sin{ ( \omega_2(t_0) t)} dt $ for $r=0,1$. For the \textbf{second} integral $Q_1(t_0)= \int_{-\infty}^{t_0}    E_0(t) e^{-2 \sigma t}  \sin{ ( \omega_2(t_0) t)} dt$, we use $u= \sin{ ( \omega_2(t_0) t)}, dv = t^{r} dt, v = \frac{t^{(r+1)}}{(r+1)}, du =    \omega_2(t_0)\cos{ ( \omega_2(t_0) t)} dt $ for $r=0,1$. 
%We expand the terms $I_1(t_0)$ and $Q_1(t_0)$ in $R(t_0)$ in Eq.~\ref{app_F_eq_1} as follows.


\begin{eqnarray*}\label{app_F_eq_2_0}  
I_1(t_0)= \int_{-\infty}^{t_0}    E_0(t) e^{-2 \sigma t}  \cos{ ( \omega_2(t_0) t)} dt =   e_0 [  ( t_0  \cos{ ( \omega_2(t_0) t_0)} + \frac{t_0^{2}}{!2}  \sin{ ( \omega_2(t_0) t_0)}\omega_2(t_0) + ...) ] \\
- 2 \sigma e_0 [  ( \frac{t_0^{2}}{2}  \cos{ ( \omega_2(t_0) t_0)} + \frac{t_0^{3}}{!3}  \sin{ ( \omega_2(t_0) t_0)}\omega_2(t_0) + ...)  \\
+ \int [ t^{2}  (\frac{e_2}{!2} +  2 e_0 \sigma^{2}) + ... ]  \cos{ ( \omega_2(t_0) t)} dt ]_{t=t_0} ] + (M(t_0)+K_{1c}) \\
%-(f_{c1}(t))_{-\infty} - K_{1c} = (f_{c1}(t))_{t=t_0} + K_{1c} -(f_{c1}(t))_{-\infty} - K_{1c} \\
Q_1(t_0)= \int_{-\infty}^{t_0}    E_0(t) e^{-2 \sigma t}  \sin{ ( \omega_2(t_0) t)} dt = e_0 [   ( t_0  \sin{ ( \omega_2(t_0) t_0)} - \frac{t_0^{2}}{!2}  \cos{ ( \omega_2(t_0) t_0)}\omega_2(t_0) + ...) ] \\
- 2 \sigma e_0 [  ( \frac{t_0^{2}}{2}  \sin{ ( \omega_2(t_0) t_0)} - \frac{t_0^{3}}{!3}  \cos{ ( \omega_2(t_0) t_0)}\omega_2(t_0) + ...)  ] \\
+ \int [ t^{2}  (\frac{e_2}{!2} +  2 e_0 \sigma^{2}) + ... ]  \sin{ ( \omega_2(t_0) t)} dt ]_{t=t_0} ] + (N(t_0)+K_{1s})
%-(f_{s1}(t))_{-\infty} - K_{1s} = (f_{s1}(t))_{t=t_0} + K_{1s} -(f_{s1}(t))_{-\infty} - K_{1s}
\end{eqnarray*}
\begin{equation} \end{equation}


We can simplify $R(t_0)$ in eq.~\ref{app_F_eq_1} as follows.


\begin{eqnarray*}\label{app_F_eq_2}   
R(t_0)=   e^{ 2 \sigma t_0} [  e_0 [ \cos{ (\omega_2(t_0) t_0)} ( t_0  \cos{ ( \omega_2(t_0) t_0)} + \frac{t_0^{2}}{!2}  \sin{ ( \omega_2(t_0) t_0)}\omega_2(t_0) + ...) \\ + \sin{ (\omega_2(t_0) t_0)} ( t_0  \sin{ ( \omega_2(t_0) t_0)} - \frac{t_0^{2}}{!2}  \cos{ ( \omega_2(t_0) t_0)}\omega_2(t_0) + ...) ] \\
- 2 \sigma e_0 [ \cos{ (\omega_2(t_0) t_0)} ( \frac{t_0^{2}}{2}  \cos{ ( \omega_2(t_0) t_0)} + \frac{t_0^{3}}{!3}  \sin{ ( \omega_2(t_0) t_0)}\omega_2(t_0) + ...) \\ + \sin{ (\omega_2(t_0) t_0)} ( \frac{t_0^{2}}{2}  \sin{ ( \omega_2(t_0) t_0)} - \frac{t_0^{3}}{!3}  \cos{ ( \omega_2(t_0) t_0)}\omega_2(t_0) + ...)  ] \\
+ \cos{ (\omega_2(t_0) t_0)} [\int [ t^{2}  (\frac{e_2}{!2} +  2 e_0 \sigma^{2}) + ... ]  \cos{ ( \omega_2(t_0) t)} dt ]_{t=t_0}\\
 +   \sin{ (\omega_2(t_0) t_0)} 
 [\int [  t^{2}  (\frac{e_2}{!2} +  2 e_0 \sigma^{2}) + ... ] \sin{ ( \omega_2(t_0) t)}  dt ]_{t=t_0} ] \\
 +  e^{ 2 \sigma t_0} [ (K_{1c}  \cos{ (\omega_2(t_0) t_0)} + K_{1s} \sin{ (\omega_2(t_0) t_0)} ]
+  e^{ 2 \sigma t_0} [ (M(t_0)  \cos{ (\omega_2(t_0) t_0)} + N(t_0) \sin{ (\omega_2(t_0) t_0)} ]
\end{eqnarray*}
\begin{equation} \end{equation}


This can be further simplified as follows by cancelling common terms in the term involving $e_0$ and $2 \sigma e_0$.
Using $e^{ 2 \sigma t_0} = 1 + 2 \sigma t_0 + 2 \sigma^{2} t_0^{2}+... =  \displaystyle\sum_{k=0}^{\infty} (2 \sigma)^{k}  \frac{t_0^{k}}{!k}$, we get

\begin{eqnarray*}\label{app_F_eq_3}   
R(t_0)=   (1 + 2 \sigma t_0 + 2 \sigma^{2} t_0^{2}+...) [ e_0 [ t_0 + \frac{t_0^{3}}{!3} \omega_2^2(t_0) + ...] -  2\sigma e_0 [ \frac{t_0^{2}}{!2} + \frac{t_0^{4}}{!4} \omega_2^2(t_0) + ...  ]  \\
+ \cos{ (\omega_2(t_0) t_0)} [\int [ t^{2}  (\frac{e_2}{!2} +  2 e_0 \sigma^{2}) + ... ]  \cos{ ( \omega_2(t_0) t)} dt ]_{t=t_0}\\
 +   \sin{ (\omega_2(t_0) t_0)} 
 [\int [  t^{2}  (\frac{e_2}{!2} +  2 e_0 \sigma^{2}) + ... ] \sin{ ( \omega_2(t_0) t)}  dt ]_{t=t_0} ] \\
 +  e^{ 2 \sigma t_0} [ (K_{1c}  \cos{ (\omega_2(t_0) t_0)} + K_{1s} \sin{ (\omega_2(t_0) t_0)} ]
+  e^{ 2 \sigma t_0} [ (M(t_0)  \cos{ (\omega_2(t_0) t_0)} + N(t_0) \sin{ (\omega_2(t_0) t_0)} ]
\end{eqnarray*}
\begin{equation} \end{equation}


Integration constants $K_{1c},K_{1s}$ get \textbf{cancelled} at upper and lower limits of the integrals.
The terms inside the integrals in above equation can be shown to have terms of the order of $t_{0}^{3}$ and above.
Hence we can write as follows, where $a_k$ are the coefficients of the terms $ \frac{t_0^{k}}{!k}$.


\begin{eqnarray*}\label{app_F_eq_4}   
R(t_0)=   (1 + 2 \sigma t_0 + 2 \sigma^{2} t_0^{2}+...) [ ( e_0 t_0 -  2\sigma e_0 \frac{t_0^{2}}{2}  + a_3 \frac{t_0^{3}}{3}  +  a_4 \frac{t_0^{4}}{4} + ...) ] \\ e^{ 2 \sigma t_0} [ (M(t_0)  \cos{ (\omega_2(t_0) t_0)} + N(t_0) \sin{ (\omega_2(t_0) t_0)} ]\\
R(t_0)=   (1 + 2 \sigma t_0 + 2 \sigma^{2} t_0^{2}+...) [ ( e_0 t_0 -  \sigma e_0 t_0^{2} + a_3 \frac{t_0^{3}}{3}  +  a_4 \frac{t_0^{4}}{4} + ...) ] \\ + e^{ 2 \sigma t_0} [ (M(t_0)  \cos{ (\omega_2(t_0) t_0)} + N(t_0) \sin{ (\omega_2(t_0) t_0)} ]\\
R(t_0)= ( e_0 t_0 +t_0^2 ( -  \sigma e_0 + 2 \sigma e_0) + t_0^3 () + ..... ) + e^{ 2 \sigma t_0} [ (M(t_0)  \cos{ (\omega_2(t_0) t_0)} + N(t_0) \sin{ (\omega_2(t_0) t_0)} ]
\end{eqnarray*}
\begin{equation} \end{equation}



We want to evaluate the first and second derivative of $R(t_0)$ in section below.





%Thus, using proof by contradiction, we have shown that there is \textbf{at least one value} of $t_0$ for which the Fourier transform of $g(t)$ does not have real zeros and hence the Fourier transform of $E_p(t)=E_0(t) e^{\sigma t}$
%does not have real zeros.\\



%\textbf{Computation of first two derivatives of $M(t_0), N(t_0)$:}\\
\subsection{\label{sec:appendix_F_3} \textbf{Computation of first two derivatives of $M(t_0), N(t_0)$:} \protect\\  \lowercase{} }

%If $\omega_2(t_0)$ is a Weierstrass type of function which is differentiable nowhere (\textbf{Statement A}), then its first 2 derivatives are not finite. Given that derivatives of Dirac delta functions are defined by a well known equation \href{https://mathworld.wolfram.com/DeltaFunction.html}{(Eq. 17 in link)}, we can proceed with equations below, \textbf{even if} derivatives are not finite and derive further results. In ~\ref{sec:Appendix_D_7}, we show that Statement 1 implies that \textbf{Statement A is false} and that $\omega_2(t_0)$ is a continuous function, which is differentiable at least twice.\\


Define $\theta(t_0) = \omega_2(t_0) t_0$, we have $\frac{d\theta(t_0)}{dt_0} = t_0 \frac{d\omega_2(t_0)}{dt_0} + \omega_2(t_0)$ which equals $\omega_{20}$ at $t_0=0$.  $\frac{d^{2}\theta(t_0)}{dt_0^{2}} =  t_0 \frac{d^{2}\omega_2(t_0)}{dt_0^{2}} + 2 \frac{d\omega_2(t_0)}{dt_0}$ which equals zero at $t_0=0$, given that $\omega_2(t_0)$ is an even function of $t_0$. We substitute $(\frac{dM(t_0)}{dt_0})_{t_0=0}=0$ and $(\frac{dN(t_0)}{dt_0})_{t_0=0}=0$ from  Eq.~\ref{app_F_4_eq_1} and Eq.~\ref{app_F_4_eq_2} in Eq.~\ref{app_F_3_eq_1}. We can write Eq.~\ref{app_F_eq_4}  as follows.


\begin{eqnarray*}\label{app_F_3_eq_1}   
R(t_0)= ( e_0 t_0 +t_0^2 (  \sigma e_0) + t_0^3 () + ..... ) + MN(t_0)  \\
MN(t_0) =  e^{2 \sigma t_0} (M(t_0)  \cos{(\theta(t_0))} + N(t_0) \sin{ (\theta(t_0))}) \\
MN(0) = m_0 \\
\frac{dMN(t_0)}{dt_0} = e^{2 \sigma t_0} [  \cos{ (\theta(t_0))} [ 2 \sigma M(t_0) + \frac{dM(t_0)}{dt_0} +  N(t_0) \frac{d\theta(t_0)}{dt_0} ]  +   \sin{ (\theta(t_0))} [ 2 \sigma N(t_0) +  \frac{dN(t_0)}{dt_0} - M(t_0) \frac{d\theta(t_0)}{dt_0} ] ]\\
(\frac{dMN(t_0)}{dt_0})_{t_0=0}  = 2 \sigma M(0) + (\frac{dM(t_0)}{dt_0})_{t_0=0} +  N(0) \omega_{20} = 2 \sigma m_0 + n_0  \omega_{20}
\end{eqnarray*}
\begin{equation} \end{equation}

Now we compute the second derivative as follows. We use $ m_2 = \frac{1}{2} (\frac{d^{2}M(t_0)}{dt_0^{2}})_{t_0=0} $.


\begin{eqnarray*}\label{app_F_3_eq_2}   
\frac{d^{2}MN(t_0)}{dt_0^{2}} = e^{2 \sigma t_0} [  \cos{ (\theta(t_0))} [ 2 \sigma ( 2 \sigma M(t_0) + \frac{dM(t_0)}{dt_0} +  N(t_0) \frac{d\theta(t_0)}{dt_0} ) +  2 \sigma  \frac{dM(t_0)}{dt_0}+ \frac{d^2M(t_0)}{dt_0^2}+  N(t_0) \frac{d^2\theta(t_0)}{dt_0^2} \\ +  \frac{d\theta(t_0)}{dt_0} \frac{dN(t_0)}{dt_0} + \frac{d\theta(t_0)}{dt_0} (2 \sigma N(t_0) +  \frac{dN(t_0)}{dt_0} - M(t_0) \frac{d\theta(t_0)}{dt_0}) ]  \\
+   \sin{ (\theta(t_0))} [ 2 \sigma ( 2 \sigma N(t_0) +  \frac{dN(t_0)}{dt_0} - M(t_0) \frac{d\theta(t_0)}{dt_0} ) - \frac{d\theta(t_0)}{dt_0}(2 \sigma M(t_0) + \frac{dM(t_0)}{dt_0} +  N(t_0) \frac{d\theta(t_0)}{dt_0}) \\ +   2 \sigma  \frac{dN(t_0)}{dt_0}+ \frac{d^2N(t_0)}{dt_0^2}-  M(t_0) \frac{d^2\theta(t_0)}{dt_0^2} - \frac{d\theta(t_0)}{dt_0} \frac{dM(t_0)}{dt_0} ] ]\\
\frac{1}{2} (\frac{d^{2}MN(t_0)}{dt_0^{2}})_{t_0=0} =  \sigma ( 2 \sigma m_0  +  n_0 \omega_{20} ) + m_2 + \frac{1}{2}\omega_{20} (2 \sigma n_0  - m_0 \omega_{20} ) \\
\frac{1}{2} (\frac{d^{2}MN(t_0)}{dt_0^{2}})_{t_0=0} = m_2 +  2 \sigma n_0 \omega_{20} + 2 \sigma^2 m_0 - \frac{m_0}{2} \omega_{20}^2
\end{eqnarray*}
\begin{equation} \end{equation}

We substitute above result in Eq.~\ref{app_F_eq_4}  and derive as follows.

\begin{eqnarray*}\label{app_F_3_eq_3}  
R(t_0)= ( e_0 t_0 +t_0^2 (  \sigma e_0) + t_0^3 () + ..... ) + MN(t_0) \\
R(0) = MN(0) = m_0 \\
(\frac{dR(t_0)}{dt_0})_{t_0=0}  = e_0 +  (\frac{dMN(t_0)}{dt_0})_{t_0=0} = e_0 + 2 \sigma m_0 + n_0  \omega_{20}\\
\frac{1}{2} (\frac{d^{2}R(t_0)}{dt_0^{2}})_{t_0=0} = \sigma e_0 + \frac{1}{2} (\frac{d^{2}MN(t_0)}{dt_0^{2}})_{t_0=0} =  \sigma e_0 +  m_2 +  2 \sigma n_0 \omega_{20} + 2 \sigma^2 m_0 - \frac{m_0}{2} \omega_{20}^2  
\end{eqnarray*}
\begin{equation} \end{equation}

We can simplify as follows and get the result in Eq.~\ref{sec_2_2_eq_1}. %The results in equation below are exactly the same as those derived in Eq.~\ref{app_E_1_eq_8} using Taylor series expansion of $\omega_2(t_0)$ reproduced below.

\begin{eqnarray*}\label{app_F_3_eq_4}   
[ R(t_0) ]_{t_0=0} = m_0 \\
(\frac{dR(t_0)}{dt_0})_{t_0=0} =   e_0  + n_0  \omega_{20} + 2 \sigma m_0 \\
(\frac{d^2R(t_0)}{dt_0^2})_{t_0=0} =   m_2 + \sigma e_0  +2 \sigma n_0  \omega_{20}   + 2 \sigma^{2} m_0 - m_0 \frac{\omega_{20}^2}{2} 
\end{eqnarray*}
\begin{equation} \end{equation}


\subsection{\label{sec:Appendix_F_4} \textbf{Computation of $m_0, m_1, m_2, n_0, n_1, n_2$} \protect\\  \lowercase{} }

%We assume that the \textbf{even} symmetric function $\omega_2(t_0)$ is analytic in $|t_0| \leq \infty$ and hence can be written in Taylor series as $\omega_2(t_0) = w_{20} + w_{22} t^{2} +... = \displaystyle\sum_{k=0}^{\infty} w_{2(2k)} t_0^{2k}$. 

In Section~\ref{sec:Section_2_2}, we see that $f(t) = e^{-\sigma t_0} E_p(t - t_0) +  e^{\sigma t_0} E_p(t + t_0)$ is \textbf{unchanged} by the substitution $t_0=-t_0$ and hence $\omega_2(t_0)$ is an \textbf{even} function of variable $t_0$. Hence $\frac{d\omega_2(t_0)}{dt_0}$ is an \textbf{odd} function of variable $t_0$. We define the first 2 derivatives of $\omega_2(t_0)$ as $\omega_{2}(0)=\omega_{20}$ and $[\frac{d\omega_2(t_0)}{dt_0}]_{t_0=0}=\omega_{21}=0$ and $[\frac{d^{2}\omega_2(t_0)}{dt_{0}^{2}}]_{t_0=0} = 2 \omega_{22} $.\\

%If $\omega_2(t_0)$ is a Weierstrass type of function which is differentiable nowhere (\textbf{Statement A}), then its first 2 derivatives are not finite. Given that derivatives of Dirac delta functions are defined by a well known equation \href{https://mathworld.wolfram.com/DeltaFunction.html}{(Eq. 17 in link)}, we can proceed with equations below, \textbf{even if} derivatives are not finite and derive further results. In ~\ref{sec:Appendix_D_7}, we show that Statement 1 implies that \textbf{Statement A is false} and that $\omega_2(t_0)$ is a continuous function, which is differentiable at least twice.\\



We can compute $m_0, m_1, m_2, n_0, n_1, n_2$ as follows. We define $[M(t_0)]_{t_0=0}=m_{0}$, $[\frac{dM(t_0)}{dt_0}]_{t_0=0}=m_{1}$, $[\frac{d^{2}M(t_0)}{dt_{0}^{2}}]_{t_0=0} = 2 m_{2} $ and $[N(t_0)]_{t_0=0}=n_{0}$, $[\frac{dN(t_0)}{dt_0}]_{t_0=0}=n_{1}$, $[\frac{d^{2}N(t_0)}{dt_{0}^{2}}]_{t_0=0} = 2 n_{2} $ . Define $\theta(t_0) = \omega_2(t_0) \tau$, we have $\frac{d\theta(t_0)}{dt_0} = \tau \frac{d\omega_2(t_0)}{dt_0}$ and equals $\omega_{21} \tau = 0$ at $t_0=0$.  $\frac{d^{2}\theta(t_0)}{dt_0^{2}} =  \tau \frac{d^{2}\omega(t_0)}{dt_0^{2}}$ and equals $2 \omega_{22} \tau$ at $t_0=0$.


\begin{eqnarray*}\label{app_F_4_eq_1}   
M(t_0) =  \int_{-\infty}^{0}    E_0(\tau)e^{-2 \sigma \tau}  \cos{ (\omega_2(t_0) \tau)} d\tau \\
m_0 =  \int_{-\infty}^{0}    E_0(\tau)e^{-2 \sigma \tau}  \cos{ (\omega_{20} \tau)} d\tau \\
\frac{dM(t_0)}{dt_0} = - \int_{-\infty}^{0}    E_0(\tau)e^{-2 \sigma \tau}  \sin{ (\omega_2(t_0) \tau)} \frac{d\theta(t_0)}{dt_0}  d\tau = - \frac{d\omega_2(t_0)}{dt_0}  \int_{-\infty}^{0}  \tau  E_0(\tau)e^{-2 \sigma \tau}  \sin{ (\omega_2(t_0) \tau)} d\tau  \\
m_1 = (\frac{dM(t_0)}{dt_0})_{t_0=0} = - \omega_{21} \int_{-\infty}^{0}  \tau  E_0(\tau)e^{-2 \sigma \tau}  \sin{ (\omega_{20} \tau)}   d\tau = 0 \\
\frac{d^{2}M(t_0)}{dt_0^{2}} = - \int_{-\infty}^{0}    E_0(\tau)e^{-2 \sigma \tau}  \sin{ (\omega_2(t_0) \tau)}) \frac{d^{2}\theta(t_0)}{dt_0^{2}}  d\tau - \int_{-\infty}^{0}    E_0(\tau)e^{-2 \sigma \tau}  \cos{ (\omega_2(t_0) \tau)}) (\frac{d\theta(t_0)}{dt_0})^{2}  d\tau  \\
m_2 = \frac{1}{2} (\frac{d^{2}M(t_0)}{dt_0^{2}})_{t_0=0} = - \omega_{22} \int_{-\infty}^{0}  \tau  E_0(\tau)e^{-2 \sigma \tau}  \sin{ (\omega_{20} \tau)}) d\tau  
\end{eqnarray*}
\begin{equation} \end{equation}

Similarly, we can compute $n_0, n_1, n_2$ as follows.

\begin{eqnarray*}\label{app_F_4_eq_2}   
N(t_0) = \int_{-\infty}^{0}    E_0(\tau)e^{-2 \sigma \tau}  \sin{ (\omega_2(t_0) \tau)} d\tau \\
n_0 = \int_{-\infty}^{0}    E_0(\tau)e^{-2 \sigma \tau} \sin{ (\omega_{20} \tau)} d\tau \\
\frac{dN(t_0)}{dt_0} =  \int_{-\infty}^{0}    E_0(\tau)e^{-2 \sigma \tau} \cos{ (\omega_2(t_0) \tau)} \frac{d\theta(t_0)}{dt_0}  d\tau = \frac{d\omega_2(t_0)}{dt_0} \int_{-\infty}^{0}  \tau   E_0(\tau)e^{-2 \sigma \tau} \cos{ (\omega_2(t_0) \tau)}   d\tau  \\
n_1 = (\frac{dN(t_0)}{dt_0})_{t_0=0} =   \omega_{21} \int_{-\infty}^{0}  \tau   E_0(\tau)e^{-2 \sigma \tau} \cos{ (\omega_{20} \tau)}  d\tau = 0 \\
\frac{d^{2}N(t_0)}{dt_0^{2}} =  \int_{-\infty}^{0}    E_0(\tau)e^{-2 \sigma \tau}  \cos{ (\omega_2(t_0) \tau)}) \frac{d^{2}\theta(t_0)}{dt_0^{2}}  d\tau - \int_{-\infty}^{0}    E_0(\tau)e^{-2 \sigma \tau} \sin{ (\omega_2(t_0) \tau)}) (\frac{d\theta(t_0)}{dt_0})^{2}  d\tau  \\
n_2 = \frac{1}{2}  (\frac{d^{2}N(t_0)}{dt_0^{2}})_{t_0=0} =   \omega_{22} \int_{-\infty}^{0}  \tau  E_0(\tau)e^{-2 \sigma \tau}  \cos{ (\omega_{20} \tau)}) d\tau   
\end{eqnarray*}
\begin{equation} \end{equation}



%\clearpage
\subsection{\label{sec:Appendix_F_7} \textbf{ Derivation of $f_c(t), f_s(t)$ at $t=-\infty$ } \protect\\  \lowercase{} }


In this section, we compare $(f_{c}(t))_{-\infty} = [\int    E_0(t) e^{-2 \sigma t}  \cos{ ( \omega_2(t_0) t)} dt ]_{t=-\infty} -K_{1c}$ and \\$f_s(t)=  [\int    E_0(t) e^{-2 \sigma t}  \sin{ ( \omega_2(t_0) t)} dt ]_{t=-\infty}  -K_{1s} $  in para 3 of ~\ref{sec:appendix_F} with corresponding version $f_{c0}(t), f_{s0}(t)$ using Taylor series representation of $E_0(t)$ in Eq.~\ref{sec:Section_1_2}  as follows and obtain the values of  $f_c(t), f_s(t)$ at $t=-\infty$. We use the fact that $[f_{c0}(t)]_{-\infty} = [f_{s0}(t)]_{-\infty} = 0$. We copy $f_c(t), f_s(t)$ from Eq.~\ref{app_F_eq_2_0}.\\



%By comparing the \textbf{integrands} in above equation with corresponding integrands in Eq.~\ref{sec:app_s0_eq_0}, we can evaluate the value of integrands in above equation evaluated at the lower limit. Replacing dummy variable $\tau$ by $t$ and we can show that $[f_{c0}(t)]_{-\infty}  = 0, [f_{s0}(t)]_{-\infty}  = 0$. We see that $E_{0}(t) =    \sum_{n=1}^{\infty}  [ 2 \pi^{2} n^{4} e^{4t}    - 3 \pi n^{2}   e^{2t} ]  e^{- \pi n^{2} e^{2t}} e^{\frac{t}{2}}$ is of the order of $O[e^{\frac{5t}{2}}]$ as $\lim_{t \to -\infty}$. Hence $E_p(t)= E_0(t) e^{-\Delta_s t}$ is of the order of $O[e^{(\frac{5}{2}- \Delta_s) t}]$ as $\lim_{t \to -\infty}$, hence the two integrals $f_{c0}(t)]_{-\infty}, f_{s0}(t)]_{-\infty}$ are of the order of $O[e^{(\frac{5}{2}- \Delta_s) t}]$ as $\lim_{t \to -\infty}$, which tends to zero, if $\Delta_s < \frac{5}{2}$. We can write


\begin{eqnarray*}\label{app_F_7_eq_6}   
f_{c0}(t)= \displaystyle\sum\limits_{n,k,r,p}  c_{nkrp}  \frac{ e^{(b_{krp}-2\sigma) t}}{ (b_{krp}^{2} + \omega_2^{2}(t_0))} [(b_{krp}-2\sigma) \cos{ (\omega_2(t_0) t)} +  \omega_2(t_0) \sin{ (\omega_2(t) t)}] \\
f_c(t) =  e_0 ( t  \cos{ ( \omega_2(t_0) t)} + \frac{t^{2}}{!2}  \sin{ ( \omega_2(t_0) t)}\omega_2(t_0)) - 2 \sigma e_0 (( \frac{t^{2}}{2}  \cos{ ( \omega_2(t_0) t)} )  + \frac{t^{3}}{3}  ( ) + ... \\  % e_0 t  -  2 \sigma e_0 \frac{t^2}{2} 
K_{1c}(t_0) + f_c(t) = K_{0c}(t_0) + f_{c0}(t) \\
(f_c(t))_{-\infty} = [f_{c0}(t)]_{-\infty}  + K_{0c}(t_0) - K_{1c}(t_0) =  K_{0c}(t_0) - K_{1c}(t_0) 
\end{eqnarray*}
\begin{equation} \end{equation}

Similarly, we get

\begin{eqnarray*}\label{app_F_7_eq_7}   
f_{s0}(t)= \displaystyle\sum\limits_{n,k,r,p}  c_{nkrp}  \frac{ e^{(b_{krp}-2\sigma) t}}{ (b_{krp}^{2} + \omega_2^{2}(t_0))} [(b_{krp}-2\sigma) \sin{ (\omega_2(t_0) t)} -  \omega_2(t_0) \cos{ (\omega_2(t_0) t)}] \\
f_s(t) = e_0 ( t  \sin{ ( \omega_2(t_0) t)} - \frac{t^{2}}{!2}  \cos{ ( \omega_2(t_0) t)}\omega_2(t_0)) - 2 \sigma e_0 (( \frac{t^{2}}{2}  \sin{ ( \omega_2(t_0) t)} )   + \frac{t^{3}}{3}  ( ) + ...  \\ %(e_0 \omega_{20} \frac{( t)^2}{2}  
K_{1s}(t_0) + f_s(t) = K_{0s}(t_0) + f_{s0}(t) \\
(f_s(t))_{-\infty} = [f_{s0}(t)]_{-\infty}  + K_{0s}(t_0) - K_{1s}(t_0) =  K_{0s}(t_0) - K_{1s}(t_0) 
\end{eqnarray*}
\begin{equation} \end{equation}




We can evaluate integration constants  $K_{0c}(t_0), K_{0s}(t_0)$, $K_{1c}(t_0), K_{1s}(t_0)$  by comparing above equations for $f_{c0}(t)$ and $f_{c}(t)$, at $t = 0$ and similarly for $f_{s0}(t)$ and $f_{s}(t)$, at $t = 0$. We  see that $(f_{c}(t))_{t=0} = (f_{s}(t))_{t=0} = 0$.

\begin{eqnarray*}\label{app_F_7_eq_8}   
(f_c(t))_{-\infty}  = K_{0c}(t_0) - K_{1c}(t_0) = (f_{c}(t))_{t=0} - (f_{c0}(t))_{t=0} =  - (f_{c0}(t))_{t=0} = -\displaystyle\sum\limits_{n,k,r,p}  c_{nkrp} \frac{(b_{krp}-2\sigma)}{ ((b_{krp}-2\sigma^{2}) + \omega_2^{2}(t_0))}\\=  -\int_{-\infty}^{0}    E_0(\tau)  e^{-2\sigma \tau} \cos{ (\omega_2(t_0) \tau)} d\tau \\
(f_s(t))_{-\infty} = K_{0s}(t_0) - K_{1s}(t_0) = (f_{s}(t))_{t=0} - (f_{s0}(t))_{t=0} =  - (f_{s0}(t))_{t=0} = \displaystyle\sum\limits_{n,k,r,p}  c_{nkrp} \frac{\omega_2(t_0)}{ ((b_{krp}-2\sigma)^{2} + \omega_2^{2}(t_0))} \\= -\int_{-\infty}^{0}    E_0(\tau)e^{-2 \sigma \tau}  \sin{ (\omega_2(t_0) \tau)} d\tau\\
(f_c(t))_{-\infty}  =  -\int_{-\infty}^{0}    E_0(\tau) e^{- 2 \sigma \tau} \cos{ (\omega_2(t_0) \tau)} d\tau = -M(t_0)  \\
(f_s(t))_{-\infty} =  -\int_{-\infty}^{0}    E_0(\tau) e^{-2 \sigma \tau}  \sin{ (\omega_2(t_0) \tau)} d\tau = -N(t_0) 
\end{eqnarray*}
\begin{equation} \end{equation}
\\
 



\clearpage



\clearpage
\section{\label{sec:Appendix_D_5} \textbf{ On the zeros of a related function $G(\omega)$ } \protect\\  \lowercase{} }

\textbf{Statement 1}: Let us assume that Riemann's Xi function $\xi(\frac{1}{2} + \sigma + i \omega)= E_{p\omega}(\omega)$ has a zero at $\omega = \omega_{0}$ where $\omega_{0}$ is real and finite and $0 < |\sigma| < \frac{1}{2}$, corresponding to the critical strip excluding the critical line.  \\

Let us consider a new function  $g(t) = E_p(t) e^{-\sigma t}  u(-t) + E_p(t) e^{\sigma t}  u(t)  $ where $g(t)$ is a real function of variable $t$ and $u(t)$ is Heaviside unit step function and $0 < \sigma < \frac{1}{2}$. We can see that $g(t) h(t) = E_p(t)$ where $h(t)=  e^{ \sigma t} u(-t) + e^{ - \sigma t} u(t) $. \\

We can show that $E_p(t), h(t), g(t)$ are real absolutely integrable functions and go to zero as $t \to \pm \infty$. Hence their respective Fourier transforms given by $E_{p\omega}(\omega), H(\omega), G(\omega)$ are finite for $|\omega| \leq \infty$ and go to zero as $|\omega| \to \infty$, as per Riemann Lebesgue Lemma \href{https://en.wikipedia.org/wiki/Riemann-Lebesgue\_lemma}{(link)}. This is shown in detail in ~\ref{sec:appendix_C_1}.\\


If we take the Fourier transform of the equation $g(t)  h(t) = E_p(t)$, we get $ \frac{1}{2 \pi}  [ G(\omega) \ast H(\omega)] = E_{p\omega}(\omega)$ as per convolution theorem \href{https://mathworld.wolfram.com/ConvolutionTheorem.html}{(link)}, where $\ast$ denotes \textbf{convolution} operation given by $E_{p\omega}(\omega) =  \frac{1}{2 \pi} \int_{-\infty}^{\infty} G(\omega') H(\omega - \omega') d\omega'$ and $H(\omega)=  [ \frac{1}{  \sigma - i \omega} +  \frac{1}{  \sigma + i \omega}   ]  = \frac{2 \sigma}{(\sigma^{2} + \omega^{2})} $ is the Fourier transform of the function $h(t)$ and $G(\omega)=  G_{R}(\omega) + i G_{I}(\omega)$ is the Fourier transform of the function $g(t)$. This is shown in detail in ~\ref{sec:appendix_I_1}. \\

We can write $g(t)= g_{even}(t) + g_{odd}(t)$ where $g_{even}(t)$ is an even function and $g_{odd}(t)$ is an odd function of variable $t$. If Statement 1 is true, then the \textbf{real} part of the Fourier transform of the \textbf{even function} $g_{even}(t)=\frac{1}{2} [g(t)+g(-t) ] $  given by $G_{R}(\omega)$ must have \textbf{at least one zero} at $\omega = \omega_{1} \neq 0$ where $\omega_{1}$ is real and finite and can be different from $\omega_0$ in general. We call this \textbf{Statement 2}. \\

Because $H(\omega) = \frac{2 \sigma}{(\sigma^{2} + \omega^{2})}$ is real and does not have zeros for any finite value of $\omega$, \textbf{if} $G_{R}(\omega)$ does not have at least one zero for some $\omega  = \omega_{1} \neq 0$, \textbf{then} the \textbf{real part} of $E_{p\omega}(\omega)$ given by $E_{R}(\omega)= \frac{1}{2 \pi}  [ G_{R}(\omega) \ast H(\omega)]$, obtained by the convolution of $H(\omega)$ and $G_{R}(\omega)$, \textbf{cannot} possibly have zeros for any non-zero finite value of $\omega$, which goes against \textbf{Statement 1}. This is shown in detail in Lemma 1. \\


\textbf{Lemma 1:} If Riemann's Xi function $\xi(\frac{1}{2} + \sigma + i \omega)= E_{p\omega}(\omega)$ has a zero at $\omega = \omega_{0} \neq 0$ where $\omega_{0}$ is real and finite, then the \textbf{real} part of the Fourier transform of the \textbf{even function} $g_{even}(t)=\frac{1}{2} [g(t)+g(-t) ] $  given by $G_{R}(\omega)$  must have \textbf{at least one zero} at $\omega = \omega_{1} \neq 0$, where $\omega_{1}$ is real and finite, where $g(t) h(t) = E_p(t)$ and $h(t)= e^{ \sigma t} u(-t) + e^{ - \sigma t} u(t) $ and $0 < \sigma < \frac{1}{2}$.\\

\textbf{Proof}: If $E_{p\omega}(\omega)$ has a zero at finite $\omega = \omega_{0} \neq 0$ to satisfy Statement 1, then its real part given by $E_{R}(\omega)$ also has a zero at the same location $\omega = \omega_{0} \neq 0$.\\

Let us consider the case where $G_{R}(\omega)$ \textbf{does not} have at least one zero for finite $\omega = \omega_{1}  \neq 0$ and show that $E_{R}(\omega)$ does not have at least one zero at finite $\omega \neq 0$ for this case, which \textbf{contradicts} Statement 1.  Given that $H(\omega)$ is real, we can write the convolution theorem only for the real parts as follows.
 
\begin{equation} \label{app_D_5_eq_1}   
E_{R}(\omega) = \frac{1}{2 \pi}  \int_{-\infty}^{\infty} G_R(\omega') H(\omega - \omega') d\omega' 
\end{equation}

We can show that the above integral converges for all $|\omega| \leq \infty$, given that  $G(\omega)$ and $H(\omega)$ have fall-off rate of $\frac{1}{\omega^2}$ as $|\omega| \to \infty$ because the first derivatives of $g(t)$ and $h(t)$ are discontinuous at $t=0$.(~\ref{sec:appendix_C_2})\\




We substitute $H(\omega) = \frac{2 \sigma}{(\sigma^{2} + \omega^{2})}$ in Eq.~\ref{app_D_5_eq_1}  and we get

\begin{equation} \label{app_D_51_eq_1_1}   
E_{R}(\omega) = \frac{\sigma}{\pi}  \int_{-\infty}^{\infty} G_R(\omega') \frac{1}{(\sigma^{2} + (\omega - \omega')^{2})}  d\omega'
\end{equation}

We can split the integral in Eq.~\ref{app_D_51_eq_1_1} as follows.

\begin{eqnarray*} \label{app_D_5_eq_1_2}   
E_{R}(\omega)  = \frac{\sigma}{ \pi}  [ \int_{-\infty}^{0} G_R(\omega') \frac{1}{(\sigma^{2} + (\omega - \omega')^{2})}  d\omega' + \int_{0}^{\infty} G_R(\omega') \frac{1}{(\sigma^{2} + (\omega - \omega')^{2})}  d\omega' ]
\end{eqnarray*} 
\begin{equation} \end{equation}


We see that $G_{R}(-\omega)= G_{R}(\omega)$ because $g(t)$ is a real function (~\ref{sec:appendix_I_2}). We can substitute $\omega' = -\omega''$ in the first integral in Eq.~\ref{app_D_5_eq_1_2} and substituting $\omega'' = \omega'$ in the result, we can write as follows.



\begin{eqnarray*} \label{app_D_5_eq_2}   
E_{R}(\omega) = \frac{\sigma}{\pi}   \int_{0}^{\infty} G_R(\omega') [  \frac{1}{(\sigma^{2} + (\omega - \omega')^{2})} + \frac{1}{(\sigma^{2} + (\omega + \omega')^{2})} ]  d\omega'  
\end{eqnarray*} 
  \begin{equation}\end{equation}

In ~\ref{sec:appendix_C_1} last paragraph, it is shown that $G(\omega)$ is finite for $|\omega| \leq \infty$ and goes to zero as $|\omega| \to \infty$. We can see that for $\omega' = 0$ and $\omega'=\infty$, the integrand in Eq.~\ref{sec_2_1_eq_2} is zero. For finite $\omega > 0$, and $0 < \omega' < \infty$, we can see that the term $ \frac{1}{(\sigma^{2} + (\omega - \omega')^{2})} + \frac{1}{(\sigma^{2} + (\omega + \omega')^{2})} > 0$. \\



$\bullet$ \textbf{\textbf{Case 1:} $G_R(\omega') > 0$ for all finite $\omega' > 0$ } \\

We see that $E_{R}(\omega) > 0$ for all finite $\omega > 0$. We see that $E_{R}(-\omega)= E_{R}(\omega)$ because $E_p(t)$ is a real function (~\ref{sec:appendix_I_2}). Hence $E_{R}(\omega) > 0$ for all finite $\omega < 0$.\\

 This \textbf{contradicts} Statement 1 which requires $E_{R}(\omega)$ to have at least one zero at finite $\omega \neq 0$ because we showed that $\omega_0 \neq 0$ in \textbf{Section~\ref{sec:Section_2}} paragraph 5. Therefore $G_{R}(\omega')$ must have\textbf{ at least one zero} at $\omega' =  \omega_1 \neq 0$ , where $\omega_{1}$ is real and finite. \\


$\bullet$ \textbf{\textbf{Case 2:} $G_R(\omega') < 0$ for all finite  $\omega' > 0$ } \\

We see that  $E_{R}(\omega) < 0$ for all finite $\omega > 0$. We see that $E_{R}(-\omega)= E_{R}(\omega)$ because $E_p(t)$ is a real function (~\ref{sec:appendix_I_2}). Hence $E_{R}(\omega) < 0$ for all finite $\omega < 0$.\\


This \textbf{contradicts} Statement 1 which requires $E_{R}(\omega)$ to have at least one zero at finite $\omega \neq 0$. Therefore $G_{R}(\omega')$ must have\textbf{ at least one zero} at $\omega' =  \omega_1 \neq 0$ , where $\omega_{1}$ is real and finite. \\



We have shown that, $G_{R}(\omega)$ must have\textbf{ at least one zero} at finite $\omega =  \omega_1 \neq 0$ to satisfy \textbf{Statement 1}. We call this \textbf{Statement 2}. 



\clearpage
\section{\label{sec:Appendix_D_6} \textbf{ On the zeros of a related function $G(\omega)$ Full version } \protect\\  \lowercase{} }

\textbf{Statement 1}: Let us assume that Riemann's Xi function $\xi(\frac{1}{2} + \sigma + i \omega)= E_{p\omega}(\omega)$ has a zero at $\omega = \omega_{0}$ where $\omega_{0}$ is real and finite and $0 < |\sigma| < \frac{1}{2}$, corresponding to the critical strip excluding the critical line.  \\

Let us consider an even function  $g(t) = f(t) e^{-\sigma t}  u(-t) + f(-t)  e^{\sigma t}  u(t)  $ where $g(t)$ is a real function of variable $t$, $f(t)=[ e^{-\sigma t_0} E_p(t - t_0) +  e^{\sigma t_0} E_p(t + t_0) ]$ and $u(t)$ is Heaviside unit step function and $0 < \sigma < \frac{1}{2}$. We can see that $g(t) h(t) = f(t)$ where $h(t)=  e^{ \sigma t} u(-t) + e^{ - 3 \sigma t} u(t) $ as shown in Section~\ref{sec:Section_2_1}. \\

We can show that $E_p(t), h(t), g(t)$ are real absolutely integrable functions and go to zero as $t \to \pm \infty$. Hence their respective Fourier transforms given by $E_{p\omega}(\omega), H(\omega), G(\omega)$ are finite for $|\omega| \leq \infty$ and go to zero as $|\omega| \to \infty$, as per Riemann Lebesgue Lemma \href{https://en.wikipedia.org/wiki/Riemann-Lebesgue\_lemma}{(link)}. This is shown in detail in ~\ref{sec:appendix_C_1}.\\


If we take the Fourier transform of the equation $g(t)  h(t) = f(t)$, we get $ \frac{1}{2 \pi}  [ G(\omega) \ast H(\omega)] = F(\omega)$ as per convolution theorem \href{https://mathworld.wolfram.com/ConvolutionTheorem.html}{(link)}, where $\ast$ denotes \textbf{convolution} operation given by $F(\omega) =  \frac{1}{2 \pi} \int_{-\infty}^{\infty} G(\omega') H(\omega - \omega') d\omega'$ and $H(\omega)=  H_{R}(\omega) + i H_{I}(\omega) = [ \frac{\sigma}{(\sigma^{2} + \omega^{2})} + \frac{3 \sigma}{(9 \sigma^{2} + \omega^{2})} ] + i \omega [ \frac{1}{(\sigma^{2} - \omega^{2})} - \frac{1}{(9 \sigma^{2} + \omega^{2})}  ]  $ is the Fourier transform of the function $h(t)$ and $G(\omega)=  G_{R}(\omega) $ is the Fourier transform of the function $g(t)$. This is shown in detail in ~\ref{sec:appendix_I_1}. \\


If Statement 1 is true, then the \textbf{real} part of the Fourier transform of the \textbf{even function} $g(t)$  given by $G_{R}(\omega)$ must have \textbf{at least one zero} at $\omega = \omega_{2}(t_0) \neq 0$ for every value of $t_0$, where $\omega_{2}(t_0)$ is real and finite and can be different from $\omega_0$ in general. We call this \textbf{Statement 2}. \\

Because $H_R(\omega) = \frac{\sigma}{(\sigma^{2} + \omega^{2})} + \frac{3 \sigma}{(9 \sigma^{2} + \omega^{2})}$ is real and does not have zeros for any finite value of $\omega$, \textbf{if} $G_{R}(\omega)$ does not have at least one zero for some $\omega  = \omega_{2}(t_0) \neq 0$, \textbf{then} the \textbf{real part} of $F(\omega)$ given by $F_{R}(\omega)= \frac{1}{2 \pi}  [ G_{R}(\omega) \ast H_R(\omega)]$, obtained by the convolution of $H_R(\omega)$ and $G_{R}(\omega)$, \textbf{cannot} possibly have zeros for any non-zero finite value of $\omega$, which goes against \textbf{Statement 1}. This is shown in detail in Lemma 1. \\


\textbf{Lemma 1:} If Riemann's Xi function $\xi(\frac{1}{2} + \sigma + i \omega)= E_{p\omega}(\omega)$ has a zero at $\omega = \omega_{0} \neq 0$ where $\omega_{0}$ is real and finite, then the \textbf{real} part of the Fourier transform of the \textbf{even function} $g(t) $  given by $G_{R}(\omega)$  must have \textbf{at least one zero} at $\omega = \omega_{2}(t_0) \neq 0$, for every value of $t_0$, where $\omega_{2}(t_0)$ is real and finite, where $g(t) h(t) = f(t)$,  $f(t)=[ e^{-\sigma t_0} E_p(t - t_0) +  e^{\sigma t_0} E_p(t + t_0) ]$ and $h(t)= e^{ \sigma t} u(-t) + e^{ -3 \sigma t} u(t) $ and $0 < \sigma < \frac{1}{2}$.\\

\textbf{Proof}: If $E_{p\omega}(\omega)$ has a zero at finite $\omega = \omega_{0} \neq 0$ to satisfy Statement 1, then its real part given by $E_{R}(\omega)$ also has a zero at the same location $\omega = \omega_{0} \neq 0$.\\

Let us consider the case where $G_{R}(\omega)$ \textbf{does not} have at least one zero for finite $\omega = \omega_{2}(t_0)  \neq 0$ and show that $E_{R}(\omega)$ does not have at least one zero at finite $\omega \neq 0$ for this case, which \textbf{contradicts} Statement 1.  Given that $H_R(\omega)$ is real, we can write the convolution theorem only for the real parts as follows.
 
\begin{equation} \label{app_D_6_eq_1}   
E_{R}(\omega) = \frac{1}{2 \pi}  \int_{-\infty}^{\infty} G_R(\omega') H_R(\omega - \omega') d\omega' 
\end{equation}

We can show that the above integral converges for all $|\omega| \leq \infty$, given that  $G(\omega)$ and $H_R(\omega)$ have fall-off rate of $\frac{1}{\omega^2}$ as $|\omega| \to \infty$ because the first derivatives of $g(t)$ and $h(t)$ are discontinuous at $t=0$.(~\ref{sec:appendix_C_2})\\




We substitute $H_R(\omega) = \frac{\sigma}{(\sigma^{2} + \omega^{2})} + \frac{3 \sigma}{(9 \sigma^{2} + \omega^{2})}$ in Eq.~\ref{app_D_6_eq_1} and we get

\begin{equation} \label{app_D_6_eq_1_1}   
E_{R}(\omega) = \frac{\sigma}{2 \pi}  \int_{-\infty}^{\infty} G_R(\omega') [ \frac{1}{(\sigma^{2} + (\omega - \omega')^{2})} + \frac{3}{(9\sigma^{2} + (\omega - \omega')^{2})} ] d\omega'
\end{equation}

We can split the integral in Eq.~\ref{app_D_6_eq_1_1} as follows.

\begin{eqnarray*} \label{app_D_6_eq_1_2}   
E_{R}(\omega)  = \frac{\sigma}{ 2\pi}  [ \int_{-\infty}^{0} G_R(\omega') [ \frac{1}{(\sigma^{2} + (\omega - \omega')^{2})} + \frac{3}{(9\sigma^{2} + (\omega - \omega')^{2})} ] d\omega'\\ + \int_{0}^{\infty} G_R(\omega')[ \frac{1}{(\sigma^{2} + (\omega - \omega')^{2})} + \frac{3}{(9\sigma^{2} + (\omega - \omega')^{2})} ] d\omega' ]
\end{eqnarray*} 
\begin{equation} \end{equation}


We see that $G_{R}(-\omega)= G_{R}(\omega)$ because $g(t)$ is a real function (~\ref{sec:appendix_I_2}). We can substitute $\omega' = -\omega''$ in the first integral in Eq.~\ref{app_D_6_eq_1_2} and substituting $\omega'' = \omega'$ in the result, we can write as follows.



\begin{eqnarray*} \label{app_D_6_eq_2}   
E_{R}(\omega) = \frac{\sigma}{2\pi}   \int_{0}^{\infty} G_R(\omega') [  (\frac{1}{(\sigma^{2} + (\omega - \omega')^{2})} + \frac{1}{(\sigma^{2} + (\omega + \omega')^{2})} )+ (\frac{3}{(9\sigma^{2} + (\omega - \omega')^{2})} + \frac{3}{(9\sigma^{2} + (\omega + \omega')^{2})}) ]  d\omega'  
\end{eqnarray*} 
  \begin{equation}\end{equation}

In ~\ref{sec:appendix_C_1} last paragraph, it is shown that $G(\omega)$ is finite for $|\omega| \leq \infty$ and goes to zero as $|\omega| \to \infty$. We can see that for $\omega' = 0$ and $\omega'=\infty$, the integrand in Eq.~\ref{app_D_6_eq_2} is zero. For finite $\omega > 0$, and $0 < \omega' < \infty$, we can see that the term $[  (\frac{1}{(\sigma^{2} + (\omega - \omega')^{2})} + \frac{1}{(\sigma^{2} + (\omega + \omega')^{2})} )+ (\frac{3}{(9\sigma^{2} + (\omega - \omega')^{2})} + \frac{3}{(9\sigma^{2} + (\omega + \omega')^{2})}) ]  > 0$. \\



$\bullet$ \textbf{\textbf{Case 1:} $G_R(\omega') > 0$ for all finite $\omega' > 0$ } \\

We see that $E_{R}(\omega) > 0$ for all finite $\omega > 0$. We see that $E_{R}(-\omega)= E_{R}(\omega)$ because $E_p(t)$ is a real function (~\ref{sec:appendix_I_2}). Hence $E_{R}(\omega) > 0$ for all finite $\omega < 0$.\\

 This \textbf{contradicts} Statement 1 which requires $E_{R}(\omega)$ to have at least one zero at finite $\omega \neq 0$ because we showed that $\omega_0 \neq 0$ in \textbf{Section~\ref{sec:Section_2}} paragraph 5. Therefore $G_{R}(\omega')$ must have\textbf{ at least one zero} at $\omega' =  \omega_1 \neq 0$ , where $\omega_{2}(t_0)$ is real and finite. \\


$\bullet$ \textbf{\textbf{Case 2:} $G_R(\omega') < 0$ for all finite  $\omega' > 0$ } \\

We see that  $E_{R}(\omega) < 0$ for all finite $\omega > 0$. We see that $E_{R}(-\omega)= E_{R}(\omega)$ because $E_p(t)$ is a real function (~\ref{sec:appendix_I_2}). Hence $E_{R}(\omega) < 0$ for all finite $\omega < 0$.\\


This \textbf{contradicts} Statement 1 which requires $E_{R}(\omega)$ to have at least one zero at finite $\omega \neq 0$. Therefore $G_{R}(\omega')$ must have\textbf{ at least one zero} at $\omega' =  \omega_1 \neq 0$ , where $\omega_{2}(t_0)$ is real and finite. \\



We have shown that, $G_{R}(\omega)$ must have\textbf{ at least one zero} at finite $\omega =  \omega_1 \neq 0$ to satisfy \textbf{Statement 1}. We call this \textbf{Statement 2}. 


\clearpage
\subsection{\label{Appendix_D_7} \textbf{ Method 1: Asymptotic Fall off rate argument. Previous Version} \protect\\  \lowercase{} }

This method \textbf{does not} require differentability of $\omega_2(t_0)$ and is \textbf{independent} of Method 2 in Section~\ref{sec:Section_A_2}.\\

In Section~\ref{sec:Section_3_1}, we show that $\lim_{t_0 \to \infty} g(t)$ is an \textbf{analytic} function, with the \textbf{magnitude} of the step discontinuity at $t=0$ \textbf{decreasing to zero}, and its Fourier transform is an analytic function with \textbf{isolated zeros} and each isolated zero has a single value, as $\lim_{t_0 \to \infty}$. \\ %Hence $\lim_{t_0 \to \infty} \omega_{2}(t_0) = \omega_z \neq 0$ which is a constant. \\

In Section~\ref{sec:Section_A_1_1}, we show that $\lim_{t_0 \to \infty} \omega_{2}(t_0) = \omega_z $ is a constant and we \textbf{rule  out} the pathological case of $\omega_{2}(t_0)$ which is discontinuous everywhere and/or ill-defined. It is shown that the integrals $I_1(t_0) =  \int_{-\infty}^{t_0}     E_{0}( \tau)  e^{ -2 \sigma \tau}  \cos{ (\omega_2(t_0) \tau)} d\tau$ and $I_2(t_0) =  \int_{-\infty}^{t_0}  E_{0}( \tau)  e^{ -2 \sigma \tau}  \sin{ (\omega_2(t_0) \tau)} d\tau$ in Eq.~\ref{sec_a_1_eq_1} \textbf{converge} as $\lim_{t_0 \to \infty}$.
\\% Hence $\lim_{t_0 \to \infty} \omega_{2}(t_0) = \omega_z $ is a constant. \\

As  $\lim_{t_0 \to \infty}$, we can compute $S(t_0)$ in Eq.~\ref{sec_1_eq_7} as follows. The expression for $R(-t_0)$ goes to zero as $\lim_{t_0 \to \infty}$, due to the term $e^{-2 \sigma t_0}$. In the equation for $R(t_0)$, the term $\lim_{t_0 \to \infty} e^{2 \sigma t_0} =  \infty$. Hence we require \\$\lim_{t_0 \to \infty} \cos{ (\omega_z  t_0)} \int_{-\infty}^{t_0}    E_0(\tau)  e^{ - 2 \sigma \tau}  \cos{ ( \omega_z  \tau)} d\tau + \sin{ (\omega_z  t_0)}  \int_{-\infty}^{t_0}  E_0(\tau)  e^{ - 2 \sigma \tau} \sin{ (\omega_z  \tau)} d\tau   = 0$. \\We use $\lim_{t_0 \to \infty} \omega_{2}(t_0) = \omega_z $ and write as follows.

\begin{eqnarray*}\label{app_a_1_eq_1}   
\lim_{t_0 \to \infty}  S(t_0)  =  \lim_{t_0 \to \infty} e^{2 \sigma t_0} [ \cos{ (\omega_2(t_0) t_0)} \int_{-\infty}^{t_0}    E_0(\tau)  e^{ - 2 \sigma \tau}  \cos{ ( \omega_2(t_0) \tau)} d\tau + \sin{ (\omega_2(t_0) t_0)}  \int_{-\infty}^{t_0}  E_0(\tau)  e^{ - 2 \sigma \tau} \sin{ (\omega_2(t_0) \tau)} d\tau ] = 0 \\
\lim_{t_0 \to \infty}  \cos{ (\omega_z  t_0)} \int_{-\infty}^{t_0}    E_0(\tau)  e^{ - 2 \sigma \tau}  \cos{ ( \omega_z  \tau)} d\tau + \sin{ (\omega_z  t_0)}  \int_{-\infty}^{t_0}  E_0(\tau)  e^{ - 2 \sigma \tau} \sin{ (\omega_z  \tau)} d\tau   = 0
\end{eqnarray*}
\begin{equation} \end{equation}

We define $I_1(t_0) =  \int_{-\infty}^{t_0}     E_{0}( \tau)  e^{ -2 \sigma \tau}  \cos{ (\omega_2(t_0) \tau)} d\tau$ and $I_2(t_0) =  \int_{-\infty}^{t_0}  E_{0}( \tau)  e^{ -2 \sigma \tau}  \sin{ (\omega_2(t_0) \tau)} d\tau$ in Eq.~\ref{app_a_1_eq_1}  and note that $\lim_{t_0 \to \infty} I_1(t_0)$ and $\lim_{t_0 \to \infty} I_2(t_0)$ tend to a constant, which is finite and determinate, given that $\lim_{t_0 \to \infty} \omega_{2}(t_0) = \omega_z$. We see that the terms $I_1(t_0)$ and $I_2(t_0)$ have an \textbf{asymptotic fall-off }rate of $e^{-K t_0}$, as  $\lim_{t_0 \to \infty}$, where $K > 2 \sigma$, to satisfy the equation $S(t_0)= R(t_0) + R(-t_0)=0$. Hence we can write a \textbf{new equation}  by interchanging $I_1(t_0)$ and $I_2(t_0)$ in Eq.~\ref{app_a_1_eq_1} as follows.


\begin{eqnarray*}\label{app_a_1_eq_2}   
\lim_{t_0 \to \infty}  \cos{ (\omega_z  t_0)} \int_{-\infty}^{t_0}    E_0(\tau)  e^{ - 2 \sigma \tau}  \sin{ ( \omega_z  \tau)} d\tau - \sin{ (\omega_z  t_0)}  \int_{-\infty}^{t_0}  E_0(\tau)  e^{ - 2 \sigma \tau} \cos{ (\omega_z  \tau)} d\tau   = 0
\end{eqnarray*}
\begin{equation} \end{equation}

We use $I_1(t_0) =  \int_{-\infty}^{t_0}     E_{0}( \tau)  e^{ -2 \sigma \tau}  \cos{ (\omega_z \tau)} d\tau$ and $I_2(t_0) =  \int_{-\infty}^{t_0}  E_{0}( \tau)  e^{ -2 \sigma \tau}  \sin{ (\omega_z \tau)} d\tau$,  we can write Eq.~\ref{app_a_1_eq_1} and Eq.~\ref{app_a_1_eq_2} as follows. 


\begin{eqnarray*}\label{app_a_1_eq_3}   
\lim_{t_0 \to \infty}  \cos{ ( \omega_z t_0)}  I_1(t_0) + \lim_{t_0 \to \infty}  \sin{ ( \omega_z t_0)}  I_2(t_0) = 0 \\
\lim_{t_0 \to \infty}  \cos{ ( \omega_z t_0)}  I_2(t_0) - \lim_{t_0 \to \infty}  \sin{ ( \omega_z t_0)}  I_1(t_0)  = 0 \\
\lim_{t_0 \to \infty} \frac{I_2(t_0)}{I_1(t_0)} = \lim_{t_0 \to \infty} \frac{ \sin{ ( \omega_z t_0)}}{ \cos{ ( \omega_z t_0)}} = \lim_{t_0 \to \infty} -\frac{I_1(t_0)}{I_2(t_0)} 
\end{eqnarray*}
\begin{equation} \end{equation}  

For the general case of $\lim_{t_0 \to \infty} \frac{ \sin{ ( \omega_z t_0)}}{ \cos{ ( \omega_z t_0)}} \neq 0, \pm \infty$, we get $\lim_{t_0 \to \infty} I_1(t_0)^{2} + I_2(t_0)^{2} = 0$. This implies that $\lim_{t_0 \to \infty} I_1(t_0)= \lim_{t_0 \to \infty}I_2(t_0) = 0$ and $\int_{-\infty}^{\infty}     E_0(\tau) e^{-2 \sigma \tau} e^{-i  \omega_z \tau} d\tau = 0$. \\

We started with \textbf{Statement 1} that the Fourier Transform of the function $E_p(t) = E_0(t) e^{-\sigma t} $ has a zero at $\omega = \omega_{0}$ which means that $\int_{-\infty}^{\infty}    E_0(\tau) e^{- \sigma \tau} e^{-i \omega_0 \tau} d\tau = 0$ and we derived the result that $\int_{-\infty}^{\infty}    E_0(\tau) e^{-2 \sigma \tau} e^{-i  \omega_z \tau} d\tau = 0$.\\

Now we can repeat the steps in Section 2, starting with the new result that $\int_{-\infty}^{\infty}    E_0(\tau) e^{-2 \sigma \tau} e^{-i \omega_z \tau} d\tau = 0$ and $\sigma$ replaced by $2 \sigma$ and derive the next result that $\int_{-\infty}^{\infty}    E_0(\tau) e^{-4 \sigma \tau} e^{-i \omega_{(z1)} \tau} d\tau = 0$.\\

We can repeat above steps N times till $(2^{N+1} \sigma) > \frac{1}{2}$ and get the result $\int_{-\infty}^{\infty}    E_0(\tau) e^{-(2^{N+1} \sigma) \tau} e^{-i \omega_{(zN)} \tau} d\tau = 0$. In each iteration $n$, we use $h(t)=  e^{ (2^{N+1} \sigma) t} u(-t) + e^{ - 3*(2^{N+1} \sigma) t} u(t) $, $\omega_2(t_0)$ replaced by $\omega_{2n}(t_0)$ and $\omega_z$ replaced by $\omega_{(zn)}$. We know that  the Fourier Transform of $E_{0}(t) e^{-(2^{N+1} \sigma) t} =    \sum_{n=1}^{\infty}  [ 4 \pi^{2} n^{4} e^{4t}    - 6 \pi n^{2}   e^{2t} ]  e^{- \pi n^{2} e^{2t}} e^{\frac{t}{2}} e^{-(2^{N+1} \sigma) t}$ given by $E_{p\omega N}(\omega)=\xi(\frac{1}{2}+ 2^N \sigma + i \omega)$ \textbf{does not} have a real zero for $(2^{N+1} \sigma) > \frac{1}{2}$ , corresponding to $Re[s] > 1$. \\% for  $0 < |\sigma| < \frac{1}{2}$ Here we use the well known fact that $E_0(t)=E_0(-t)$. \\

We have shown this result for $0 < \sigma < \frac{1}{2}$ and then use the property $\xi(\frac{1}{2} + \sigma + i \omega) = \xi(\frac{1}{2} - \sigma - i \omega)$ to show the result for $-\frac{1}{2} < \sigma < 0$. Hence we have produced a \textbf{contradiction} of  \textbf{Statement 1} that the Fourier Transform of the function $E_p(t) = E_0(t) e^{-\sigma t} $ has a zero at $\omega = \omega_{0}$ for  $0 < |\sigma| < \frac{1}{2}$.



\end{document}



